\chapter{Conclusões e Trabalhos Futuros}\label{cap:conclusion}

Este trabalho investigou o uso de redes neurais artificiais, como algoritmos SVM e MLP, para a estimativa do impacto do risco, através da proposta de uma metodologia para a análise de risco em projetos de \textit{software}, a partir de dados históricos de registros de riscos. 

Os resultados mostraram que as redes neurais artificiais apresentaram resultados promissores, principalmente porque um modelo MLP chamado ``MLPRegressor" obteve os menores erros de previsão. Além disso, alguns resultados apresentaram informações importantes: a SMC apresentou resultados piores comparado com o modelo de linha de base, o MRLM; o modelo ANFIS apresentou resultados piores comparado com a "MLPRegressor", porém é mais preciso 52\% na média comparado com o último modelo; é difícil melhorar ainda mais o desempenho da Análise PERT e da SMC já que todas as medidas foram tomadas para que com as informações disponíveis esses modelos pudessem apresentar os melhores resultados.

Além disso, uma pesquisa rápida realizada com duzentos profissionais (membros da comunidade de prática de gerenciamento de risco do PMI, dos mais diversos campos de aplicação e setores), os quais dez responderam às perguntas feitas, informaram que, no geral, entre 0\% e 5\% é um intervalo de erros ideal de previsão de impacto de risco; 5\% e 10\% é um intervalo aceitável de erro na estimativa e 10\% e 15\% é um intervalo indesejado. Portanto, como o ``MLPRegressor" apresentou alguns valores no primeiro intervalo, REMQ médio de 0.05168 ou 5,168\%, é possível afirmar que esse modelo apresenta resultados satisfatórios.

% Listar os problemas e dificuldades encontradas
Algumas dificuldades foram encontradas nesse estudo, principalmente quanto ao pré-processamento dos dados e a seleção das redes neurais e suas variações para se obter os resultados desejados. Porque não há qualquer trabalho científico utilizando a PERIL para se tomar como base, segundo a revisão bibliográfica feita.

Alguns resultados interessantes foram alcançados porém é preciso melhorar algumas técnicas utilizadas para que os resultados possam ser mais precisos e menores erros de previsão sejam obtidos. Além disso, há uma carga computacional elevada no procedimento de utilização de um algoritmo de otimização para otimizar o desempenho das redes neurais, o que demanda tempo. Como atividades futuras, foram identificadas:
\begin{itemize}
\item Realizar a validação prática dessa metodologia com informações reais para a estimativa e acompanhamento de riscos;
\item Desenvolver uma abordagem eficiente e precisa para quando houver poucas informações sobre os riscos identificados num projeto e nenhum registro de riscos em projetos similares anteriores;
\item Desenvolver uma abordagem inovadora e mais eficiente para a análise qualitativa dos riscos, baseada na classificação da natureza dos riscos;
\item Desenvolver uma técnica para a avaliação de estratégias de mitigação de risco, baseadas no impacto se o risco ocorrer, no esforço para mitigação de risco, nas interações entre os fatores de risco e nos recursos disponíveis para os planos de mitigação;
\item Desenvolver uma metodologia para a avaliação qualitativa, a avaliação quantitativa e planos de contingência de riscos em projetos, do ponto de vista do gerenciamento de portfólio de projetos para o alcance de objetivos estratégicos.
\item Desenvolvimento de um \textit{canvas} para o gerenciamento de riscos em projetos de forma ágil;
\item Implementar outros métodos de validação cruzada.
\end{itemize}

\pagebreak

