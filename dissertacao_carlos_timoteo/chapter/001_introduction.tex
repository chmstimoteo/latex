\chapter{Introdução}\label{cap:introduction}

Quão arriscados são os projetos de \textit{software}? Diversos estudos sobre a efetividade de técnicas de previsão de custo, escopo, cronograma; \textit{surveys} com profissionais em \textit{software} na indústria; e análise de portfólio de projetos foi realizada para responder essa questão \cite{budzier2013double} \cite{flyvbjerg2013quality} \cite{jones1998estimating} \cite{CHAOS2009} \cite{jones2008applied} \cite{molokken2003review}. No entanto, não há um consenso.
%How risky are software projects? Several studies about effectiveness of software cost, scope, schedule estimation techniques; surveys from software professionals in industry; and analysis of projects portfolio have been done to answer this question \cite{budzier2013double}. However, there is not a consensus. 

Todos os projetos envolvem risco. Há sempre pelo menos algum nível de incerteza no resultado de um projeto, independentemente do que o gráfico de Gannt \cite{maylor2001beyond} pareça indicar. Projetos de alta tecnologia são particularmente arriscados, por uma série de motivos. Primeiro, há uma grande variedade de projetos técnicos. Esses projetos tem objetivos e aspectos únicos que os diferem significativamente dos trabalhos anteriores, além de apresentar um ambiente de projetos que evolue rapidamente. Além disso, projetos técnicos são frequentemente ``enxutos", ou seja, desafiados a trabalhar com financiamento, pessoal e equipamentos inadequados. Para piorar a situação, há uma expectativa generalizada que não corresponde à realidade de que por mais rápido que tenha sido o último projeto, o próximo deve ser ainda mais rápido \cite{kendrick2003identifying}.
%Every project involves risk. There is always at least some level of uncertainty in a project's outcome, regardless of what the Gantt chart on the wall seems to imply. High-tech projects are particularly risky, for a number of reasons. First, technical projects are high varied. These projects have unique aspects and objectives that significantly differ from previous work, and the environment for technical projects evolve quickly. In addition, technical projects are frequently "lean", challenged to work with inadequate funding, staff and equipment. To make matters worse, there is a pervasive expectation that however fast the last project may have been, the next one should be even quicker \cite{kendrick2003identifying}.

Projetos que tiveram sucesso geralmente conseguiram isso porque duas das ações tomadas pelos seus líderes foram determinantes. Primeiro, eles reconheceram que alguns dos trabalhos em qualquer projeto, mesmo projetos de alta tecnologia, não são novos. Nos trabalhos por eles desenvolvidos as notas, registros e lições aprendidas em projetos anteriores puderam ser utilizados como um roteiro para identificar, e em muitos casos evitar, muitos problemas potenciais. Segundo, eles planejaram com afinco o trabalho do projeto, especialmente as partes que exigiam inovação, para possibilitar a compreensão dos desafios futuros e antecipar muitos dos riscos \cite{kendrick2003identifying}.
%Projects that succeed generally do so because their leaders do two things well. First, they recognize that a few of the work on any project, even a high-tech project, is not new. For this work, the notes, records, and lessons learned on earlier projects can be a road map for identifying, and in many cases avoiding, many potential problems. Second, they plan project work thoroughly, especially the portions that require innovation, to understand the challenges ahead and to anticipate many of the risks \cite{kendrick2003identifying}.

Alguns benefícios da boa gestão de riscos de projetos de software são apresentados abaixo. Tais fatores podem determinar o sucesso dos projetos \cite{HIGUERAHAIMES1996} \cite{PMBOK2008}:
%Some benefits of good risk management of software projects are presented above. Such factors may determine the success of projects \cite{HIGUERAHAIMES1996} \cite{PMBOK2008}. 

\begin{itemize}
\item redução de custos incorridos com mudanças no software;
\item desenvolvimento de um plano de respostas a eventos inesperados, conhecido como plano de contingência de riscos;
\item previsão da probabilidade da ocorrência de eventos indesejados;
\item seguimento das linhas de base de custo, cronograma e qualidade.
\end{itemize}
%\begin{enumerate}
%\item reduction of costs associated with changes in software;
%\item development of a response plan to unexpected events, i.e., a contingency risk plan; 
%\item prediction of likelihood of undesired events;
%\item tracking the baselines of cost, schedule and quality.
%\end{enumerate}

\section{Motivação}

Em 2009, o CHAOS Report \cite{CHAOS2009} mostrou que 32\% dos projetos de software alcançaram sucesso, foram entregues no prazo, de acordo com o orçamento estabelecido e com os requisitos prometidos; 44\% dos projetos foram desafiados, em que o prazo ou o orçamento ou os requisitos não foram cumpridos; não menos importante, 24\% dos projetos falharam e foram cancelados no Mundo. Isso ocorre devido aos riscos envolvidos nas atividades do projeto e a um gerenciamento de riscos ausente ou deficiente \cite{ISLAM2009}. 
%In 2009, CHAOS Report \cite{CHAOS2009} showed that 32\% of projects achieved success - were delivered on time, on budget and with the promised requirements -; 44\% of the projects were challenged - or schedule or budget or requirements were not fulfilled; not least, 24\% of projects failed and were canceled. That is due to the risks involved in project activities and to a absent or defective software risk management \cite{ISLAM2009}.

Schmidt e outros autores \cite{schmidt2001identifying} notaram que muitos dos projetos de desenvolvimento de \textit{software} terminavam sem alcançar todos os objetivos. Eles mostraram que cerca de 25\% de todos os projetos de software são cancelados e cerca de 80\% de todos os projetos de software ultrapassaram seus orçamentos, excedendo-os em 50\% na média. Paul Bannerman \cite{bannerman2008risk} afirma que pesquisas na indústria sugerem que somente um quarto dos projetos de software tem sucesso imediato, e bilhões de dólares são perdidos anualmente por meio de falhas ou projetos que não cumprem a entrega dos benefícios prometidos. Além disso, o autor mostra evidências de que isso é um assunto global, impactando organizações do setor privado e público \cite{KPMG2005}.
%Schmidt et al. \cite{schmidt2001identifying} have noticed that many software development projects end in failure. They showed that around 25\% of all software projects are canceled outright and as many as 80\% of all software projects run over their budget, exceeding it by 50\% in average. Paul Bannerman \cite{bannerman2008risk} states that industry surveys suggest that only a quarter of software projects succeed outright, and billions of dollars are lost annually through project failures or projects that do not deliver promised benefits. Moreover, the author shows evidences that it's a global issue, impacting private and public sector organizations \cite{KPMG2005}.

A previsão de possíveis eventos a curto, médio e longo prazos muitas vezes é falha. Ao analisar os riscos e as incertezas, os gerentes de projeto comumente confiam na própria intuição, em vez de utilizarem a lógica e uma análise detalhada. No entanto, o pensamento intuitivo é frequentemente alvo de ilusões, que causam erros previsíveis e decisões eventualmente não embasadas. O método para conciliar o efeito dessas ilusões é uma avaliação sistemática dos riscos e esforços na mitigação dos mesmos através de métodos analíticos.
%Predict possible events in short, medium and long term is often failure. By analyzing risks and uncertainties, project managers commonly rely on intuition rather than logic and analysis. However, intuitive thinking is often subject to delusions, causing predictable mental errors and eventually poor decisions. A way to balance the effect of these psychological illusions is a systematic risk analysis and efforts to mitigate them through analytical methods. 

É difícil estimar algo que não pode ser medido. Gerentes de projeto devem quantificar a probabilidade de risco, os resultados, e seu efeito cumulativo em um projeto. Além disso, é importante avaliar as várias opções de mitigação: o custo de cada opção e o tempo necessário para a sua realização \cite{VIRINE2009}.
%It is difficult to manage something that can not be quantified. Project managers should quantify the probability of risk, the impact, and their cumulative effect on a project. Furthermore, it is important to evaluate the various mitigation options: the cost for each option and the required time to perform the mitigation \cite{VIRINE2009}. 

Existe uma dificuldade na interpretação do conceito de risco, principalmente quanto a aplicação desse conhecimento no desenvolvimento e utilização de técnicas eficientes para a análise de risco no gerenciamento de projetos de \textit{software}. A gestão de riscos e incertezas em projetos de software, é fundamental para a disciplina de gerenciamento de projetos. Entretanto, em momentos econômicos de crise torna-se muito mais difícil realizar o gerenciamento de riscos, devido aos custos incorridos.
%Interpreting the concept of risk is a hard task, especially regarding the application of this knowledge in the development and use of efficient procedures for risk analysis in software project management techniques. Managing risk and uncertainty in software projects is critical to project management discipline. However, in times of economic crisis becomes much more difficult to perform risk management, due to the costs incurred.

Esta é uma área de pesquisa que certamente irá continuar evoluindo e, portanto, novas e melhores metodologias para identificação, medição e controle de itens de risco de \textit{software} precisam ser desenvolvidas. Segundo Keshlaf e Riddle \cite{KESHLAFRIDDLE2010}, mesmo que existam muitas abordagens na academia ainda há uma grande lacuna com relação ao que é praticado pelas indústrias de \textit{software}, muitas dessas abordagens não são avaliadas e há uma grande resistência ao que é produzido na academia. 
%Since it is an area of research that is growing, new and better methodologies to identify, measure and control risk items of software need to be developed. Keshlaf and Riddle \cite{KESHLAFRIDDLE2010} conclude that even if there are many approaches there is still a large gap regarding what is practiced by software industries.

\section{Descrição do Problema}

Embora o gerenciamento de risco na gestão de projetos de \textit{software} seja um processo recomendado, sua utilização ainda está aquém das expectativas. Algumas causas disso são o acúmulo de responsabilidades dos gerentes de projetos, a baixa importância atribuída a essa área, a falta de conhecimento em gestão de riscos, os custos envolvidos nas atividades de gestão de risco, a falta de habilidade para lidar com as técnicas e ferramentas específicas. Como consequência, o projeto está sujeito à influência negativa de riscos sem haver um plano de contingência, o que pode ocasionar o fracasso do projeto. Conforme identificado por Kwak e Ibbs \cite{kwak2000calculating}, o gerenciamento de risco é a disciplina menos praticada dentre as diferentes áreas do conhecimento no gerenciamento de projetos. Os autores mencionam que, provavelmente, um motivo é que desenvolvedores de software e gerentes de projetos consideram gerenciar processos e atividades que envolvam incerteza como trabalho e custo extras. De acordo com o \textit{benchmarking} realizado em 2009 pelo \textit{Project Management Institute}, em 20\% dos projetos os seus gerentes não realizam todos os processos de planejamento e em apenas 35\% dos projetos o gerenciamento de riscos é realizado de acordo com uma metodologia formal, estruturada por políticas, procedimentos e formulários. Além disso, 46\% dos gerentes realizam atividades de gerenciamento em tempo parcial.
%Even though risk management in software project management is a healthy process, its adoption is still far from expectations. A few causes are the overloading of responsibilities on project managers, the low importance attributed to the area, the lack of knowledge of risk management, the costs incurred in risk management activities, the lack of technical skill and familiarity with specific tools. Consequently, the project is prone to the negative influence of risks without a contingency plan, which may lead to project failure. Kwak and Ibbs \cite{kwak2000calculating} identified risk management as the least practiced discipline among different project management knowledge areas. The authors mention that, probably, a cause for it is that software developers and project managers perceive managing uncertainty processes and activities as extra work and expense. According to the benchmarking conducted in 2009 by the Project Management Institute, in 20\% of the projects their managers do not perform all the planning processes and in only 35\% of the projects, risk management is conducted according to a formal methodology, structured by policies, procedures and forms. Also, 46\% of managers carry out management activities part a time.

Barry Boehm \cite{BOEHM1991} definiu risco como a possibilidade de perda ou dano. Essa definição pode ser expressa pela fórmula de exposição ao risco. Mesmo que Boehm cite a exposição ao risco como a técnica mais efetiva para a priorização do risco depois de sua análise, Paul Bannerman \cite{bannerman2008risk} considera essa definição limitada e inapropriada. Na teoria clássica da decisão, risco reflete a variação na distribuição de probabilidade de possíveis resultados, seja negativo ou positivo, associado a uma decisão particular. Esse estudo leva em consideração a definição do \textit{Project Management Institute} \cite{PMBOK2008} em que risco em projeto é um evento ou condição específica que, se ocorrer, tem um efeito seja positivo quanto negativo em um ou mais objetivos do projeto, já que alguns autores não consideram o efeito positivo de um risco. Uma definição complementar proposta por Haimes \cite{haimes2011risk} também é considerada, a qual expressa o risco como uma medida da probabilidade e severidade de efeitos adversos.
%Barry Boehm \cite{BOEHM1991} defined risk as the possibility of loss or injury. That definition can be expressed by risk exposure formula. Even Boehm cites risk exposure as the most effective technique for risk priorization after risk analysis, Paul Bannerman \cite{bannerman2008risk} considers this definition limited and unsuitable. In classical decision theory, risk was viewed as reflecting variation in the probability distribution of possible outcomes, whether negative or positive, associated with a particular decision. This study takes into account Project Management Institute \cite{PMBOK2008} definition whereupon project risk is a certain event or condition that, if it occurs, has a positive or negative effect on one or more project objectives. A complementary definition proposed by Haimes \cite{haimes2011risk} is also considered, which express risk as a measure of the probability and severity of adverse effects.

Um fator de risco é uma variável associada com a ocorrência de um evento inesperado. Fatores de risco são correlacionados, não necessariamente causais e, se um deles ocorrer, implicará em um ou mais impactos. De acordo com Haimes \cite{haimes2011risk}, riscos podem comumente surgir como o resultado não apenas de um processo estocástico não percebido ocorrendo no tempo e no espaço, mas também, baseado em fatores de riscos determinísticos. A estimativa do risco pode ser alcançada baseada em informações históricas ou conhecimento de projetos anteriores similares, ou ainda de outra fonte de informação \cite{PMBOK2008}.
%A risk factor is a variable associated with the occurrence of an unexpected event. Risk factors are correlational, not necessarily causal and, if one of them occurs, it may have one or more impacts. According to Haimes \cite{haimes2011risk}, risks can often arise as the result of an underlying stochastic process occurring over time and space, but also, can occur based on deterministic risk factors. Risk estimation can be achieved based on historical information and knowledge from previous similar projects and from other information sources \cite{PMBOK2008}. 

Risco é um conceito que muitos consideram difícil de ser compreendido por envolver duas métricas complexas: uma medida da probabilidade e da severidade de efeitos adversos \cite{Haimes2009}. Uma limitação dessa definição é a dificuldade prática em se estimar a probabilidade e o impacto de diversos fatores de risco, especialmente em projetos de \textit{software}. As probabilidades somente podem ser definidas de um modo preciso para atividades que são repetidas muitas vezes sob circunstâncias controladas. No entanto, a natureza única de muitas atividades de projetos de \textit{software} não permite a estimativa precisa de suas probabilidades. Outra limitação dessa definição é que ela abrange somente ameaças conhecidas ou previsíveis, oferecendo opções limitadas para gerenciar ameaças não percebidas. Essa é uma consequência da definição de risco em termos de probabilidade e impacto; uma vez que para se avaliar a probabilidade e o impacto é necessário ter a capacidade de se prever uma eventualidade. Existe ainda uma outra questão em que se questiona se as melhores decisões são baseadas na quantificação numérica, determinada pelos padrões do passado, ou na avaliação subjetiva das incertezas. Não é possível quantificar o efeito de um evento futuro com certeza, mas através da probabilidade, é possível estimá-lo a partir do passado. Apesar de ser difícil encontrar um projeto de \textit{software} padrão, é possível classificar atividades e definir padrões que possibilitem a estimativa. Para Bannerman \cite{bannerman2008risk}, a solução comum para esse problema em projetos de \textit{software} consiste em observar o risco de um modo mais geral, em termos da incerteza, e avaliá-lo qualitativamente  \cite{bannerman2008risk}.
%Risk is a concept that many find difficult to be understood because it involves two complex metrics: probability and severity of adverse effects \cite{Haimes2009}. A limitation in this definition is the practical difficulty of estimating the probability and impact of various risk factors, especially in software projects. Probabilities can only be defined with significance for activities that are repeated many times under controlled conditions. However, the unique nature of many software projects activities do not allow accurate estimation of their probabilities. Other limitation of that definition is that it only encompasses known or foreseeable threats, providing limited options to manage unnoticed threats, but also does not recognize unforeseeable threats. That is a consequence of the definition of risk in terms of probability and impact; since, to assess likelihood and impact is necessary to be able to predict an eventuality. In addition, there is another issue: the best decisions are based on numerical quantification - determined by patterns from the past - or on subjective evaluation of the uncertainties? It is not possible to quantify the future with certainty, but through likelihood, it is possible to predict it based on the past. Although it is difficult to find a standard software project, it is possible to classify activities and set patterns that enable the estimation. To Paul Bannerman \cite{bannerman2008risk}, the usual solution to that problem in software projects is to observe the risk in a broad way, in terms of uncertainty, and evaluate it qualitatively.

Haimes \cite{Haimes2009} considera duas premissas na pesquisa de análise de riscos, que também serão consideradas neste estudo. Uma, que o risco é comumente quantificado através da fórmula matemática de expectativa. No entanto, mesmo que essa fórmula permita uma medida valiosa do risco, ela falha em reconhecer e/ou acentuar consequências de eventos extremos. Tom Kendrick apresenta no seu livro \cite{KEND2003BOOK} um \textit{framework} para a identificação e gerenciamento de catástrofes. A outra premissa, por sua vez, afirma que uma das tarefas mais difícil em análise de sistemas é saber como modelar o risco. Portanto, novas propostas para a análise quantitativa e modelagem de sistemas sob o ponto de vista de seus riscos, poderão contribuir para o avanço científico da área.
%Haimes \cite{Haimes2009} considers two premises in research of risk analysis, which will also be considered during this study. First is that risk is usually quantified by mathematical formula of expectation. However, even though that formula enables a valuable measure of risk, it fails to recognize and or exacerbate the consequences of extreme events. Tom Kendrick presents in his book \cite{KEND2003BOOK} a framework to identify and manage disasters. Second, states that one of the most difficult tasks in systems analysis is to know how to model it. Therefore, new proposals for quantitative analysis and modeling of systems taking into account its risks will contribute to the scientific advances in the field.

A necessidade de gerenciar riscos (eventos indesejados) cresceu exponencialmente com a complexidade dos sistemas \cite{higuera1996software}. Gerenciar esses eventos nesses sistemas torna difícil identificar e estimar a ocorrência de eventos indesejados esperados ou inesperados por conta da imensa quantidade de fatores de riscos envolvidos e suas relações. Há uma necessidade crescente por métodos mais sistemáticos e ferramentas para suplementar o conhecimento individual, julgamento e experiência. Essas características humanas são muitas vezes suficientes para enfrentar riscos de menor complexidade e isolados. Por exemplo, uma parte dos problemas mais sérios encontrados na aquisição de um sistema são os resultados de riscos que são ignorados, devido a sua baixa probabilidade, até que eles já tenham criado consequências mais sérias \cite{higuera1996software}.
%The need to manage risks (undesired events) increases exponentially with system complexity. Managing those events in such complex systems becomes difficult to identify and predict undesirable expected or unexpected events occurrence because of huge amount of risk factors involved and their relations. There is an increasing need for more systematic methods and tools to supplement individual knowledge, judgment and experience. These human traits are often sufficient to address less complex and isolated risks. For example, a portion of the most serious issues encountered in system acquisition are the result of risks that are ignored, due to its low likelihood, until they have already created serious consequences \cite{higuera1996software}.

O Guia de Boas Práticas para o Gerenciamento de Projetos, PMBOK \cite{PMBOK2008}, apresenta a Simulação de Monte Carlo (SMC) como uma boa prática para a análise de risco de projetos. No entanto, existem algumas limitações na adoção dessa abordagem que o torna inviável \cite{Ibbotson2005}. As simulações podem levar a resultados completamente equivocados se entradas inapropriadas, derivadas da inserção de parâmetros manualmente, são inseridas no modelo; já que ela é muito sensível aos parâmetros de entrada. Comumente, o usuário deve estar preparado para realizar os ajustes necessários se os resultados que são gerados parecem fora de rumo. Além disso, Simulação de Monte Carlo não pode modelar correlações entre riscos. Isso significa que os números que surgem em cada sorteio são aleatórios e, em consequência, um resultado pode variar de seu valor mais baixo, em um período, para o mais alto no próximo. Portanto, abordagens alternativas devem ser consideradas para prever a probabilidade de risco e impacto, levando em consideração as características de risco do projeto e as limitações da Simulação de Monte Carlo. Assim, a análise de risco deve ser uma tarefa mais precisa e mais fácil, do ponto de vista dos usuários. Este trabalho considera Redes Neurais Artificial (RNAs) uma alternativa valiosa a ser considerada na análise de risco de projetos de \textit{software}, devido ao fato dela ser interpretada como um modelo de regressão não-linear de grande desempenho.
%The Guide to the Project Management Body of Knowledgment \cite{PMBOK2008} presents Monte Carlo Simulation as a good practice method to project risk analysis. However, there are some limitations in the adoption of this approach that makes it unfeasible \cite{Ibbotson2005}. Simulations can lead to misleading result if inappropriate inputs, derived from subjective parametrization, are entered into the model. Commonly, the user should be prepared to make the necessary adjustments if the results that are generated seem out of line. Moreover, Monte Carlo can not model risks correlations. That means the numbers coming out in each draw are random and in consequence, an outcome can vary from its lowest value, in one period, to the highest in the next. Therefore, alternative approaches must be considered to predict risk likelihood and impact, taking into account project risk characteristics and Monte Carlo Simulation limitations. Thus, risk analysis should be a more accurate and easier task, from users point of view. This work finds artificial neural networks as a valuable alternative to be considered in software project risk analysis.

\section{Objetivos}

\subsection{Questões de Pesquisa}

Como analisar os riscos no gerenciamento de projetos de \textit{software}, também considerando as catástrofes? \\
Como analisar quantitativamente os riscos no gerenciamento de projetos de \textit{software}? \\
Quais dados de registros de riscos de projetos de \textit{software} estão disponíveis para realizar os estudos? \\
Como desenvolver um método para previsão de riscos em gerenciamento de projetos de \textit{software} eficiente para o suporte a tomada de decisão? \\
%How to analyze risks in software project management considering disasters? \\
%How quantitatively analyze risks in software project management? \\
%Which data of risk registers of software project are available to perform the study? \\
%How to develop a method to predict risks of software project management in order to efficiently support decision making? \\

\subsection{Objetivo Geral}

O objetivo principal dessa dissertação é definir uma metodologia para determinar qual é uma abordagem mais eficiente e precisa para a análise de riscos em projetos de software. Algumas técnicas avaliadas são Simulação de Monte Carlo (SMC), Análise PERT, Modelo de Regressão Linear Múltipla (MRLM), Modelo de Regressão em Árvore (MRA), as alternativas de Redes Neurais Artificiais (RNA's) - especificamente, Perceptron de Múltiplas Camadas (MLP), Máquina de Vetor de Suporte (SVM), Redes de Função de Base Radial (RBF) - ou um Sistema \textit{Neuro-Fuzzy} para diminuir o erro na estimativa de impacto dos riscos e a variância das estimativas.
%The main purpose of this dissertation is to define a methodology to determine which is a more efficient approach to software project risk analysis: Monte Carlo Simulation (MCS) technique, Linear Regression Models (LRM's) or Artificial Neural Networks (ANN's) alternatives - Multilayer Perceptron (MLP), Support Vector Machine (SVM), Radial Basis Function (RBF) and Neuro Fuzzy System (NFS) - to improve accuracy and decrease the error prone risk impact estimation.

\subsection{Objetivos Específicos}

\begin{itemize}
\item Desenvolver uma metodologia para previsão do impacto de riscos através da adoção de Redes Neurais Artificiais para gerenciar os riscos num projeto de software;
\item Determinar uma linha de base para a estimativa do impacto de riscos já que não temos uma métrica de comparação ou um \textit{benchmarking}, para a base de dados selecionada;
\item Avaliar abordagens tradicionais de estimativa de impacto de riscos (SMC, Análise PERT, MRLM e MRA) no que se refere ao erro de previsão;
\item Avaliar diversas Redes Neurais Artificias (MLP, RBF) para obter uma melhor configuração para a base de dados selecionada;
\item Determinar um limiar de erro satisfatório para a estimativa de impacto de riscos, através de entrevista com gerentes de projetos;
\item Apresentar resultados compreensíveis, do ponto de vista de gerentes e analistas de projetos e riscos.
\end{itemize}
%\begin{itemize}
%\item Develop a method to predict risk impacts through adoption of artificial neural networks to manage risks in software projects;
%\item Evaluate traditional approaches to risk impact estimation in terms of prediction error;
%\item Evaluate several artificial neural networks to obtain a better configuration to the chosen risk dataset;
%\item Determine a satisfactory limiar error to risk impact estimation.
%\end{itemize}

O primeiro objetivo consiste em determinar uma metodologia para a previsão do impacto de riscos com o propósito de alcançar a maior eficiência possível (quanto mais próximo de 100 \%, melhor), através da minimização do erro de previsão. O segundo objetivo é justo, mediante ao fato de não existir uma linha de base para comparação. Já o terceiro, envolve estudar Simulação de Monte Carlo (SMC) e Análise PERT para compará-las com Modelos de Regressão em Árvore (MRA) e Modelo de Regressão Linear Múltipla (MRLM), que servirão como linha de base (objetivo 2), com o propósito de comprovar se é uma boa prática mesmo adotar SMC e Análise PERT. 

O terceiro objetivo específico é alcançado após experimentar diversas Redes Neurais Artificiais como \textit{Multilayer Perceptron} (MLP), \textit{Support Vector Machine} (SVM) e \textit{Radial Basis Function} (RBF). Além disso, um Sistema \textit{Neuro-fuzzy} (NFS) também foi incluído nesse estudo. Em penúltimo, foi necessário realizar uma pesquisa com profissionais e acadêmicos para identificar um limiar aceitável de erro na estimativa do impacto de um risco, essa entrevista foi feita através de perguntas enviadas por email. Por último, apresentar resultados na forma em que as técnicas tradicionais apresentam, pois são resultados que esses profissionais estão acostumados a interpretar.
%The first objective consists on determining a methodology to estimate risk impact aiming to reach high efficiency, through minimizing estimation error. The second one involves studying Monte Carlo Simulation (MCS) and PERT Analysis to compare with Multiple Linear Regression Model (MLRM) and Regression Tree Model (RTM) aiming to comprove if it is a good practice to utilize MCS and PERT. THe third object is reached after experimenting several artificial neural networks as Multilayer Perceptron (MLP), Support Vector Machine (SVM) e Radial Basis Function (RBF). Moreover a Neuro-fuzzy System (NFS) was also considered on this study. Finally, a survey is performed with Project Management Institute (PMI) Risk Community of Practice (Risk CoP) to determine a satisfactory limiar error to estimate risk impact.

Em resumo, a metodologia adotada neste estudo é realizar uma série de experimentos para avaliar os erros de previsão dos impactos dos riscos oriundos da base de dados do PERIL \cite{kendrick2003identifying}, um \textit{framework} para identificar riscos no gerenciamento de projetos de software. As técnicas selecionadas estimarão o resultado de impactos de risco. A Raiz do Erro Médio Quadrático (REMQ) será calculada trinta vezes para cada abordagem, e, então, um teste de hipótese pode ser necessário para afirmar qual deles é um método mais preciso que se ajusta as particularidades dessa base de dados. Mais detalhes sobre essa metodologia é apresentado no Capítulo \ref{cap:experiments}.
%In summary, the methodology adopted in this study is to make statistical experiments to evaluate the prediction error of risk impact from PERIL dataset \cite{kendrick2003identifying}, a framework to identify risks in software project management. The selected techniques will estimate the outcome of risk impacts. RMSE (Root Mean Square Error) will be calculated thirty times for each approach, and then a hypothesis test may be necessary to assert which is a more accurate method that fits the dataset. Lastly, the method to estimate risk impact is determined. More details are presented in Chapter \ref{cap:experiments}.

É concluído que uma variação da MLP chamada de``MLPRegressor", é uma abordagem vencedora para estimar o impacto de riscos. Além disso, observou-se que todas as alternativas de Redes Neurais Artificiais são melhores que os Modelos de Regressão Linear e Simulação de Monte Carlo. Portanto, não foi descoberto qualquer motivo para atribuir SMC e PERT como métodos recomendados para a análise de riscos, de acordo com os experimentos conduzidos.
%It is concluded that a MLP variation called MLPReg, is the successful approach to estimate risk impacts in this study. Besides that, all artificial neural networks alternatives are better than , both, linear regression models, monte carlo simulation either PERT analysis. Therefore, it could not be found any reason to assign monte carlo simulation and PERT recommended methods to risk analysis according to statistical experiments conducted here.

O restante da dissertação está organizado nos seguintes capítulos: Capítulo \ref{cap:background} aborda gerenciamento de risco de projetos, conceitos de análise de risco qualitativa e quantitativa, Simulação de Monte Carlo, análise PERT, Modelos de Regressão Linear e conceitos de Redes Neurais Artificiais e suas características. O Capítulo \ref{cap:methodology} descreve o banco de dados do PERIL, métodos de pré processamento de dados para preparar os dados para esse estudo e define as configurações dos algoritmos. O Capítulo \ref{cap:experiments} descreve cada experimento. O Capítulo \ref{cap:results} apresenta o resultado dos experimentos. Finalmente, o Capítulo \ref{cap:conclusion} apresenta as conclusões e as sugestões de trabalhos futuros.
%The rest of the dissertation is organized in the following chapters: Chapter \ref{cap:background} address project risk management, qualitative and quantitative risk analysis concepts, monte carlo simulation, linear regression models and artificial neural networks concepts and characteristics. Chapter \ref{cap:methodology} describes PERIL database, data preprocessing methods to prepare the database to the study and describes algorithms configuration. Chapter \ref{cap:experiments} describes each experiment. Chapter \ref{cap:results} presents the obtained results for each experiment. Finally, Chapter \ref{cap:conclusion} presents the conclusions and suggestions of future works.

O trabalho descrito nessa dissertação teve alguns dos seus resultados publicados no artigo:
%The works described in this dissertation had results published in the following papers:
\begin{itemize}
\item C. H. M. S. Timoteo, M. J. S. Valença, S. M. M. Fernandes,``\textit{Evaluating Artificial Neural Networks and Traditional Approaches for Risk Analysis in Software Project Management - A case study with PERIL dataset}", ICEIS 2014: \textit{16th International Conference on Enterprise Information Systems}, Abril, 2014.
\end{itemize}

