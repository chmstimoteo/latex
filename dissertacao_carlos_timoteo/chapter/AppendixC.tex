\chapter*{Apêndice A}


\textbf{Publicações}
\\
\\


\textbf{Título:} Assessing Particle Swarm Optimizers Using Network Science Metrics.

\textbf{Autores:} Marcos A. C. Oliveira-Junior, Carmelo J. A. Bastos-Filho, Ronaldo Menezes.

\textbf{Publicado em:} Springer Complex Networks IV -- Studies in Computational Intelligence Volume 476, 2013, pp 173-184. Proceedings of the 4th Workshop on Complex Networks CompleNet 2013.

\textbf{Abstract:}Particle Swarm Optimizers (PSOs) have been widely used for optimization problems, but the scientific community still does not have sophisticated mechanisms to analyze the behavior of the swarm during the optimization process. We propose in this paper to use some metrics described in network sciences, specifically the R-value, the number of zero eigenvalues of the Laplacian Matrix, and the Spectral Density, in order to assess the behavior of the particles during the search and diagnose stagnation processes. Assessor methods can be very useful for designing novel PSOs or when one needs to evaluate the performance of a PSO variation applied to a specific problem. In order to apply these metrics, we observed that it is not possible to analyze the dynamics of the swarm by using the communication topology because it does not change. Therefore, we propose in this paper the definition of the influence graph of the swarm. We used this novel concept to assess the dynamics of the swarm. We tested our proposed methodology in three different PSOs in a well-known multimodal benchmark function. We observed that one can retrieve interesting information from the swarm by using this methodology.

\

\
\textbf{Título:} Using Network Science to Define a Dynamic Communication Topology for Particle Swarm Optimizers.

\textbf{Autores:} Marcos A. C. Oliveira-Junior, Carmelo J. A. Bastos-Filho, Ronaldo Menezes.

\textbf{Publicado em:} Springer Complex Networks -- Studies in Computational Intelligence Volume 424, 2013, pp 39-47. Proceedings of the 3rd Workshop on Complex Networks CompleNet 2012.

\textbf{Abstract:} We propose here to use network sciences, specifically an approach based on the Barabási-Albert model, to define a dynamic communication topology for Particle Swarm Optimizers. We compared our proposal to previous approaches, including a simpler Barabási-Albert-based approach and other most used approaches, and we obtained better results in average for well known benchmark functions.



\pagebreak
