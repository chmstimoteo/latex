\begin{flushbottom}
\begin{flushleft}
{\huge \textbf{Resumo}}
\linebreak
\end{flushleft}
Muitos projetos de desenvolvimento de software/ti finalizam sem alcançar todos os seus objetivos. Segundo estudos já realizados, somente um quarto dos projetos de software têm sucesso imediato, e bilhões de dólares são perdidos anualmente por meio de falhas ou projetos que não cumprem a entrega dos benefícios prometidos. Além disso, ainda há algumas dificuldades na interpretação do conceito de risco e na aplicação de métodos para a análise de risco de forma eficiente e precisa, conforme descrito nesta dissertação. Isso ocorre devido a vários fatores: falta de habilidade para manusear as ferramentas existentes; falta de conhecimento dos benefícios obtidos pela gestão de risco, bem como as limitações encontradas nas ferramentas sugeridas como boas práticas; a ausência de estudos na indústria que comprovem o valor de cada técnica e a dificuldade na obtenção de uma base de dados de riscos adequada.

Dentre as técnicas convencionais na literatura, há a Simulação de Monte Carlo (SMC) e Análise PERT. Na SMC, as simulações podem levar a resultados enganosos se entradas inapropriadas, através da inserção de parâmetros manualmente, forem inseridas no modelo. Além disso, como ela não pode modelar correlações entre riscos, então os números que surgem em cada sorteio são aleatórios. Na Análise PERT, por sua vez, como há necessidade da opinião de um especialista, basta inserir valores de entrada errôneos para gerar resultados incoerentes.

Nesse trabalho, uma base de dados especial, chamada PERIL, é utilizada e é proposta uma metodologia para a análise de risco cujo diferencial é encontrar uma Rede Neural Artificial que seja mais eficiente e precisa que as técnicas convencionais até então utilizadas (SMC e Análise PERT), independentemente da base de dados a ser empregada. Perceptron de Múltiplas Camadas (MLP), Máquina de Vetor de Suporte, Redes com Função de Base Radial, um Sistema Adaptativo \textit{Neuro-Fuzzy}, Modelos de Regressão Linear, Simulação de Monte Carlo e Análise PERT foram as técnicas avaliadas. Em seguida, são conduzidos experimentos para determinar qual o método mais eficiente e preciso para a estimativa do impacto de risco relativo ao número de dias atrasados para a finalização de uma atividade, baseada em informações da PERIL. Por fim, um intervalo de confiança pôde ser obtido para uma determinada previsão.

Os resultados mostram que uma das variações da MLP (MLPRegressor) obteve melhores resultados que qualquer outra técnica. Além disso, comprova-se que a Simulação de Monte Carlo apresenta os piores resultados ainda que se compare com modelos de regressão linear simples. Por fim, a análise PERT mostra ser bastante eficiente quando se utiliza a opinião especializada e uma base de dados de riscos como a PERIL.
\\
\\
\textbf{Palavras-chave:} \\ Gerenciamento de Projetos, Análise de Riscos, PERIL, Redes Neurais Artificiais, Simulação de Monte Carlo, Análise PERT.\end{flushbottom}
\newpage
