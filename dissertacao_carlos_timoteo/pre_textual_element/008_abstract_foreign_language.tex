\begin{flushbottom}
\begin{flushleft}
{\huge \textbf{Abstract}}
\linebreak
\end{flushleft}
Many software/it projects end in failure, without reaching their objectives. According to previous studies, only a quarter of software/it projects have immediate success, and billions of dollars are lost annually through failures or projects that do not meet the delivery of promised benefits. Moreover, there are still some difficulties in the interpretation of the concept of risk and application of methods for risk analysis efficiently and accurately. This is due to several factors: lack of ability to handle existing tools; lack of knowledge of the benefits from risk management as well as limitations found in the tools suggested as good practice; the lack of studies in the industry proving the value of each technique and the difficulty in obtaining a database of proper risk. 

Among traditional techniques in literature, there are Monte Carlo Simulation (MCS) and PERT Analysis. In MCS, simulations can return misleading results if inappropriate inputs, through manual insertion of parameters, are incorporated into the model. Besides, as it can not model correlations between risks, then the numbers appearing in each drawing are random. In PERT analysis, in turn, as there is a need for specialized opinion, just enter erroneous input values to generate inconsistent results.

In this work, a special database, called PERIL, is used and is proposed a methodology for risk analysis whose diferential is to find an Artificial Neural Network which is more efficient and accurate than standard techniques previously used such as MCS and PERT, regardless of the database employed. Multilayer Perceptron, Support Vector Machine, Radial Basis Function Networks, an Adaptive Neuro-Fuzzy System, Linear Regression Models, Monte Carlo Simulation and PERT Analysis techniques were evaluated. Then, experiments are conducted to determine which is a more efficient and accurate method for risk impact estimation related to the number of delayed days to finish an activity based on PERIL. Finally, a confidence interval could be obtained for a given prediction. 

The results show that a MLP variation (MLPRegressor) has better results than any other technique. Furthermore, it was proven that Monte Carlo Simulation shows the worst results still compared with Linear Regression Models. Finally, PERT analysis shows to be quite efficient when using expert judgment and a risk database as PERIL.
\\
\\
\textbf{Keywords:} \\ Project Management, Risk Analysis, PERIL, Artificial Neural Networks, Monte Carlo Simulation, PERT Analysis.\end{flushbottom}
\newpage
