% ################################################################
% # Recife, 18 February 2011                                     #
% # Polytechnic School of Pernambuco - University of Pernambuco  #
% # Course: Computer Engineer                                    #
% # Author: Marcos Alvares Barbosa Junior                        #
% #                                                              #
% # UPE-POLI MSc Thesis Template    V0.1                         #
% ################################################################
%   TODO:
%     #1:  Move the Default Configuration section to a separated file
%     #2:  Create a LaTeX Class
%     #3:  Create examples of Symbols, Equations, Tables, Figures, sections and sub-sections ...
%     #4:  Discard all specific contents (: replace the thesis title with an generic value like: "Here You Put Your Thesis Title")
%     #5:  Verify ununsed packages
%
\documentclass[a4paper,12pt]{upethesis}
\usepackage{graphicx}
\usepackage{pslatex}
\usepackage{adjustbox}
\usepackage{enumerate}
\usepackage{url}
\usepackage{epstopdf} 
\usepackage{amsmath}
\usepackage[linesnumbered,portuguese,ruled]{algorithm2e} % Formatação de código
%\usepackage[ruled,linesnumbered,noend]{algorithm2e}
%\usepackage{amsmath}% http://ctan.org/pkg/amsmath
%\usepackage{kbordermatrix}%
\usepackage{blkarray}
\graphicspath{{images/}}
% Macro for 'List of Symbols', 'List of Notations' etc...
\def\listofsymbols{%%%%%%%%%%%%%%%%%%%%%%%
%Sample List of Symbols
%%%%%%%%%%%%%%%%%%%%%%%
\begin{tabbing}
% YOU NEED TO ADD THE FIRST ONE MANUALLY TO ADJUST THE TABBING AND SPACES
\parbox{5in}{$\alpha$: The sigmoid parameter in the ABeePSO algorithm\dotfill \pageref{eq:ABeePSO_Step}}\\
\\
%ADD THE REST OF SYMBOLS WITH THE HELP OF MACRO
\parbox{5in}{$\beta$: The displacement of the point of the sigmoid function in the ABeePSO algorithm\dotfill \pageref{eq:ABeePSO_Step}}\\
\\
\parbox{5in}{$\vec B(t)$: The centroid of the fish school in the FSS algorithm\dotfill \pageref{eq:FSS_centroid}}\\
\\
\parbox{5in}{$c_1$: The cognitive acceleration coefficient in the PSO algorithm\dotfill \pageref{eq:PSO_velocity}}\\
\\
\parbox{5in}{$c_2$: The social acceleration coefficient in the PSO algorithm\dotfill \pageref{eq:PSO_velocity}}\\
\\
\parbox{5in}{$\delta$: The acceleration rate in the APSO algorithm\dotfill \pageref{eq:APSO_normalized}}\\
\\
\parbox{5in}{$\Delta f_i$: The difference of fitness values between consecutive iterations\dotfill \pageref{eq:FSS_feed}}\\
\\
\parbox{5in}{$f_{evol}$: The evolutionary factor of the APSO algorithm\dotfill \pageref{eq:APSO_factor}}\\
\\
\parbox{5in}{$f_{s_i}$: The food source selected by guided bee in the ABeePSO algorithm\dotfill \pageref{eq:ABeePSO_Guided}}\\
\\
\parbox{5in}{$MaxTrial$: Stagnation Counter of the ABC algorithm\dotfill \pageref{eq:ABC_scout}}\\
\\
\parbox{5in}{$\vec n(t)$: The social memory in the PSO algorithm\dotfill \pageref{eq:PSO_velocity}}\\
\\
\parbox{5in}{$\omega$: The inertial factor of the in the PSO algorithm\dotfill \pageref{eq:PSO_inertial}}\\
\\
\parbox{5in}{$\vec p(t)$: The cognitive memory in the PSO algorithm\dotfill \pageref{eq:PSO_velocity}}\\
\\
\parbox{5in}{$p_i$: The probability to select the food source in the ABC algorithm\dotfill \pageref{eq:ABC_probability}}\\
\\
\parbox{5in}{$\sigma$: The standard deviation in the APSO algorithm\dotfill \pageref{eq:apso_sigma}}\\
\\
\parbox{5in}{$step$: The step of guide bees in the ABeePSO algorithm\dotfill \pageref{eq:ABeePSO_Step}}\\
\\
\parbox{5in}{$step_{ind}$: The individual step in the FSS algorithm\dotfill \pageref{eq:FSS_stepInd}}\\
\\
\parbox{5in}{$step_{vol}$: The volitive step in the FSS algorithm\dotfill \pageref{eq:FSS_stepVol}}\\
\\
\parbox{5in}{$\vec v(t)$: Velocity of particle in the PSO algorithm\dotfill \pageref{eq:PSO_velocity}}\\
\\
\parbox{5in}{$W_i(t)$: The weight of the fish in the FSS algorithm\dotfill \pageref{eq:FSS_feed}}\\
\\
\parbox{5in}{$\vec x(t)$: Position of particle in the PSO algorithm\dotfill \pageref{eq:PSO_position}}\\

% .
% .
% .
% ALWAYS KEEP THE FOLLOWING LINE
\end{tabbing}
 \clearpage}
\def\addsymbol #1: #2#3{$#1$ \> \parbox{5in}{#2 \hfill \pageref{#3}}\\} 
\def\addacronym[#1][#2]{$#1$ &\parbox{0.5in}{\hfill} #2\\} 
\def\newnot#1{\label{#1}} 
\renewcommand{\listalgorithmcfname}{Lista de Algoritmos}
\newtheorem{Algorithm}{Algoritmo}
\newcommand{\pkfail}{}
\providecommand{\e}[1]{\ensuremath{\times 10^{#1}}} 	% it helps on scientific notation
\setlength{\unitlength}{20pt}
\def\quadrado[#1]{\begin{picture}(2,2)
				\put(0,0){\framebox(2,2){#1}}
				\end{picture}}
				
\def\quadradonb[#1]{\begin{picture}(2,2)
				\put(0,0){\makebox(2,2){#1}}
				\end{picture}}
\usepackage[table]{xcolor}

% comment here if you do not want that
\def\citacao[#1][#2]{\begin{flushright}{\vspace{0.5cm}}
\small``\emph{#1}''\\ 
{\footnotesize \textbf{#2}}
\end{flushright}{\vspace{0.15cm}}}
% uncomment here if you do not want that
%\def\citacao[#1][#2]{}


% % % % % % % % % %
\usepackage{tikz}
\usetikzlibrary{calc,fit}

% code by Andrew Stacey
% http://tex.stackexchange.com/a/50054/3954    
\makeatletter
\tikzset{%
  remember picture with id/.style={%
    remember picture,
    overlay,
    save picture id=#1,
  },
  save picture id/.code={%
    \edef\pgf@temp{#1}%
    \immediate\write\pgfutil@auxout{%
      \noexpand\savepointas{\pgf@temp}{\pgfpictureid}}%
  },
  if picture id/.code args={#1#2#3}{%
    \@ifundefined{save@pt@#1}{%
      \pgfkeysalso{#3}%
    }{
      \pgfkeysalso{#2}%
    }
  }
}

\def\savepointas#1#2{%
  \expandafter\gdef\csname save@pt@#1\endcsname{#2}%
}

\def\tmk@labeldef#1,#2\@nil{%
  \def\tmk@label{#1}%
  \def\tmk@def{#2}%
}

\tikzdeclarecoordinatesystem{pic}{%
  \pgfutil@in@,{#1}%
  \ifpgfutil@in@%
    \tmk@labeldef#1\@nil
  \else
    \tmk@labeldef#1,(0pt,0pt)\@nil
  \fi
  \@ifundefined{save@pt@\tmk@label}{%
    \tikz@scan@one@point\pgfutil@firstofone\tmk@def
  }{%
  \pgfsys@getposition{\csname save@pt@\tmk@label\endcsname}\save@orig@pic%
  \pgfsys@getposition{\pgfpictureid}\save@this@pic%
  \pgf@process{\pgfpointorigin\save@this@pic}%
  \pgf@xa=\pgf@x
  \pgf@ya=\pgf@y
  \pgf@process{\pgfpointorigin\save@orig@pic}%
  \advance\pgf@x by -\pgf@xa
  \advance\pgf@y by -\pgf@ya
  }%
}
\newcommand\tikzmark[2][]{%
\tikz[remember picture with id=#2] #1;}
\makeatother
% end of code by Andrew Stacey

\newcommand\MyBox[5][-1ex]{%
  \tikz[remember picture,overlay,pin distance=0cm]
  {\draw[draw=#4,line width=1pt,fill=#4!20,rectangle,rounded corners]
( $ (pic cs:#2) + (-1ex,2ex) $ ) rectangle ( $ (pic cs:#3) + (#5,#1) $ );
}
}
%
%\fancyhead[L]{
%    \parbox{2cm}{\includegraphics[scale=0.4]{image/poli_logo_white.jpg}}
%}
%
%\fancyhead[R]{
%    \parbox{1cm}{\includegraphics[scale=0.4]{image/upe_logo_white.jpg}}
%}

\fancyfoot[R]{
    \footnotesize {\color{gray}{\thepage}}
}


\begin{document}
  % Thesis Pre-textual elements [START]
\begin{titlepage}

  \begin{center}
  \includegraphics[scale=0.17]{image/UPE_brasao}\\
	\textbf{Universidade de Pernambuco}\\
	\textbf{Escola Politécnica de Pernambuco}\\
    \textbf{Programa de Pós-Graduação em Engenharia da Computação}\\[3cm]

    Carlos Henrique Maciel Sobral Timóteo\\[2cm]


    {\large \textbf{Metodologia para a Análise Quantitativa de Riscos no Gerenciamento de Projetos de Software}}\\[3cm]

    Dissertação de Mestrado\\[2cm]
  \end{center}


\begin{center}
  {\color{white} xx}\vfill
  Recife, Julho de 2014
\end{center}

\end{titlepage}

\begin{titlepage}

  \begin{center}
	\includegraphics[scale=0.17]{image/UPE_brasao}\\
	\textbf{Universidade de Pernambuco}\\
	\textbf{Escola Politécnica de Pernambuco}\\
    \textbf{Programa de Pós-Graduação em Engenharia da Computação}\\[3cm]

    Carlos Henrique Maciel Sobral Timóteo\\[2cm]
{
\Large
    \textbf{Metodologia para a Análise Quantitativa de Riscos no Gerenciamento de Projetos de Software}\\[2cm]
}
  \end{center}

\noindent\adjustbox{varwidth=10cm,right,margin=0pt \bigskipamount}{% (`margin` adds a vertical skip above and belo)
        Dissertação de Mestrado apresentada ao
            Programa de Pós-Graduação em Engenharia
            da Computação da Universidade de
            Pernambuco como requisito parcial para
            obtenção do título de Mestre em Engenharia
            da Computação.
}

  \begin{flushright}
    Prof. Dr. Sérgio Murilo Maciel Fernandes\\
    Orientador\\
    \qquad\\
    Prof. Dr. Mêuser Jorge Valença\\
    Coorientador\\
  \end{flushright}

\begin{center}
  {\color{white} xx}\vfill
  Recife, Julho de 2014
\end{center}

\end{titlepage}


\begin{center}
  {\color{white} xx}\vfill
  REVIEWER NOTES.\\
  Recife, Mar 2014
\end{center}
\newpage 
\pagenumbering{gobble}
%\begin{flushbottom}
\begin{flushleft}
{\huge \textbf{Agradecimentos}}
\linebreak
\linebreak
\end{flushleft}

Agradeço a Deus pela oportunidade da vida para estar me desenvolvendo pessoalmente e profissionalmente. Agradeço a minha família, minha namorada e meus amigos pelo apoio e contribuição para que eu possa estar desempenhando tantas atividades ao mesmo tempo.

Agradeço a Escola Politécnica de Pernambuco, aos meus professores e aos meus colegas de sete anos de convivência e experiências marcantes. Em especial, agradeço ao meu professor orientador, Prof. Dr. Sérgio Murilo, por estar me apoiando e colaborando positivamente nos momentos acadêmicos mais marcantes da minha carreira, como a orientação do trabalho de conclusão de curso de graduação e a orientação para o desenvolvimento da minha pesquisa e da escrita desta dissertação. Agradeço também ao meu professor coorientador, Prof. Dr. Mêuser Valença, pelas discussões, pelos ensinamentos e pela disposição em colaborar com o desenvolvimento da minha pesquisa.

Agradeço a FITec (Fundação para Inovações Tecnológicas LTDA) por me proporcionar um ambiente de crescimento profissional, acadêmico e pessoal. Uma extensão da minha residência e da Universidade. Em especial, aos meus amigos Rodrigo Lira, Gildo Ferrúcio e Sérgio Ribeiro pelas discussões e incentivo para vencer os impedimentos.

Por fim, agradeço a todas pessoas que conviveram comigo nos últimos quatro anos que desejarem o meu crescimento, que apoiaram a realização dos meus sonhos. Vocês me tornaram mais sábio e me deram bons exemplos.

\end{flushbottom}
\newpage

\chapter*{}
\vfill

\begin{flushright}{}
\emph{Jorge sentou praça, na cavalaria \\
E eu estou feliz, porque eu também sou da sua companhia \\
Eu estou vestido com as roupas e as armas de Jorge \\
Para que meus inimigos tenham mãos e não me toquem \\
Para que meus inimigos tenham pés e não me alcancem \\
Para que meus inimigos tenham olhos e não me veja \\
E nem mesmo um pensamento, eles possam ter para me fazerem mal \\
Armas de fogo, meu corpo não alcançarão. \\
Facas e lanças se quebrem sem o meu corpo tocar \\
Cordas e correntes se arrebentem sem o meu corpo amarrar \\
Pois eu estou vestido com as roupas e as armas de Jorge \\
Jorge é da Capadócia, salve Jorge! \\
Perseverança, ganhou do sórdido fingimento \\
E disso tudo nasceu o amor.}\\ 
\quad \\
{\small \textbf{Jorge da Capadócia} \quad \\ Jorge Ben Jor}
\end{flushright}{\small \par}

\begin{flushright}{}
\emph{What is, is, everything happen for a reason, when life kicks you, get up and rip its heart out, never stop fighting.}\\ 
\quad \\
{\small \textbf{Roger W. Kivell} \quad \\ WWII Navy Frogman (Navy Seal)}
\end{flushright}{\small \par}

\newpage

\pagenumbering{roman}
\setcounter{page}{4}
%DESCOMENTAR...
% % % % % % % % %
%\begin{flushbottom}
\textbf{PORTUGUESE LANGUAGE ABSTRACT}
\\
\\
\\
A \'{a}rea de Intelig\^{e}ncia de enxames se tornou uma das \'{a}reas de pesquisa da intelig\^{e}ncia computacional que mais se expandiu nos \'{u}ltimos anos. Os algoritmos de intelig\^{e}ncia de enxames s\~{a}o geralmente usados para otimiza\c{c}\~{a}o e busca, e normalmente apresentam mecanismos para simultaneamente representar sucesso, garantir converg\^{e}ncia e manter a diversidade. O algoritmo otimiza\c{c}\~{a}o por enxames de part\'{i}culas (PSO) utiliza a melhor posi\c{c}\~{a}o encontrada no espa\c{c}o de busca e intensifica a busca em determinadas regi\~{o}es mais promissoras. Por outro lado, o algoritmo de otimiza\c{c}\~{a}o por colmeias artificiais (ABC) utiliza o conceito de fontes de alimento, as quais as abelhas exploram e caso n\~{a}o hajam melhorias de qualidade, elas v\~{a}o buscar novas fontes de alimento.

Estes algoritmos apresentam excelentes habilidades para resolver problemas complexos, contudo muitos perdem a sua efici\^{e}ncia quanto aplicados em problemas de alta dimensionalidade, por exemplo com 1000 vari\'{a}veis de decis\~{a}o. Diante deste cen\'{a}rio, combinar algoritmos de intelig\^{e}ncia de enxames pode ser uma alternativa para resolver problemas complexos de alta dimensionalidade.

Nesta disserta\c{c}\~{a}o, a proposta \'{e} combinar o algoritmo PSO com comportamento adaptativo, que possui um bom mecanismo de converg\^{e}ncia, com o ABC, com a sua capacidade de manter a diversidade. Esse algoritmo combinado \'{e} chamado de \textit{Adaptive Bee and Particle Swarm Optimization} (ABeePSO). O desempenho do algoritmo ABeePSO foi avaliado utilizando as fun\c{c}\~{o}es de teste apresentadas no (\textit{IEEE Congress on Evolutionary Computation 2010}). Foram analisadas  converg\^{e}ncia, diversidade e escalabilidade. Tamb\'{e}m foram realizadas compara\c{c}\~{o}es com outras t\'{e}cnicas de intelig\^{e}ncia de enxames presentes na literatura. Os resultados indicam que o desempenho do ABeePSO foi superior principalmente em espa\c{c}os de busca de alta dimensionalidade.  	
\\
\\
\\  Recife, July 2012
\end{flushbottom}
\newpage

%\begin{flushbottom}
\begin{flushleft}
{\huge \textbf{Abstract}}
\linebreak
\end{flushleft}
Many software/it projects end in failure, without reaching their objectives. According to previous studies, only a quarter of software/it projects have immediate success, and billions of dollars are lost annually through failures or projects that do not meet the delivery of promised benefits. Moreover, there are still some difficulties in the interpretation of the concept of risk and application of methods for risk analysis efficiently and accurately. This is due to several factors: lack of ability to handle existing tools; lack of knowledge of the benefits from risk management as well as limitations found in the tools suggested as good practice; the lack of studies in the industry proving the value of each technique and the difficulty in obtaining a database of proper risk. 

Among traditional techniques in literature, there are Monte Carlo Simulation (MCS) and PERT Analysis. In MCS, simulations can return misleading results if inappropriate inputs, through manual insertion of parameters, are incorporated into the model. Besides, as it can not model correlations between risks, then the numbers appearing in each drawing are random. In PERT analysis, in turn, as there is a need for specialized opinion, just enter erroneous input values to generate inconsistent results.

In this work, a special database, called PERIL, is used and is proposed a methodology for risk analysis whose diferential is to find an Artificial Neural Network which is more efficient and accurate than standard techniques previously used such as MCS and PERT, regardless of the database employed. Multilayer Perceptron, Support Vector Machine, Radial Basis Function Networks, an Adaptive Neuro-Fuzzy System, Linear Regression Models, Monte Carlo Simulation and PERT Analysis techniques were evaluated. Then, experiments are conducted to determine which is a more efficient and accurate method for risk impact estimation related to the number of delayed days to finish an activity based on PERIL. Finally, a confidence interval could be obtained for a given prediction. 

The results show that a MLP variation (MLPRegressor) has better results than any other technique. Furthermore, it was proven that Monte Carlo Simulation shows the worst results still compared with Linear Regression Models. Finally, PERT analysis shows to be quite efficient when using expert judgment and a risk database as PERIL.
\\
\\
\textbf{Keywords:} \\ Project Management, Risk Analysis, PERIL, Artificial Neural Networks, Monte Carlo Simulation, PERT Analysis.\end{flushbottom}
\newpage

\tableofcontents                                                           % CONTENTS
%\listoffigures\addcontentsline{toc}{chapter}{Figures List}              % LIST OF FIGURES
% %\chapter*{Symbols List\hfill} \addcontentsline{toc}{chapter}{Symbols List}
%\newpage
%\chapter*{Sigles List\hfill} \addcontentsline{toc}{chapter}{Sigles List}
%\listofacronyms
%\newpage
%\chapter*{Symbols List\hfill} \addcontentsline{toc}{chapter}{Symbols List}
%\listofsymbols
%\newpage
% %\addcontentsline{toc}{chapter}{Figures List}
% %\listoffigures 
% %\newpage
%\listofalgorithms\addcontentsline{toc}{chapter}{Algorithms List}
%\newpage
%\listoftables\addcontentsline{toc}{chapter}{Tables List}
%\newpage
% % % % % % % % % 
% % % % % % % %
%\newpage
%\addcontentsline{toc}{chapter}{Lista de Figuras}
%\listoffigures 
%\newpage
%\listoftables\addcontentsline{toc}{chapter}{Lista de Tabelas}                % LIST OF TABLES
%%\listofsymbols                                                             % LIST OF SYMBOLS
%\newpage
%\chapter*{Lista de Símbolos\hfill} \addcontentsline{toc}{chapter}{Lista de Símbolos}
%\listofsymbols
%\newpage
%\chapter*{Lista de Siglas\hfill} \addcontentsline{toc}{chapter}{Lista de Siglas}
%\listofacronyms
%\newpage
%%\addcontentsline{toc}{chapter}{Lista de Símbolos}
%%\newpage
%\listofalgorithms\addcontentsline{toc}{chapter}{Lista de Algoritmos}
%\newpage
% % % % % % % %
\setcounter{page}{1}
\pagenumbering{arabic}


% Thesis Chapters [START]
  \chapter{Introduction}\label{cap:introduction}

%Quão arriscados são os projetos de software? Diversos estudos sobre a efetividade de técnicas de previsão de custo, escopo, cronograma; \textit{surveys} com profissionais em \textit{software} na indústria; e análise de portfólio de projetos foi realizada para responder essa questão \cite{budzier2013double}. No entanto, não há um consenso.
How risky are software projects? Several studies about effectiveness of software cost, scope, schedule estimation techniques; surveys from software professionals in industry; and analysis of projects portfolio have been done to answer this question \cite{budzier2013double}. However, there is not a consensus. 

%Todos os projetos envolvem risco. Há sempre pelo menos algum nível de incerteza no resultado de um projeto, independentemente do que o gráfico de Gannt pareça indicar. Projetos de alta tecnologia são particularmente arriscados, por uma série de motivos. Primeiro, há uma grande variedade de projetos técnicos. Esses projetos tem objetivos e aspectos únicos que os diferem significativamente dos trabalhos anteriores, além de apresentar um ambiente de projetos que evolue rapidamente. Além disso, projetos técnicos são frequentemente "enxutos", ou seja, desafiados a trabalhar com financiamento, pessoal e equipamentos inadequados. Para piorar a situação, há uma expectativa generalizada que não corresponde à realidade de que por mais rápido que tenha sido o último projeto, o próximo deve ser ainda mais rápido \cite{kendrick2003identifying}.
Every project involves risk. There is always at least some level of uncertainty in a project's outcome, regardless of what the Gantt chart on the wall seems to imply. High-tech projects are particularly risky, for a number of reasons. First, technical projects are high varied. These projects have unique aspects and objectives that significantly differ from previous work, and the environment for technical projects evolve quickly. In addition, technical projects are frequently "lean", challenged to work with inadequate funding, staff and equipment. Second, to make matters worse, there is a pervasive expectation that however fast the last project may have been, the next one should be even quicker \cite{kendrick2003identifying}.

%Projetos que tiveram sucesso geralmente conseguiram isso porque duas das ações tomadas pelos seus líderes foram determinantes. Primeiro, eles reconheceram que alguns dos trabalhos em qualquer projeto, mesmo projetos de alta tecnologia, não são novos. Nos trabalhos por eles desenvolvidos as notas, registros e lições aprendidas em projetos anteriores puderam ser utilizados como um roteiro para identificar, e em muitos casos evitar, muitos problemas potenciais. Segundo, eles planejaram com afinco o trabalho do projeto, especialmente as partes que exigiam inovação, para possibilitar a compreensão dos desafios futuros e antecipar muitos dos riscos \cite{kendrick2003identifying}.
Projects that succeed generally do so because their leaders do two things well. First, they recognize that a few of the work on any project, even a high-tech project, is not new. For this work, the notes, records, and lessons learned on earlier projects can be a road map for identifying, and in many cases avoiding, many potential problems. Second, they plan project work thoroughly, especially the portions that require innovation, to understand the challenges ahead and to anticipate many of the risks \cite{kendrick2003identifying}.

%Alguns benefícios da boa gestão de riscos de projetos de software são: 
%\begin{enumerate}
%\item redução de custos incorridos com mudanças no software;
%\item desenvolvimento de um plano de respostas a eventos inesperados, conhecido como plano de contingência de riscos;
%\item previsão da probabilidade da ocorrência de eventos indesejados;
%\item seguimento das linhas de base de custo, cronograma e qualidade.
%\end{enumerate}
Some benefits of good risk management of software projects are:
\begin{enumerate}
\item reduction of costs associated with changes in software;
\item reduce management by crisis;
%\item development of a response plan to unexpected events, i.e., a contingency risk plan; 
\item prediction of likelihood of undesired events;
\item tracking the baselines of cost, schedule and quality;
\item increase the chance of project success.
\end{enumerate}

%Tais fatores podem determinar o sucesso dos projetos \cite{HIGUERAHAIMES1996} \cite{PMBOK2008}.
Such factors may determine the success of projects \cite{HIGUERAHAIMES1996} \cite{PMBOK2008}.

\section{Motivation}

%Em 2009, o CHAOS Report \cite{CHAOS2009} mostrou que 32\% dos projetos de software alcançaram sucesso, foram entregues no prazo, de acordo com o orçamento estabelecido e com os requisitos prometidos; 44\% dos projetos foram desafiados, ou o prazo ou o orçamento ou os requisitos não foram cumpridos; não menos importante, 24\% dos projetos falharam e foram cancelados. Isso ocorre devido aos riscos envolvidos nas atividades do projeto e a um gerenciamento de risco de software ausente ou deficiente \cite{ISLAM2009}. 
In 2009, CHAOS Report \cite{CHAOS2009} showed that 32\% of projects achieved success - were delivered on time, on budget and with the promised requirements -; 44\% of the projects were challenged - or schedule or budget or requirements were not fulfilled; not least, 24\% of projects failed and were canceled. That is due to the risks involved in project activities and to a absent or defective software risk management \cite{ISLAM2009}.

% Schmidt e outros autores \cite{schmidt2001identifying} notaram que muitos dos projetos de desenvolvimento de software terminavam com falha. Eles mostraram que cerca de 25\% de todos os projetos de software são cancelados e cerca de 80\% de todos os projetos de software ultrapassaram seus orçamentos, excedendo-os em 50\% na média. Paul Bannerman \cite{bannerman2008risk} afirma que pesquisas na indústria sugerem que somente um quarto dos projetos de software tem sucesso imediato, e bilhões de dólares são perdidos anualmente por meio de falhas ou projetos que não cumprem a entrega dos benefícios prometidos. Além disso, o autor mostra evidências de que isso é um assunto global, impactando organizações do setor privado e público \cite{KPMG2005}.
Schmidt et al. \cite{schmidt2001identifying} have noticed that many software development projects end in failure. They showed that around 25\% of all software projects are canceled outright and as many as 80\% of all software projects run over their budget, exceeding it by 50\% in average. Paul Bannerman \cite{bannerman2008risk} states that industry surveys suggest that only a quarter of software projects succeed outright, and billions of dollars are lost annually through project failures or projects that do not deliver promised benefits. Moreover, the author shows evidences that it's a global issue, impacting private and public sector organizations \cite{KPMG2005}.

%A previsão de possíveis eventos a curto, médio e longo prazos muitas vezes é falha. Ao analisar os riscos e as incertezas, os gerentes de projeto comummente confiam na própria intuição, em vez de utilizarem a lógica e uma análise detalhada. No entanto, o pensamento intuitivo é frequentemente alvo de ilusões, que causam erros mentais previsíveis e decisões eventualmente não embasadas. O método para conciliar o efeito dessas ilusões psicológicas é uma avaliação sistemática dos riscos e esforços na mitigação dos mesmos através de métodos analíticos.
Predict possible events in short, medium and long term is often failure. By analyzing risks and uncertainties, project managers commonly rely on intuition rather than logic and analysis. However, intuitive thinking is often subject to delusions, causing predictable mental errors and eventually poor decisions. A way to balance the effect of these psychological illusions is a systematic risk analysis and efforts to mitigate them through analytical methods. 

%É difícil gerenciar algo que não pode ser medido. Gerentes de projeto devem quantificar a probabilidade de risco, os resultados, e seu efeito cumulativo em um projeto. Além disso, é importante avaliar as várias opções de mitigação: o custo de cada opção e o tempo necessário para a sua realização \cite{VIRINE2009}.
It is difficult to manage something that can not be quantified. Project managers should quantify the probability of risk, the impact, and their cumulative effect on a project. Furthermore, for instance, it is important to evaluate the various mitigation options: the cost for each option and the required time to perform the mitigation \cite{VIRINE2009}. 

%Existe uma dificuldade na interpretação do conceito de risco, principalmente quanto a aplicação desse conhecimento no desenvolvimento e utilização de técnicas eficientes para a análise de risco no gerenciamento de projetos de software. A gestão de riscos e incertezas em projetos de software, é fundamental para a disciplina de gerenciamento de projetos. Entretanto, em momentos econômicos de crise torna-se muito mais difícil realizar o gerenciamento de riscos, devido aos custos incorridos.
Interpreting the concept of risk is a hard task, especially regarding the application of this knowledge in the development and use of efficient procedures for risk analysis in software project management techniques. Managing risk and uncertainty in software projects is critical to project management discipline. However, in times of economic crisis becomes much more difficult to perform risk management, due to the costs incurred.

%Esta é uma área de pesquisa que vem evoluindo e, portanto, novas e melhores metodologias para identificação, medição e controle de itens de risco de software precisam ser desenvolvidas. Segundo Keshlaf e Riddle \cite{KESHLAFRIDDLE2010}, mesmo que existam muitas abordagens ainda há uma grande lacuna com relação ao que é praticado pelas indústrias de software. 
Since it is an area of research that is growing, new and better methodologies to identify, measure and control risk items of software need to be developed. Keshlaf and Riddle \cite{KESHLAFRIDDLE2010} conclude that even if there are many approaches there is still a large gap regarding what is practiced by software industries.

\section{Problem Description}

%Embora o gerenciamento de risco na gestão de projetos de software seja um processo saudável, sua utilização ainda está aquém das expectativas. Algumas causas disso são o acúmulo de responsabilidades dos gerentes de projetos, a baixa importância atribuída a essa área, a falta de conhecimento em gestão de riscos, os custos envolvidos nas atividades de gestão de risco, a falta de habilidade para lidar com as técnicas e ferramentas específicas. Como consequência, o projeto está sujeito à influência negativa de riscos sem haver um plano de contingência, o que pode ocasionar o fracasso do projeto. Conforme identificado por Kwak e Ibbs \cite{kwak2000calculating}, o gerenciamento de risco é a disciplina menos praticada dentre as diferentes áreas do conhecimento no gerenciamento de projetos. Os autores mencionam que, provavelmente, um motivo é que desenvolvedores de software e gerentes de projetos consideram gerenciar processos e atividades que envolvam incerteza como trabalho e custo extras. De acordo com o benchmarking realizado em 2009 pelo Project Management Institute, em 20% dos projetos os seus gerentes não realizam todos os processos de planejamento e em apenas 35% dos projetos o gerenciamento de riscos é realizado de acordo com uma metodologia formal, estruturada por políticas, procedimentos e formulários. Além disso, 46% dos gerentes realizam atividades de gerenciamento em tempo parcial.
Even though risk management in software project management is a healthy process, its adoption is still far from expectations. A few causes are the overloading of responsibilities on project managers, the low importance attributed to the area, the lack of knowledge of risk management, the costs incurred in risk management activities, the lack of technical skill and familiarity with specific tools. Consequently, the project is prone to the negative influence of risks without a contingency plan, which may lead to project failure. Kwak and Ibbs \cite{kwak2000calculating} identified risk management as the least practiced discipline among different project management knowledge areas. The authors mention that, probably, a cause for it is that software developers and project managers perceive managing uncertainty processes and activities as extra work and expense. According to the benchmarking conducted in 2009 by the Project Management Institute, in 20\% of the projects their managers do not perform all the planning processes and in only 35\% of the projects, risk management is conducted according to a formal methodology, structured by policies, procedures and forms. Also, 46\% of managers carry out management activities part a time.

%Barry Boehm \cite{BOEHM1991} definiu risco como a possibilidade de perda ou dano. Essa definição pode ser expressa pela fórmula de exposição ao risco. Mesmo que Boehm cite a exposição ao risco como a técnica mais efetiva para a priorização do risco depois de sua análise, Paul Bannerman \cite{bannerman2008risk} considera essa definição limitada e inapropriada. Na teoria clássica da decisão, risco reflete a variação na distribuição de probabilidade de possiveis resultados, seja negativo ou positivo, associado a uma decisão particular. Esse estudo leva em consideração a definição do \textit{Project Management Institute} \cite{PMBOK2008} em que risco em projeto é um evento ou condição específica que, se ocorrer, tem um efeito positivo ou negativo em ou mais objetivos do projeto. Uma definição complementar proposta por Haimes \cite{haimes2011risk} também é considerada, a qual expressa o risco como uma medida da probabilidade e severidade de efeitos adversos.
Barry Boehm \cite{BOEHM1991} defined risk as the possibility of loss or injury. That definition can be expressed by risk exposure formula. Even Boehm cites risk exposure as the most effective technique for risk priorization after risk analysis, Paul Bannerman \cite{bannerman2008risk} considers this definition limited and unsuitable. In classical decision theory, risk was viewed as reflecting variation in the probability distribution of possible outcomes, whether negative or positive, associated with a particular decision. This study takes into account Project Management Institute \cite{PMBOK2008} definition whereupon project risk is a certain event or condition that, if it occurs, has a positive or negative effect on one or more project objectives. A complementary definition proposed by Haimes \cite{haimes2011risk} is also considered, which express risk as a measure of the probability and severity of adverse effects.

%Um fator de risco é uma variável associada com a ocorrência de um evento inesperado. Fatores de risco são correlacionados, não necessariamente causais e, se um deles ocorrer, implicará em um ou mais impactos. De acordo com Haimes \cite{haimes2011risk}, riscos podem comumente surgir como o resultado não apenas de um processo estocástico não percebido ocorrendo no tempo e no espaço, mas também, baseado em fatores de riscos determinísticos. A estimativa do risco pode ser alcançada baseada em informações históricas ou conhecimento de projetos anteriores similares, ou ainda de outra fonte de informação \cite{PMBOK2008}.
A risk factor is a variable associated with the occurrence of an unexpected event. Risk factors are correlational, not necessarily causal and, if one of them occurs, it may have one or more impacts. According to Haimes \cite{haimes2011risk}, risks can often arise as the result of an underlying stochastic process occurring over time and space, but also, can occur based on deterministic risk factors. Risk estimation can be achieved based on historical information and knowledge from previous similar projects and from other information sources \cite{PMBOK2008}. 

%Risco é um conceito que muitos consideram difícil de ser compreendido por envolver duas métricas complexas: uma medida da probabilidade e da severidade de efeitos adversos \cite{Haimes2009}. Uma limitação dessa definição é a dificuldade prática em se estimar a probabilidade e o impacto de diversos fatores de risco, especialmente em projetos de software. As probabilidades somente podem ser definidas de um modo preciso para atividades que são repetidas muitas vezes sob circunstâncias controladas. No entanto, a natureza única de muitas atividades de projetos de software não permite a estimativa precisa de suas probabilidades. Outra limitação dessa definição é que ela abrange somente ameaças conhecidas ou previsíveis, oferecendo opções limitadas para gerenciar ameaças não percebidas, além de não reconhecer ameaças imprevisíveis. Essa é uma consequência da definição de risco em termos de probabilidade e impacto; uma vez que para se avaliar a probabilidade e o impacto é necessário ter a capacidade de se prever uma eventualidade. Existe ainda uma outra questão em que se questiona se as melhores decisões são baseadas na quantificação numérica, determinada pelos padrões do passado, ou na avaliação subjetiva das incertezas. Não é possível quantificar o futuro com certeza, mas através da probabilidade, é possível prevê-lo a partir do passado. Apesar de ser difícil encontrar um projeto de software padrão, é possível classificar atividades e definir padrões que possibilitem a estimativa. Para Bannerman \cite{bannerman2008risk}, a solução comum para esse problema em projetos de software consiste em observar o risco de um modo mais geral, em termos da incerteza, e avaliá-lo qualitativamente  \cite{bannerman2008risk}.
Risk is a concept that many find difficult to be understood because it involves two complex metrics: probability and severity of adverse effects \cite{Haimes2009}. A limitation in this definition is the practical difficulty of estimating the probability and impact of various risk factors, especially in software projects. Probabilities can only be defined with significance for activities that are repeated many times under controlled conditions. However, due to the unique nature of some software projects, its activities do not allow accurate estimation of probabilities. That is because some software projects involves innovation activities and  dependends on staff experience. Other limitation of that definition is that it only encompasses known or foreseeable threats, providing limited options to manage unnoticed threats, but also does not recognize unforeseeable threats. That is a consequence of the definition of risk in terms of probability and impact; since, to assess likelihood and impact is necessary to be able to predict an eventuality. In addition, there is another issue: the best decisions are based on numerical quantification - determined by patterns from the past - or on subjective evaluation of the uncertainties? It is not possible to quantify the future with certainty, but through likelihood, it is possible to predict it based on the past. Although it is difficult to find a standard software project, it is possible to classify activities and set patterns that enable the estimation. To Paul Bannerman \cite{bannerman2008risk}, the usual solution to that problem in software projects is to observe the risk in a broad way, in terms of uncertainty, and evaluate it qualitatively.

%Haimes \cite{Haimes2009} considera duas premissas na pesquisa de análise de riscos, que também serão consideradas neste estudo. Uma, que o risco é comumente quantificado através da fórmula matemática de expectativa. No entanto, mesmo que essa fórmula permita uma medida valiosa do risco, ela falha em reconhecer e/ou acentuar consequências de eventos extremos. Tom Kendrick apresenta no seu livro \cite{KEND2003BOOK} um \textit{framework} para a identificação e gerenciamento de catástrofes. A outra premissa, por sua vez, afirma que uma das tarefas mais difícil em análise de sistemas é saber como modelar o risco. Portanto, novas propostas para a análise quantitativa e modelagem de sistemas sob o ponto de vista de seus riscos, poderão contribuir para o avanço científico da área.
Haimes \cite{Haimes2009} considers two premises in research of risk analysis, which will also be considered during this study. First is that risk ($R$) is usually quantified by mathematical formula of expectation, i. e. $R = P \times I$, where $P$ is risk probability and $I$ is risk impact. However, even though that formula enables a valuable measure of risk, it fails to recognize and or exacerbate the consequences of extreme events due to its low probability which make the product result a lower value. Tom Kendrick presents in his book \cite{KEND2003BOOK} a framework to identify and manage disasters. Second, states that one of the most difficult tasks in systems analysis is to know how to model it. Therefore, new proposals for quantitative analysis and modeling of systems taking into account its risks will contribute to the scientific advances in the field.

%A necessidade de gerenciar riscos (eventos indesejados) cresceu exponencialmente com a complexidade dos sistemas. Gerenciar esses eventos nesses sistemas complexos torna difícil identificar e estimar a ocorrência de eventos indesejados esperados ou inesperados por conta da imensa quantidade de fatores de riscos envolvidos e suas relações. Há uma necessidade crescente por métodos mais sistemáticos e ferramentas para suplementar o conhecimento individual, julgamento e experiência. Essas características humanas são muitas vezes suficientes para enfrentar riscos de menor complexidade e isolados. Por exemplo, uma parte dos problemas mais sérios encontrados na aquisição de um sistema são os resultados de riscos que são ignorados, devido a sua baixa probabilidade, até que eles já tenham criado consequências mais sérias \cite{higuera1996software}.
The need to manage risks (undesired events) increases exponentially with system complexity \cite{higuera1996software}. Managing those events in such complex systems becomes difficult to identify and predict undesirable expected or unexpected events occurrence because of huge amount of risk factors involved and their relations. There is an increasing need for more systematic methods and tools to supplement individual knowledge, judgment and experience. These human traits are often sufficient to address less complex and isolated risks. For example, a portion of the most serious issues encountered in system acquisition are the result of risks that are ignored, due to its low likelihood, until they have already created serious consequences \cite{higuera1996software}.

% O guia para o gerenciamento de projetos de PMBOK \cite{PMBOK2008} apresenta a simulação de monte carlo como um método de boa prática para a análise de risco de projetos. No entanto, existem algumas limitações na adoção dessa abordagem que o torna inviável \cite{Ibbotson2005}. As simulações podem levar a resultados enganosos se entradas inapropriadas, derivadas da parametrização subjetiva, são inseridas no modelo. Comumente, o usuário deve estar preparado para realizar os ajustes necessários se os resultados que são gerados parecem fora de rumo. Além disso, simulação de monte carlo não pode modelar correlações entre riscos. Isso significa que os números que surgem em cada sorteio são aleatórios e, em consequência, um resultado pode variar de seu valor mais baixo, em um período, para o mais alto no próximo. Portanto, abordagens alternativas devem ser consideradas para prever a probabilidade de risco e impacto, levando em consideração as características de risco do projeto e as limitações da simulação de Monte Carlo. Assim, a análise de risco deve ser uma tarefa mais precisa e mais fácil, do ponto de vista dos usuários. Esse trabalho considera redes neurais artificial uma alternativa valiosa a ser considerada na análise de risco de projetos de software.
The Guide to the Project Management Body of Knowledgment \cite{PMBOK2008} presents Monte Carlo Simulation as a good practice method to project risk quantitative analysis. However, there are some limitations in the adoption of this approach that makes it unfeasible \cite{Ibbotson2005}. Simulations can lead to misleading result if inappropriate inputs, derived from subjective parametrization, are entered into the model. Commonly, the user should be prepared to make the necessary adjustments if the results that are generated seem out of line. Moreover, Monte Carlo can not model risks correlations. That means the numbers coming out in each draw are random and in consequence, an outcome can vary from its lowest value, in one period, to the highest in the next. Therefore, alternative approaches must be considered to predict risk likelihood and impact, taking into account project risk characteristics and Monte Carlo Simulation limitations. Thus, risk analysis should be a more accurate and easier task, from users point of view. This work finds artificial neural networks as a valuable alternative to be considered in software project risk analysis.

\section{Objectives}

\subsection{Research Questions}

%Como analizar os riscos no gerenciamento de projetos de software, também considerando as catástrofes? \\
%Como analisar quantitativamente os riscos no gerenciamento de projetos de software? \\
%Quais dados de registros de riscos de projetos de software estão disponíveis para realizar os estudos? \\
%Como desenvolver um método para previsão de riscos em gerenciamento de projetos de software eficiente para o suporte a tomada de decisão? \\
How to analyze risks in software project management considering disasters? \\
How quantitatively analyze risks in software project management? \\
Which data of risk registers of software project are available to perform the study? \\
How to develop a method to predict risks of software project management in order to efficiently support decision making? \\

\subsection{General Objective}

%O objetivo principal dessa dissertação é definir uma metodologia para determinar qual é a abordagem mais eficiente para a análise de riscos em projetos de software: Simulação de Monte Carlo (SMC), Modelos de Regressão Linear (MRL's) ou as alternativas de Redes Neurais Artificiais (RNA's) - especificamente, Perceptron de Múltiplas Camadas (MLP), Máquina de Vetor de Suporte(SVM) e Redes de Função de Base Radial (RBF) - para melhorar a precisão e diminuir a estimativa de impacto de riscos propensa ao erro.
The main purpose of this dissertation is to define a methodology to determine which is a more efficient approach to software project risk analysis: Monte Carlo Simulation (MCS) technique, Linear Regression Models (LRM's) or Artificial Neural Networks (ANN's) alternatives - Multilayer Perceptron (MLP), Support Vector Machine (SVM), Radial Basis Function (RBF) and Neuro Fuzzy System (NFS) - to improve accuracy and decrease the error prone risk impact estimation.

\subsection{Specific Objectives}

%\begin{itemize}
%\item Desenvolver uma metodologia para prever o impacto de riscos através da adoção de redes neurais artificiais para gerenciar os riscos num projeto de software;
%\item Avaliar abordagens tradicionais de estimativa de impacto de riscos no que se refere ao erro de previsão;
%\item Avaliar diversas redes neurais artificias para obter a melhor configuração para a base de dados selecionada;
%\item Determinar um limiar de erro satisfatório para a estimativa de impacto de riscos.
%\end{itemize}
\begin{itemize}
\item Develop a method to predict risk impacts through adoption of artificial neural networks to manage risks in software projects;
\item Evaluate traditional approaches to risk impact estimation in terms of prediction error;
\item Evaluate several artificial neural networks to obtain a better configuration to the chosen risk dataset;
\item Determine a satisfactory limiar error to risk impact estimation.
\end{itemize}

%O primeiro objetivo consiste em determinar uma metodologia para a previsão do impacto de riscos com o objetivo de alcançar a maior eficiência possivel, através da minimização do erro de previsão. Já o segundo, envolve estudar Simulação de Monte Carlo (SMC) e Análise PERT para compará-las com Modelo de Árvore de Regressão (MAR) e Modelo de Regressão Linear Múltipla (MRLM) com o objetivo de comprovar se é uma boa prática adotar as duas primeiras. O terceiro objetivo específico é alcançado após experimentar diversas redes neurais artificiais como \textit{Multilayer Perceptron} (MLP), \textit{Support Vector Machine} (SVM) e \textit{Radial Basis Function} (RBF). Além disso, um Sistema \textit{Neuro-fuzzy} (NFS) também foi incluído nesse estudo. Por fim, foi necessário realizar uma pesquisa com profissionais e acadêmicos para identificar um limiar aceitável de erro na estimativa do impacto de um risco.
The first objective consists on determining a methodology to estimate risk impact aiming to reach high efficiency, through minimizing estimation error. The second one involves studying Monte Carlo Simulation (MCS) and PERT Analysis to compare with Multiple Linear Regression Model (MLRM) and Regression Tree Model (RTM) aiming to comprove if it is a good practice to utilize MCS and PERT. THe third object is reached after experimenting several artificial neural networks as Multilayer Perceptron (MLP), Support Vector Machine (SVM) e Radial Basis Function (RBF). Moreover a Neuro-fuzzy System (NFS) was also considered on this study. Finally, a survey is performed with Project Management Institute (PMI) Risk Community of Practice (Risk CoP) to determine a satisfactory limiar error to estimate risk impact.

%Em resumo, a metodologia adotada neste estudo é realizar experimentos estatísticos para avaliar os erros de previsão dos impactos dos riscos oriundos da base de dados do PERIL \cite{kendrick2003identifying}, um \textit{framework} para identificar riscos no gerenciamento de projetos de software. As técnicas selecionadas estimarão o resultado de impactos de risco. A Raiz do Erro Médio Quadrático (REMQ) será calculada trinta vezes para cada abordagem, e, então, um teste de hipótese pode ser necessário para afirmar qual deles é um método mais preciso que se ajusta as particularidades dessa base de dados. Mais detalhes sobre essa metodologia é apresentado no Capítulo \ref{cap:experiments}.
In summary, the methodology adopted in this study is to make statistical experiments to evaluate the prediction error of risk impact from PERIL dataset \cite{kendrick2003identifying}, a framework to identify risks in software project management. The selected techniques will estimate the outcome of risk impacts. RMSE (Root Mean Square Error) will be calculated thirty times for each approach, and then a hypothesis test may be necessary to assert which is a more accurate method that fits the dataset. Lastly, the method to estimate risk impact is determined. More details are presented in Chapter \ref{cap:experiments}.

%É concluído que uma variação da MLP chamada de \textit{MLPReg}, é a abordagem vencedora para estimar o impactos de riscos. Além disso, observou-se que todas as alternativas de redes neurais artificiais são melhores que os modelos de regressão linear, simulação de monte carlo, tanto quanto, análise PERT. Portanto, não foi descoberto qualquer motivo para atribuir SMC e PERT como métodos recomendados para a análise de riscos, de acordo com os experimentos estatísticos conduzidos.
It is concluded that a MLP variation called MLPReg, is the successful approach to estimate risk impacts in this study. Besides that, all artificial neural networks alternatives are better than , both, linear regression models, monte carlo simulation either PERT analysis. Therefore, it could not be found any reason to assign monte carlo simulation and PERT recommended methods to risk analysis according to statistical experiments conducted here.

%O restante da dissertação está organizado nos seguintes capítulos: Capítulo \ref{cap:background} aborda gerenciamento de risco de projetos, conceitos de análise de risco qualitativa e quantitativa, simulação de monte carlo, modelos de regressão linear e definições de redes neurais artificiais e suas características. O Capítulo \ref{cap:methodology} descreve o banco de dados do PERIL, métodos de pré processamento de dados para preparar os dados para esse estudo e define as configurações dos algoritmos. O Capítulo \ref{cap:experiments} descreve cada experimento. O Capítulo \ref{cap:casestudy} apresenta um banco de dados de registros de risco mencionado anteriormente. Finalmente, o Capítulo \ref{cap:conclusion} apresenta as conclusões e as sugestões de trabalhos futuros.
The rest of the dissertation is organized in the following chapters: Chapter \ref{cap:background} address project risk management, qualitative and quantitative risk analysis concepts, monte carlo simulation, linear regression models and artificial neural networks concepts and characteristics. Chapter \ref{cap:methodology} describes PERIL database, data preprocessing methods to prepare the database to the study and describes algorithms configuration. Chapter \ref{cap:experiments} describes each experiment. Chapter \ref{cap:results} presents the obtained results for each experiment. Finally, Chapter \ref{cap:conclusion} presents the conclusions and suggestions of future works.

%O trabalho descrito nessa dissertação teve alguns dos seus resultados publicados no artigo:
The works described in this dissertation had results published in the following papers:
\begin{itemize}
\item C. H. M. S. Timoteo, M. J. S. Valença, S. M. M. Fernandes, "\textit{Evaluating Artificial Neural Networks and Traditional Approaches for Risk Analysis in Software Project Management - A case study with PERIL dataset}", ICEIS 2014: \textit{16th International Conference on Enterprise Information Systems}, Abril, 2014.
\end{itemize}


  \chapter{Swarm Intelligence Fundamentals}\label{cap:Swarm}
Swarm intelligence algorithms were inspired by several examples provided by nature. This inspiration was found on the behavior of social microorganisms, animals and insects, such as: ant colonies~\cite{ACO:Dorigo1999}\cite{ACO:Dorigo2005}\cite{ACS:Dorigo} \cite{MMAS:Kovarik}, fish schools \cite{FSS:Bastos-Filho2008}\cite{FSA:Yazdani2012}, bee colonies~\cite{ABC:Karaboga2005}\cite{BCO:Teodorovic2005}\cite{BCO:Teodorovic2006}\cite{BSO:Akbari2009}\cite{BSO:Akbari2010}\cite{BA:Pham2005}    \cite{BA:Pham2007}, flocks of birds~\cite{PSO:Eberhart1995}, swarms of fireflies~\cite{GSO:Krishnanand2009}\cite{FA:Yang2009}, bacterial colonies~\cite{BFO:Das2009}\cite{BFA:Muller2002}, among others.

The swarm intelligence algorithms are often applied to solve optimization problems without constraints. Some of these algorithms present good abilities do tackle some types of situations that may happen during optimization processes.

In the following sections, we describe some of the most used swarm intelligence algorithms for optimization and their variations to tackle continuous search spaces. Section \ref{sec:PSO} presents the Particle Swarm Optimization (PSO) algorithm. Section \ref{sec:APSO} describes the PSO algorithm with an adaptive behavior, called Adaptive Particle Swarm Optimization (APSO). Sections \ref{sec:ClanPSO} and \ref{sec:ClanAPSO} detail the PSO and APSO with communication topology based on clans, called Clan Particle Swarm Optimization (ClanPSO) and Clan Adaptive Particle Swarm Optimization (ClanAPSO), respectively. Section \ref{sec:ABC} presents the main concepts regarding Artificial Bee Colony (ABC). Finally, Section \ref{sec:FSS} presents the Fish School Search (FSS) algorithm.

\section{Particle Swarm Optimization - PSO}\label{sec:PSO}
Particle Swarm Optimization (PSO) is a computational intelligence technique proposed by James Kennedy and Russell Eberhart in 1995 \cite{PSO:Kennedy} \cite{PSO:Eberhart1995}\cite{PSO:Eberhart1995a} \cite{PSO:Schoene}. PSO is a population based algorithm inspired by the social behavior of flocks of birds that can be used to solve optimization and search problems. PSO was applied in many several real-world problems \cite{PSO:Wachowiak2004} \cite{PSO:Lu2002} \cite{PSO:Lian2006}.

In the PSO approach, the swarm is composed by a population of particles ($N$), where each particle $i$ has four attributes: the position within the search space $\vec x_i(t)$, that represents a possible solution for the problem; the velocity of the particle $\vec v_i(t)$, that is used to update the position of the particle $i$; the best position found by the particle during the search process $\vec p_i(t)$(also known as the cognitive memory); and the best position found by the neighborhood of the particle $\vec n_i(t)$ (also known as the social memory).

The neighborhood of the particle $i$ is the set of particles of the swarm from which the particle $i$ is able to acquire information. This information is used to update the swarm status (\textit{i.e.} velocity and position). The neighborhood is often defined by the communication topology. Different topologies were already proposed \cite{PSO:KennedyA}\cite{PSO:KennedyB} and the most used topologies are depicted in Figure~\ref{fig:topologies}.

\begin{figure}[!h]
\centering
\subfigure[]{\includegraphics[scale=0.2]{image/Star.png}\label{fig:Star}}
\hspace{1mm}
\subfigure[]{\includegraphics[scale=0.2]{image/Ring.png}\label{fig:Ring}}
\hspace{1mm}
\subfigure[]{\includegraphics[scale=0.2]{image/vonneumann.png}\label{fig:vonneumann}}
\caption{Standard PSO communication topologies: (a) Star Topology used in $g_{best}$, (b) Ring Topology used in $l_{best}$ and (c) Von Neumann Topology.}
\label{fig:topologies}
\end{figure}

Figure \ref{fig:Star} presents the \emph{star} topology, in which the particles share information globally through a fully-connected structure. In this case, the information is quickly spread within the swarm and allows a quick convergence of the swarm. This topology is also known as global topology or $g_{best}$. Figure \ref{fig:Ring} presents the \emph{ring} topology, in which each particle is connected solely to $n$ neighbors. In this case, the information exchange mechanism allows a slower dissemination of information and helps to avoid local minima. The Ring topology is also known as local topology or $l_{best}$. The Von Neumann topology is depicted in Figure \ref{fig:vonneumann}. This is a balanced solution that consists in particles connected by a grid \cite{ClanPSO:Carvalho2009}.

The PSO execution occurs balancing the social learning and the individual learning of the individuals within the swarm. During each algorithm iteration, the particles move through the search space by updating their velocities and positions. There are several equations that can be used to update the velocity of the particles \cite{PSO:Clerc2002}. The most used equations to update the velocity and the position of particles are presented in the equations (\ref{eq:PSO_velocity}) and (\ref{eq:PSO_position}), respectively.
\begin{equation}\label{eq:PSO_velocity}
\vec v(t+1) = \omega \vec v(t) + r_1c_1[\vec p(t) - \vec x(t)] + r_2c_2[\vec n(t) - \vec x(t)],
\end{equation}

\begin{equation}\label{eq:PSO_position}
\vec x(t+1) = \vec x(t) + \vec v(t),
\end{equation}
in which $\vec v(t)$ is the velocity of particle in time step $t$, $r_1$ and $r_2$ are random numbers selected by using an uniform probability density distribution in the interval $[0,1]$. $c_1$ is the cognitive acceleration coefficient, $c_2$ is the social acceleration coefficient and $\omega$ is the inertial factor. $r_1c_1[\vec p(t) - \vec x(t)]$ is the cognitive component, which attracts the particle to its own best position ($\vec p(t)$), and $r_2c_2[\vec n(t) - \vec x(t)]$ is the social component, which attracts the particle to the best position found by its own neighborhood ($\vec n(t)$).

In the beginning of the algorithm execution, position ($\vec x(t)$) and velocity ($\vec v(t)$) of particles are attributed randomly. In the state-of-art \cite{PSO:Shi} \cite{PSO:Bratton2007}, the standard values of parameters $c_1$, $c_2$ are equal (2.05) and the inertial factor($\omega$) linearly decreasing from $\omega_{max} = 0.9$ to $\omega_{min} = 0.4$, along the iterations. Equation (\ref{eq:PSO_inertial}) calculates the linear decrement of the inertial factor \cite{PSO:Shi}.
\begin{equation}\label{eq:PSO_inertial}
\omega = \omega_{max} - \Bigl[(\omega_{max} - \omega_{min})\frac{g(t)}{g_{end}}\Bigr],
\end{equation}
where $g(t)$ is current iteration and $g_{end}$ is total number of iterations. The inertial factor optimizes the exploration-exploitation tradeoff. This decrement is done to allow the swarm to have a higher exploration ability in the initial iterations, and a higher exploitation capability in final iterations, when probably the swarm has found a good region of the search space to exploit.

\subsection{Pseudocode of the PSO Algorithm}
The pseudocode of the PSO algorithm is shown in Algorithm \ref{alg:PSO}. One can observe that the PSO algorithm, basically, consists on particle movement in search space (lines 6 and 7) and the update process of the cognitive (lines 9 and 10) and social memories (lines 12 and 13, respectively). The adaptation of inertial factor (line 16) allows the algorithm to regulate the exploitation-exploration tradeoff.

\begin{algorithm}[!h]
    Initialize particles of swarm\;
    Initialize inertial factor\;
    Initialize social and cognitive memory\;
    \While {the stop criterion is not achieved}{
        \For {each particle}{
            Update the velocity according to Equation (\ref{eq:PSO_velocity})\;
            Update the position according to Equation (\ref{eq:PSO_position})\;
            Evaluate the position using the objective function of problem\;
            \If{current position is better than cognitive memory}
            {
                Update the cognitive memory\;
            }
            \If{current position is better than social memory}
            {
                Update the social memory\;
            }
        }
        Update inertial factor according to Equation (\ref{eq:PSO_inertial})\;
    }
    \Return the social memory.
    \caption{Pseudocode of the PSO algorithm.}
    \label{alg:PSO}
\end{algorithm}

\section{Adaptive Particle Swarm Optimization - APSO}\label{sec:APSO}
Zhan \emph{et al.} proposed the Adaptive Particle Swarm Optimization (APSO) in 2009 \cite{APSO:Zhan2009}. This variation of PSO was proposed to solve two inefficiencies of the standard PSO algorithm: low convergence velocity and incapability to avoid local minima. The Adaptive PSO aims to achieve these goals with a systematic adaptation of the parameters and the use of an elitist learning strategy.

Basically, the APSO consists in a loop with the following steps: (\emph{i}) evaluate and estimate the distribution of particles in search space through the metric proposed by authors, called evolutionary factor; (\emph{ii}) classify the evolutionary state of the swarm; (\emph{iii}) determinate the acceleration coefficients based on the evolutionary state (improving the convergence velocity); (\emph{iv}) maintain the diversity with elitist learning strategy; and, (\emph{v}) adapt the inertial factor, increasing the efficiency of the search process.

\subsection{Estimation of Evolutionary Factor}
One needs to estimate the evolutionary factor in order to control the adaptation process of the APSO. To accomplish this, it is necessary to evaluate the distribution of the particles in the search space along the iterations. One needs to calculate the average Euclidian distance of each particle to the other particles of the swarm. The average distance ($d_i$) between particle $i$ and the rest of the swarm is evaluated by Equation (\ref{eq:APSO_distance}).
\begin{equation}\label{eq:APSO_distance}
d_i = \frac{1}{N-1}\sum_{j=1,j\neq i}^{N}\sqrt{\sum_{k=1}^{D}(x_i^k - x_j^k)},
\end{equation}
in which $N$ is the number of particles in the swarm and $D$ is the number of dimensions.

Then, the evolutionary factor ($f_{evol}$) is calculated according to Equation (\ref{eq:APSO_factor}).
\begin{equation}\label{eq:APSO_factor}
f_{evol} = \frac{d_g - d_{min}}{d_{max} - d_{min}} \ \  \in \ \ [0,1],
\end{equation}
in which $d_g$ is the average distance between the best particle of the swarm and the rest of the swarm, $d_{min}$ and $d_{max}$ are the smallest and the biggest average distance among all particles, respectively. To avoid $d_{min} = d_{max}$, it is necessary to have at least three particles in the population.

$f_{evol}$ is a number in the interval $[0,1]$. If the swarm is close to the best particle, then the evolutionary factor is close to $0$. On the other hand, if the particles of swarm are completely spread over the search space, then the evolutionary factor is close to $1$.

\subsection{Classification of Evolutionary State}
The original proposal of the APSO algorithm classifies the swarm in an evolutionary state using fuzzy rules. The evolutionary state is chosen at each iteration based on the membership function with higher value. Figure \ref{fig:factor_APSO} presents the four membership functions for the four evolutionary states: Convergence, Exploitation, Exploration and Jumping out.

\begin{figure}[!h]
\centering
 \includegraphics[width=0.65\textwidth]{image/factor}
 \caption{\small{Fuzzy membership functions for the APSO algorithm.}}
 \label{fig:factor_APSO}
\end{figure}

\emph{Convergence State}: it occurs when the value of the evolutionary factor is minimal and the particles are very close of the best particle. This means that the algorithm found a good region of the search space and then it is currently refining the solutions. We calculate the fuzzy value of this state according to Equation (\ref{eq:s_convergence}).
\begin{equation}
S_{convergence}(f_{evol}) = \begin{cases}
1,                 & \mbox{if $0.0 \leq f_{evol} \leq 0.1$}, \\
-5f_{evol} + 1.5,  & \mbox{if $0.1    < f_{evol} \leq 0.3$}, \\
0,                 & \mbox{if $0.3    < f_{evol} \leq 1.0$}.
\end{cases}
\label{eq:s_convergence}
\end{equation}

\emph{Exploitation State}: it occurs when the value of the evolutionary factor is shrunk and the particles are near to the best particle, it indicates that the swarm found a good region of the search space. The membership function of this state is defined according to Equation (\ref{eq:s_exploitation}).
\begin{equation}
S_{exploitation}(f_{evol}) = \begin{cases}
0,                &\mbox{if $ 0.0 \leq f_{evol} \leq 0.2 $}, \\
10f_{evol} - 2,   &\mbox{if $ 0.2 <    f_{evol} \leq 0.3 $}, \\
1,                &\mbox{if $ 0.3 <    f_{evol} \leq 0.4 $}, \\
-5f_{evol} + 3,   &\mbox{if $ 0.4 <    f_{evol} \leq 0.6 $}, \\
0,                &\mbox{if $ 0.6 <    f_{evol} \leq 1.0 $}.
\end{cases}
\label{eq:s_exploitation}
\end{equation}

\emph{Exploration State}: it occurs when the value of the evolutionary factor is medium to large and the particles have a medium or large distant of the best particle. In this case, the algorithm is trying to find a good region of the search space. We calculate the fuzzy value of this state according to Equation (\ref{eq:s_exploration}).
\begin{equation}
S_{exploration}(f_{evol}) = \begin{cases}
0,                &\mbox{if $ 0.0 \leq f_{evol}  \leq 0.4 $}, \\
5f_{evol} - 2,    &\mbox{if $ 0.4   <  f_{evol}  \leq 0.6 $}, \\
1,                &\mbox{if $ 0.6   <  f_{evol}  \leq 0.7 $}, \\
-10f_{evol} + 8,  &\mbox{if $ 0.7   <  f_{evol}  \leq 0.8 $},  \\
0,                &\mbox{if $ 0.8   <  f_{evol}  \leq 1.0 $}.
\end{cases}
\label{eq:s_exploration}
\end{equation}

\emph{Jumping out State}: it occurs when the evolutionary factor presents large values. This means that the APSO is jumping out of a local optimum to a new region, in other words, the globally best particle is distinctively away from the cluster of the swarm. The membership function of this state is defined according to Equation (\ref{eq:s_jumpingout}).
\begin{equation}
S_{jumping_out}(f_{evol}) = \begin{cases}
0,                  &\mbox{if $ 0.0 \leq f_{evol} \leq 0.7 $}, \\
5f_{evol} - 3.5,    &\mbox{if $ 0.7    < f_{evol} \leq 0.9 $}, \\
1,                  &\mbox{if $ 0.9    < f_{evol} \leq 1.0 $}.
\end{cases}
\label{eq:s_jumpingout}
\end{equation}

\subsection{Adaptation of Acceleration Coefficients}
The next step is to update the acceleration coefficients $c_1$ and $c_2$, which are initialized with values equal to 2.0 and are updated according to the current value of $f_{evol}$. The rules to update the coefficients are depicted in Table \ref{tab:apso_strategies}.

\begin{table}[!ht]
\caption{Strategies for the control of acceleration coefficients $c_1$ and $c_2$.}
\centering
\begin{tabular}{c c c}
\hline
Evolutionary State  &   $c_{1}$               &  $c_{2}$ \\
\hline
Convergence    & Increase slightly     & Increase slightly \\
Exploitation   & Increase slightly     & Decrease slightly \\
Exploration    & Increase              & Decrease \\
Jumping out    & Decrease              & Increase  \\
\hline
\end{tabular}
\label{tab:apso_strategies}
\end{table}

In the \emph{Convergence} state, the $c_1$ value is slightly increased and the $c_2$ value slightly is increased. In this state, the swarm is located at a minima, and, hence, the influence of social coefficient should be prioritized to guide other particles to this region. Thus, the value of $c_2$ should be increased. On the other hand, the value of the cognitive coefficient should be decreased to guarantee the swarm to converge faster. The consequence of this strategy is a premature saturation of the cognitive and the social coefficients to their lower and upper bounds, respectively. Thus, the swarm will strongly be attracted by the current best region, causing premature convergence, which is unsuitable if the current best region is just a local optimum. To avoid this, both $c_1$ and $c_2$ are slightly increased.

In the \emph{Exploitation} state, the $c_1$ value is slightly increased and the $c_2$ value is slightly decreased. In the Exploitation state, each particle uses local information and the swarm form groups in potential local optimal regions, that were identified by the historical best position of each particle. Hence, $c_1$ is slowly increased and maintains a relatively large value to predominate the search and exploitation around the respective cognitive memories. Probably, the social memory is still not present in the global optimal region. Therefore, decreasing $c_2$ slowly can avoid the lock in local optimal. Furthermore, an exploitation state in more likely to occur after an exploration state and before a convergence state. Hence, changing directions for $c_1$ and $c_2$ should be slightly altered from the exploration state to the convergence state \cite{APSO:Zhan2009}.

In the \emph{Exploration} state, the $c_1$ value is increased and the $c_2$ value is decreased. In this state, is essential the swarm should explore so many optima search regions as possible. hence, increasing $c_1$ and decreasing $c_2$ can help each particle explores individually and achieve its own historical best positions. In this case, we aim to avoid that the current best particle of swarm get stucked in a local optimum.

In the \emph{Jumping out} state, the $c_1$ value is decreased and the $c_2$ value is increased. When the best particle is jumping out of the local optimum toward a better optimum, it is very likely to be far away from the core of the swarm in presented in the Convergence state. As soon as this new region is found by a particle, which becomes the (possibly new) guide, others should follow it and achieve this new region as fast as possible. To guarantee this behavior of the swarm, a large $c_2$ value together with a relatively small $c_1$ value is more indicated.

The incremented or decremented value of acceleration coefficients is called acceleration rate ($\delta$). In the APSO algorithm, this value is a random number uniformly generated  among the interval [0.05, 0.1]. The term ``slightly'' found in strategies of Convergence and Exploitation states, implies in the use of acceleration rate equal to 50\% ($\delta \cdot 0.5$).

In the coefficients adaptation process, they can achieve huge or small values, generating unstable moments in search process. To solve this, the authors of APSO proposed a normalization of acceleration coefficient values. The lower and upper boundaries of acceleration coefficients are $c_{min} = 1.5$ and $c_{max} = 2.5$, respectively. If the sum of $c_1$ and $c_2$ is larger than 4.0, the acceleration coefficients should be normalized according to Equation (\ref{eq:APSO_normalized}).
\begin{equation}\label{eq:APSO_normalized}
c_i = \frac{c_i(c_{min} + c_{max})}{c_1 + c_2}, \ \ i = 1,2.
\end{equation}

\subsection{Elitist Learning Strategy}
The third step is the application of a operator in order to generate diversity. The authors named it as Learning Strategy using Elitism and this mechanism aims to improve the global search capability of the algorithm. It was first proposed to be applied just in the best particle during the Convergence state, generating the Jumping out state.

The reason for this operator is because the best particle does not have exemplars to follow. As a means to improve its own, a perturbation was developed to help the best particle to push itself out of a local minima to a potential better search region. If the new region is better, then the rest of swarm will follow quickly the leader in order to converge to this new search region.

This is a type of greedy local search applied only in one dimension ($d$) of the current best particle ($\vec n_i (t)$) of the swarm aiming to allow this particle to escape from a local optimum. All the dimensions have the same probability to be chosen. The Gaussian mutation is generated according to Equation (\ref{eq:APSO_elitism}).
\begin{equation} \label{eq:APSO_elitism}
\vec n_i(t+1) = \vec n_i(t) + (X_{max}^d - X_{min}^d)G(\mu,\sigma^2),
\end{equation}
in which $(X_{max}^d, X_{min}^d)$ are the boundaries of search space, $G(\mu,\sigma^2)$ is a random number generated by Gaussian distribution with a zero mean $\mu$ and standard deviation $\sigma$, which is called the elitist learning rate. The authors suggested that $\sigma$ be linearly decreased with the iterations number and is calculated by Equation (\ref{eq:apso_sigma}).
\begin{equation} \label{eq:apso_sigma}
\sigma = \sigma_{max} - \Bigl[(\sigma_{max} - \sigma_{min})\frac{g(t)}{g_{final}}\Bigr],
\end{equation}
in which ($\sigma_{max},\sigma_{min}$) are the boundaries of $\sigma$, $g(t)$ is the current iterations and $g_{final}$ is the total number of iterations. In empirical tests, Zhan \emph{et al.} proposed to use $\sigma$ = 1.0 in the beginning of the simulations, with the objective to escape from a local optimum and decrease it to $\sigma$ = 0.1 at the end of the simulation, to refine the found solutions.

The position of the best particle is updated only if the new position found by the operator is better than the previous one. Otherwise, the new position will replace the worst particle in the swarm.

\subsection{Adaptation of Inertial Factor}
In the last step, the inertial factor $\omega$ is updated using equation (\ref{eq:APSO_inertial}) in order to auto-adapt the exploration-exploitation ability of the swarm.
\begin{equation}\label{eq:APSO_inertial}
\omega(f_{evol}) = \frac{1}{1 + 1.5e^{-2.6f_{evol}}} \ \ \in \ \ [0.4;0.9], \ \ \forall f_{evol} \ \ \in \ \ [0,1].
\end{equation}

\subsection{Pseudocode of the APSO Algorithm}
The pseudocode of the APSO algorithm is shown in Algorithm \ref{alg:APSO}. The APSO algorithm has the same main steps that are found in the PSO algorithm. However, one can observe the adaptation of parameters (acceleration coefficients and inertial factor) in the lines 3 to 20. In lines 3 to 6, one can observe the steps needed to estimate the evolutionary factor. The classification of evolutionary state of swarm can be observed in lines 7 and 8. In the lines 9 and 10, one can observe the adaptation of acceleration coefficients. In line 11, the adaptation of inertial factor is executed. In the lines 12 to 19, one can observe the execution of elitist learning strategy.

\begin{algorithm}[!h]
    Initialize particles of swarm\;
    \While {the stop criterion is not achieved}{
        \For {each particle}{
            Calculate the average distance according to Equation (\ref{eq:APSO_distance})\;
        }
        Calculate the evolutionary factor according to Equation (\ref{eq:APSO_factor})\;
        Calculate the membership function according to Equations (\ref{eq:s_convergence}), (\ref{eq:s_exploitation}), (\ref{eq:s_exploration}) and (\ref{eq:s_jumpingout})\;
        Classify the swarm in the evolutionary state\;
        Adapt the acceleration coefficients according to Table \ref{tab:apso_strategies}\;
        Normalize the acceleration coefficients according to Equation (\ref{eq:APSO_normalized})\;
        Update the inertial factor according to Equation (\ref{eq:APSO_inertial})\;
        \If{classified on \emph{Convergence} state}
        {
            Generate a new position using the Equations (\ref{eq:APSO_elitism}) and (\ref{eq:apso_sigma})\;
            \If{new position is better than the best cognitive memory}
            {
                Update the position and cognitive memory of the best particle\;
            }
            \Else
            {
                Update the position and cognitive memory of the worst particle\;
            }
        }
        \For {each particle}
        {
            Update the social memory in neighborhood\;
            Update the velocity according to Equation (\ref{eq:PSO_velocity})\;
            Update the position according to Equation (\ref{eq:PSO_position})\;
            Evaluate the position using the objective function of problem\;
            \If{current position is better than cognitive memory}
            {
                Update the cognitive memory\;
            }
            \If{current position is better than social memory}
            {
                Update the social memory\;
            }
        }
    }
    \Return the social memory.
    \caption{Pseudocode of the APSO algorithm.}
    \label{alg:APSO}
\end{algorithm}

\section{Clan Particle Swarm Optimization - ClanPSO}\label{sec:ClanPSO}
Clans are groups of individuals united by a kinship based on a lineage, for example. Some clans stipulate a common ancestor to become the leader of the clan, and every individual of the clan will be guided by this leader. Incorporating this characteristic, a topology was proposed for improving the PSO algorithm performance, the Clan Particle Swarm Optimization \cite{ClanPSO:Carvalho2008}. This topology consists in a set of clan, where each clan is sub-swarm and uses a fully-connected structure to share information. The structure presented in Figure \ref{fig:clans} is an example with four clans (A, B, C and D) of five particles.
\begin{figure}[!h]
\centering
 \includegraphics[width=0.35\textwidth]{image/Clan}
 \caption{\small{Example of division of the population in sub-swarms.}}
 \label{fig:clans}
\end{figure}

Other approach was proposed that enables the migration of particles among clans and some improvements were reached for some benchmark function \cite{ClanPSO:Carvalho2009}. However, there are some problems in this case, such as possibility of empty clans.

In this algorithm, the population size is divided according to the number of particle per clan ($N_{pc}$). For each iteration, each one of clans performs a search using a classical PSO and selects the particle that had achieved the best position of the entire clan. This particle is called the leader and this process of marking is called delegation. After the definition of all leaders, we form a new swarm with only the leaders of clans and we execute the PSO algorithm just with the leaders. This process is called conference of leaders. We can observe in more details these execution steps of the ClanPSO algorithm.

\subsection{Delegation of Leaders}
Is similarly to complex social behavior, with several clans with various leaders. The definition of a leader in clan is a marking process the best particle in the swarm, for example in Figure \ref{fig:leaders}. The delegation process is based in the global information exchange mechanism inside each clan, which uses $\vec{n_{best}}$ information to delegate the leader.
\begin{figure}[!h]
\centering
 \includegraphics[width=0.35\textwidth]{image/Leaders}
 \caption{\small{Example of definition of clans leaders (A, B, C and D).}}
 \label{fig:leaders}
\end{figure}

\subsection{Conference of Leaders}
After the delegation, the leaders need to adjust their positions based on the best leader. This second step consists in execution of PSO only with the leaders, and is called conference of leaders. The conference can be performed using either the star (observe Figure \ref{fig:conference_star}) or ring (observe Figure \ref{fig:conference_ring}) topology.

\begin{figure}[!h]
\centering
\subfigure[]{\includegraphics[scale=0.5]{image/conference_star}\label{fig:conference_star}}
\hspace{1mm}
\subfigure[]{\includegraphics[scale=0.5]{image/conference_ring}\label{fig:conference_ring}}
\caption{Types of conference in the ClanPSO: (a) global conference and (b) local conference.}
\label{fig:conference}
\end{figure}

The topology to be used depends on the kind of problem to be solved. When the star topology is used, the information is spread faster between leaders through of global information share, guaranteeing a better exploitation ability. When is used ring topology, the communication between leaders is slower, thus the algorithm has great exploration ability. The convergence velocity is smaller than in star topology, but the quality of the solutions are often better.

\subsection{The Information shared within the clans}
After the leaders conference, the leaders will return to its respective clans and the new information acquired in the process will be widely used inside each clan to adjust the other particles. Indirectly, allows that all other particles be guided by the best position found within the entire topology. Through indirect communication, a foreign leader does not directly influence another clan, and then preserves the  exploration capacity of clans.

\subsection{Pseudocode of the ClanPSO Algorithm}
The pseudocode of the ClanPSO algorithm is shown in Algorithm \ref{alg:ClanPSO}. In the lines 4 to 14, one can observe the PSO execution within of each clan. In lines 17 to 28, one can observe the leaders conference. We can observe the leaders delegation in the lines 5 and 16, and still the line 5, the return of information from the leaders conference.

\begin{algorithm}[!h]
    Initialize the particles with random positions and velocities\;
    Group the particles in clans with global topology\;
    \While {the stop criterion is not achieved}{
        \For {each clan}{
            Find the best particle and mark as leader\;
            Update the velocity according to Equation (\ref{eq:PSO_velocity})\;
            Update the position according to Equation (\ref{eq:PSO_position})\;
            Evaluate the position using objective function of the problem\;
            \If{new position is better than cognitive memory}
            {
                Update cognitive memory\;
            }
            \If{new position is better than social memory}
            {
                Update social memory\;
            }
        }
        Add clan leader in the leaders list\;
        \For {each leader particle of the list}
        {
            Find the best particle\;
            Update the velocity according to Equation (\ref{eq:PSO_velocity})\;
            Update the position according to Equation (\ref{eq:PSO_position})\;
            Evaluate the position using objective function of the problem\;
            \If{new position is better}
            {
                Update cognitive memory\;
            }
            \If{new position is better than social memory}
            {
                Update social memory\;
            }
        }
    }
    \Return the social memory.
    \caption{Pseudocode of the ClanPSO algorithm.}
    \label{alg:ClanPSO}
\end{algorithm}

\section{Clan Adaptive Particle Swarm Optimization - ClanAPSO}\label{sec:ClanAPSO}
The Clan Adaptive Particle Swarm Optimization (ClanAPSO) was developed in 2011 \cite{ClanAPSO:Pontes2011}. This algorithm was proposed by aggregation of the process of systematic adaptation of parameters with a search strategy of multi-swarm. This technique was inspired in the clans topology presents in the ClanPSO algorithm combined to the concepts of parameters adaptation and elitist strategy found in the APSO algorithm.

The ClanAPSO algorithm presents an adaptation process in each swarm in a multi swarm system, this allows that sub-swarms can perform different types of search by iteration of the algorithm, \textit{i.e.} it is possible to perform exploitation and exploration search simultaneously in distinct regions of the search space. The concept of clans is interesting because it allows to identify local leaders of groups of particles (clans) and carry out the information share between clans indirectly. The indirect communication between clans promotes improvement the search process because it is a distributed process of spreading information.

There are two versions of the ClanAPSO algorithm. The original version \cite{ClanAPSO:Pontes2011} executes the APSO algorithm within the clans and also execute the APSO in the conference of leaders. The proposed second version \cite{ClanAPSO:Vitorino2011} also includes the adaptation capability within the clans, but in the leaders conference is running the PSO algorithm.

\subsection{Pseudocode of the ClanAPSO Algorithm}
The simplified pseudocode of the ClanAPSO algorithm is shown in Algorithm \ref{alg:ClanAPSO}. In the lines 3 to 6, the APSO execution within of each clan is presented. In the lines 7 and 8, one can observe the conference of leaders, where they can be executed by the APSO algorithm or the PSO algorithm. We can observe the leaders delegation in the line 5.

\begin{algorithm}[!h]
    Initialize the particles with random positions and velocities\;
    \While {the stop criterion is not achieved}{
        \For {each clan}{
            Execute APSO within the clans\;
            Delegate leader\;
        }
        Create conference of leaders\;
        Execute APSO (or PSO) with the leaders\;
    }
    \Return the best particle.
    \caption{Pseudocode of the ClanAPSO algorithm.}
    \label{alg:ClanAPSO}
\end{algorithm}

\section{Artificial Bee Colony - ABC}\label{sec:ABC}
The ABC algorithm was proposed by Karaboga in 2005 \cite{ABC:Karaboga2005}. ABC is an algorithm of search and optimization modeled by the behavior of honey bees \cite{ABC:Karaboga2006} \cite{ABC:Karaboga2007} \cite{ABC:Karaboga2009a} \cite{ABC:Karaboga2009b} \cite{ABC:Ziarati2011}. The bee colony is one of the natural societies with the most specialized social divisions. In the simplified model assumed by the ABC, the bee colony is composed by three types of bees: employed bees, onlookers and scouts. Initially, the swarm is divided in equal parts of employed bees and onlookers bees. The employed bees are those that go to the food source explored by herself, there is a food source to each employed bee. The bees that wait in the hive and decide to exploit a food source depending on the information shared by the employed bees are called onlooker bees. The onlookers are guided bees. When the food source does not improve its quality, the associated employed bee is now scout bee, which is responsible to find a new valuable food source. We will call them guide bees in exploration mode.

Basically, the steps of the algorithm executed in each iteration are: (\emph{i}) the employed bees explore its respective food sources; (\emph{ii}) it is determined the quality of food sources, which are shared to onlooker bees; (\emph{iii}) the onlooker bees choose a food source and help to explore the food source selected; (\emph{iv}) if the food source does not improve, the associated employed bee becomes scout bee. The scout bees find randomly new source food and determine its quality. In the end of each iteration, the best food source is saved until the end of the algorithm execution.

A food source represents a potential solution for the problem to be solved. The nectar quantity of a food source determines the solution quality of that food source. The onlooker bees choose the best food source using the selection method by roulette wheel.

The scout bees are explorers of the hive. They do not follow orientations to search food source, then this type of bees aims to find new food sources. The quality of these food source is not relevant (the most of cases is medium or low). However, in some cases, the scout bees find good food sources, that were unknown. The low search cost makes this mechanism of maintain diversity an attractive option.

We can observe that the ABC algorithm has exploitation-exploration tradeoff because the employed and onlookers bees has exploitation capability and scout bees has exploration capability.

\subsection{Quality of Food Source}
The quality of food source is calculated based on objective function value of problem according to Equation (\ref{eq:abc_quality}) \cite{ABC:Murgan2012}.
\begin{equation}
Q[\vec{x}(t)] = \begin{cases}
\frac{1}{f[\vec{x}(t)]+1}, &\mbox{if $ f[\vec{x}(t)] \geq 0 $}, \\
1+abs{f[\vec{x}(t)]+1},    &\mbox{if $ f[\vec{x}(t)] < 0 $}, \\
\end{cases}
\label{eq:abc_quality}
\end{equation}
in which $\vec{x}(t)$ is food source position, $abs$ is value of absolute function, $Q[\vec{x}(t)]$ is the quantity of food source nectar presents in position $\vec{x}(t)$ and $f[\vec{x}(t)]$ is the value of the objective function in position $\vec{x}(t)$.

\subsection{Update of Food Source}
In the algorithm, the search space of the problem is $D$-dimensional. The number of the employed bees and the onlookers are the same and for every food source, there is only one employed bee per food source. In other words, the size of employed and onlooker bees is equal $SN$ (the number of food sources). Each $i$th source food associated to the employed bee will be optimized according to Equation (\ref{eq:ABC_employed}).
\begin{equation}\label{eq:ABC_employed}
v_{id} = x_{id} + r_{id}(x_{id} - x_{kd}),
\end{equation}
in which $d = 1,2,...,D$ is the number of dimensions, $r_{id}$ is a random generalized real number within the range [-1,1], $k = 1,2,...,SN$ is a randomly selected index number in the colony, it has to be different from the $i$. The new solution ($v_{id}$) is compared with the previous one ($x_{id}$), and the better one should be stored.

\subsection{Choose of Food Source by Onlookers Bees}
Next, the onlooker bee needs to select one of the food sources explored by the employed bees. The probability for each food source to be selected by the onlooker bee is:
\begin{equation}\label{eq:ABC_probability}
p_i = \frac{fit_i}{\sum_{j}^{SN}fit_j},
\end{equation}
in which $p_i$ is the probability to select the food source $i$, which is proportional to the quality of the food source, $fit_i$ is the fitness of the position $x_{id}$. Each onlooker bee searchers for a new solution in the selected food source by using Equation (\ref{eq:ABC_employed}).

\subsection{Stagnation of Food Source}
At each loop, the food source are evaluated and if the fitness of a source does not improve after a predetermined number of steps (called \textit{MaxTrial}), then it is abandoned. The employed bee associated with it becomes a scout and replaces the food source with a new one, through the random search given by Equation (\ref{eq:ABC_scout}).
\begin{equation}\label{eq:ABC_scout}
x_{id} =  x_d^{min} + r(x_d^{max} - x_d^{min}).
\end{equation}
in which $r$ is random number selected in range [0,1], and $x_d^{max}$ and $x_d^{min}$ are lower and upper borders in the $d^{th}$ dimension of the search space.

\subsection{Pseudocode of the ABC Algorithm}
The pseudocode of the ABC algorithm is shown in Algorithm \ref{alg:ABC}. In the pseudocode, we can observe the employed bees movement in lines 3 to 13. The probability used to selection of employed bees by onlooker bees can be observed in the line 16. Then, the onlooker bees movement can be observed in lines 18 to 28. If some food source does not improve, so the employed bee becomes a scout bee. This movement can be visualized in lines 29 to 35.
\begin{algorithm}[!h]
    Initialize the food sources in random positions\;
    \While {the stop criterion is not achieved}{
        \For {each employed bee}{
            Determine a different position randomly\;
            Update the position according to Equation (\ref{eq:ABC_employed})\;
            Evaluate the position using the objective function of problem\;
            \If{the new position is better}
            {
                Update the position of associated food source\;
            }
            \Else
            {
                Increase the stagnation counter of food source\;
            }
            Calculate the quality of new food source position according to Equation (\ref{eq:abc_quality})\;
        }
        \For {each employed bee}
        {
            Calculate the probability od roulette wheel selection using the Equation (\ref{eq:ABC_probability})\;
        }
        \For {each onlooker bee}
        {
            Determine two different positions chosen by roulette\;
            Update the position according to Equation (\ref{eq:ABC_employed})\;
            Evaluate the position using the objective function of problem\;
            \If{the new position is better}
            {
                Update the position of associated food source\;
            }
            \Else
            {
                Increase the stagnation counter of food source\;
            }
        }
        \For{each employed bee}
        {
            \If{stagnation counter achieved the threshold}
            {
                Generate a new position of associated food source randomly\;
                Restart the stagnation counter of associated food source\;
            }
        }
    }
    \Return the best food source.
    \caption{Pseudocode of the ABC algorithm.}
    \label{alg:ABC}
\end{algorithm}

\section{Fish School Search - FSS}\label{sec:FSS}
The Fish School Search (FSS) is a computational intelligence technique inspired in social behavior of schools of fish developed by Bastos-Filho and Lima-Neto in 2007 \cite{FSS:Bastos-Filho2008} \cite{FSS:BastosFilho2009}. It was conceived to solve search problems and was based in the gregarious behavior present in some fish species, with the objective to improve the survivability of the entire group through the mutual protection and synergy to perform collective tasks \cite{FSS:Lins2012}.

In the FSS algorithm, the search space, called aquarium, is limited region of objective function, the population is called school of fish and each fish has a weight. The weight of the fish represents the success of search process. Each position in the search space represents a possible solution of the problem.

During the execution, the operators of the algorithm are executed sequentially by updating the positions and weights of each fish. The FSS algorithm has four operators: individual movement (responsible by local search), feeding (indicator of success of search process), collective-instinctive movement (generates the displacement of school of fish) and collective-volitive movement (controls the exploitation-exploration granularity).

\subsection{Individual Movement Operator}
Initially, the fish try to find food and for this, the fish realize individual movements in the search space. The individual movement is executed by each fish in the school ($S$) in the beginning of each iteration.  Each fish chooses a new position in its neighborhood and then, this new position is evaluated using the objective function of problem. The individual movement operator is determined according to Equation (\ref{eq:FSS_individual}) for each dimension.
\begin{equation}\label{eq:FSS_individual}
v_{id}(t) = x_{id}(t) + rand[-1,1] \cdot step_{ind},
\end{equation}
in which $\vec v_i(t)$ is the candidate position of fish $i$, $\vec x_i(t)$ is the current position of the fish $i$, $rand[-1,1]$ is a random number generated by an uniform distribution in the range [-1,1] and $d$ is a number of dimensions. The $step_{ind}$ is a percentage of search space amplitude in dimension determined.

The $step_{ind}$ decreases linearly along the iterations according to Equation (\ref{eq:FSS_stepInd}), so the exploitation capability increases during search process.
\begin{equation}\label{eq:FSS_stepInd}
step_{ind} = step_{ind\_initial} - \Bigl[(step_{ind\_initial} - step_{ind\_end})\frac{g(t)}{g_{end}}\Bigr],
\end{equation}
in which $step_{ind\_initial}$ is the individual step in the beginning of the algorithm execution, $step_{ind\_end}$ is the individual step in the final of the algorithm execution, $g(t)$ is the current iteration and $g_{end}$ is the total number of iterations.

After the calculation of the candidate position, the movement just occurs if the new position has better fitness than the old one.

\subsection{Feeding Operator}
The fish weight can grow or decrease, depending on its success or failure in the search for food, when the individual movement is realized. In each iteration, the fish weight is updated according to Equation (\ref{eq:FSS_feed}).
\begin{equation}\label{eq:FSS_feed}
W_i(t+1) = W_i(t) + \frac{\Delta f_i}{max(\Delta f)},
\end{equation}
in which $W_i(t)$ is the weight of the fish, $\Delta f_i$  is the difference between the fitness value of new position ($f[\vec x(t+1)]$) and the fitness value of the current position for each fish ($f[\vec x(t)]$), this value is calculated according to Equation (\ref{eq:FSS_Delta}). The $max(\Delta f)$ is the maximum value of these differences in the current iteration.
\begin{equation}\label{eq:FSS_Delta}
\Delta f = f[\vec x(t+1)] - f[\vec x(t)].
\end{equation}

In order to avoid a explosion state of the weight values, the algorithm has a maximum value, called weight scale ($W_{scale}$), that is defined to limit the weight of fish. The initial weight for each fish is equal to $\frac{W_{scale}}{2}$.
\subsection{Collective-Instinctive Movement Operator}
After the individual movement, the fish are evaluated and is observed if were successful in the food search or not. Therefore, the school should move based in the successful fish, \textit{i.e.} is a global movement in direction, probably, to richer region of food.

The position of all the fish are updated with this movement according to Equation (\ref{eq:FSS_instinctive}).
\begin{equation}\label{eq:FSS_instinctive}
\vec x_i(t+1) = \vec x_i(t) + \frac{\sum_{i=1}^{N}\Delta \vec x_{ind_i} \Delta f_i}{\sum_{i=1}^{N}\Delta f_i},
\end{equation}
in which $\Delta \vec x_{ind_i}$ is the displacement of the fish $i$ due to the individual movement in the FSS cycle. One must observe that $\Delta \vec x_{ind_i} = 0$ for fish that did not execute the individual movement \cite{FSS:Lins2012}.

\subsection{Collective-Volitive Movement Operator}\label{sse:FSS_volitive}
After the other two movements, the school of fish executes the collective-volitive movement. This movement is global able to make the school fish expand or contract. If the fish school search has been successful, \textit{i.e.} its weight is increasing, the radius of the school should contract to better exploitation capability; if not, it should expand, allowing better exploration capability. Thus, this operator increases the capacity to auto-regulate the exploration-exploitation granularity \cite{FSS:Lins2012}.

To realize the school dilation or contraction in each fish position is necessary the school centroid, which can be evaluated by using the Equation (\ref{eq:FSS_centroid}).
\begin{equation}\label{eq:FSS_centroid}
\vec B(t) = \frac{\sum_{i=1}^{N}\vec x_iW_i(t)}{\sum_{i=1}^{N}W_i(t)}.
\end{equation}

The movement is executed according to Equation (\ref{eq:FSS_volitive}). To the fish school expansion, we use signal `$+$' and to the fish school contraction, we use signal `$-$'.
\begin{equation}\label{eq:FSS_volitive}
\vec x_i(t+1) = \vec x_i(t) \pm step_{vol}r_1 \frac{\vec x_i(t) - \vec B(t)}{d(\vec x_i(t) ,\vec B(t))},
\end{equation}
in which $step_{vol}$ is called volitive step, $r_1$ is a random number generated by uniform probability density function in the range [0,1]. $d(\vec x_i(t) ,\vec B(t))$ calculates the euclidian distance between the particle $i$ and the centroid.

The $step_{vol}$ value decreases linearly along the iterations of the algorithm according to Equation (\ref{eq:FSS_stepVol}). Thus, the algorithm initializes with an exploration ability and changes to an exploitation mode. The parameter value is defined as a percentage of the search space range and is bounded by two parameters ($step_{vol\_initial}$ and $step_{vol\_end}$).
\begin{equation}\label{eq:FSS_stepVol}
step_{vol}(t) = step_{vol\_initial} - \Bigl[(step_{vol\_initial} - step_{vol\_end}) \frac{g(t)}{g_{end}}\Bigr],
\end{equation}
in which $g(t)$ is the current iteration and $g_{end}$ is the total number of iterations. Usually, $step_{vol} = step_{ind}$.

After the update of all the fish, is necessary to evaluate the school fish with objective function of problem, thus they will be prepared for the next iteration.

\subsection{Pseudocode of the FSS Algorithm}
The pseudocode of the FSS algorithm is shown in Algorithm \ref{alg:FSS}. In the lines 3 to 9, we can observe the individual movement that each fish realizes. In the line 17, one can observe the collective-instinctive movement and in the lines 19 to 26, the collective-volitive movement. In the lines 5 and 27, one can observe the evaluation of objective function of problem. %Thus, to comparison of algorithms performance should realize the adjusts suitable of parameter setup.

\begin{algorithm}[!h]
    Initialize the population in random positions\;
    \While {the stop criterion is not achieved}{
        \For {each fish}{
            Execute individual movement according to Equation (\ref{eq:FSS_individual})\;
            Evaluate position using the objective function of problem\;
            \If{new position is better}
            {
                Update the fish memory\;
            }
        }
        Update the individual step according to Equation (\ref{eq:FSS_stepInd})\;
        Calculate the population weight before to adjust it\;
        \For {each fish}
        {
            Adjust the weight according to Equation (\ref{eq:FSS_feed})\;
        }
        Calculate the population weight after of the adjust of the all the fish\;
        \For{each fish}
        {
            Execute the instinctive movement according to Equation (\ref{eq:FSS_instinctive})\;
        }
        Calculate the centroid of the swarm according to Equation (\ref{eq:FSS_centroid})\;
        \For{each fish}
        {
            \If{swarm increased its weight}
            {
                Execute volitive movement of contraction according to Equation (\ref{eq:FSS_volitive})\;
            }
            \Else
            {
                Execute volitive movement of dispersion according to Equation (\ref{eq:FSS_volitive})\;
            }
            Evaluate the position using objective function of problem\;
            \If{new position is better}
            {
                Update the fish memory\;
            }
        }
        Update the volitive step according to Equation (\ref{eq:FSS_stepVol})\;
    }
    \Return the best memory of school of fish.
    \caption{Pseudocode of the FSS algorithm.}
    \label{alg:FSS}
\end{algorithm}

\section{Discussion about the Swarm Intelligence Algorithms}
This chapter described the main swarm intelligence algorithms that were essential to development of proposed algorithm in this dissertation.

\emph{Particle Swarm Optimization - PSO}: this is standard version of the algorithm. It is the most known swarm intelligence algorithm, very simple and easy deployment. This algorithm is very used in unimodal problems due to its convergence ability, but is not recommended  in multimodal problems because the algorithm has limitations to maintain diversity of swarm. Other approaches were developed  to solve these limitations \cite{PSO:Clerc99} \cite{PSO:Clerc2002} as new communications topologies \cite{ClanPSO:Carvalho2008} \cite{ClanPSO:Carvalho2009} \cite{ClanAPSO:Pontes2011} \cite{MultiPSO:Bastos2008}.

\emph{Adaptive Particle Swarm Optimization - APSO}: was proposed to solve two limitations of original PSO, the low convergence velocity and incapability to escape of local minimum. This approach presents a systematic scheme to update the PSO parameters based on an evolutionary factor, that is calculated through distribution of particles in search space. This scheme allows a better control of convergence velocity and ability to escape of local minimum.

\emph{Clan Particle Swarm Optimization - ClanPSO}: a new topology for PSO algorithm compound by particle clans which communicate in conference to share information. The exchange of information indirectly between the clans can offer improved performance of the algorithm, avoiding premature convergence. However, setting the number of particles per clans and clan may depend on the problem.

\emph{Clan Adaptive Particle Swarm Optimization - ClanAPSO}: was introduced the APSO algorithm concepts in each clan. It was used another independent APSO (or PSO) to perform the conference of leaders.  Thus, each clan adapt itself independently and, as a consequence, the algorithm will avoid to exploit excessively the same spot or converge prematurely.

\emph{Artificial Bee Colony - ABC}: was inspired in honey bees behavior and has the ability to explore richer food sources with the onlooker (guided bees) and employed (guide bees in exploitation mode) bees. This exploitation ability is controlled with stagnation counter and the algorithm maintains the diversity of bee colony with the scout bees (guide bees in exploration mode).

\emph{Fish School Search - FSS}: was inspired in school of fish behavior and has two global movements. The first movement moves the school in direction the fish that achieved the best results. The second movement regulates the search granularity, allowing the contraction or expansion of school. Due the presence of two fitness evaluation, this algorithm is slower than other approaches, such as PSO.

Table \ref{tab:Param_Algorithm} shows the number of parameters of each algorithm that need to setup. Some parameters present standard values, but the researcher can change it.

\begin{table}[!h]
\caption{\small{The number of parameters of each algorithm.}}
\centering
\begin{tabular}{>{\centering\arraybackslash}m{1in} | >{\centering\arraybackslash}m{1in} | >{\centering\arraybackslash}m{2in}}
\hline
\textbf{Algorithm}  & \textbf{Number of Parameters} & \textbf{Parameters}\\
\hline
\textbf{PSO}      &   5  & $N$, $c_1$, $c_2$, $\omega_{min}$, $\omega_{max}$\\
\textbf{APSO}     &   5  & $N$, $c_1$, $c_2$, $\omega_{min}$, $\omega_{max}$\\
\textbf{ClanPSO}  &   6  & $N$, $c_1$, $c_2$, $\omega_{min}$, $\omega_{max}$, $N_{pc}$\\
\textbf{ClanAPSO} &   6  & $N$, $c_1$, $c_2$, $\omega_{min}$, $\omega_{max}$, $N_{pc}$\\
\textbf{ABC}      &   2  & $2 \cdot SN$, $MaxTrial$\\
\textbf{FSS}      &   6  & $S$, $W_{scale}$, $step_{vol\_initial}$, $step_{vol\_final}$, $step_{ind\_initial}$, $step_{ind\_final}$ \\
\hline
\end{tabular}
\label{tab:Param_Algorithm}
\end{table}

This dissertation proposes to combine two well known swarm intelligence algorithms: the APSO algorithm and the ABC algorithm. The proposed algorithm has a operator based on bees behavior (the ABC algorithm) to generate diversity for Particle Swarm Optimization with adaptive behavior.
\pagebreak

  \chapter{Metodologia}\label{cap:methodology}

Nessa dissertação, é proposta uma metodologia para a análise de risco em projetos de \textit{software}, a partir de dados históricos de registros de riscos, por meio da utilização de redes neurais artificiais. Para o desenvolvimento dessa metodologia, primeiramente necessita-se de uma base de dados com muitos registros e real de riscos em projetos de \textit{software}, para aumentar a validade do estudo. No entanto, há uma dificuldade em se encontrar publicamente bases de dados representativas e confiáveis. Uma base de dados de risco chamada PERIL \cite{kendrick2003identifying} mostrou atender as necessidades básicas para essa pesquisa, além de já estar totalmente classificada (mais detalhes na Seção \ref{sec:perildataset}). Nesse estudo, precisa-se de uma base de dados extensa de registros de riscos oriundos de diversos projetos espalhados pelo mundo e em diversos períodos de tempo, com o objetivo de mostrar evidências de quais são os principais modos de falha em projetos e um padrão para explorar respostas aos riscos.

Segundo, é necessário realizar um pré-processamento dos dados para que as variáveis de entradas sejam transformadas, normalizadas e selecionadas para o estudo. Terceiro, deve-se avaliar o desempenho das ferramentas de estado da arte quanto ao erro de previsão: Simulação de Monte Carlo e Análise PERT.

Em quarto lugar, deve-se implementar modelos de previsão baseados em RNAs para que se possa selecionar o melhor modelo dentre estes: Perceptron de Múltiplas Camadas, Máquinas de Vetor de Suporte, Redes de Função de Base Radial e Sistema de Inferência Adaptativo \textit{Neuro-Fuzzy}. Alguns desses modelos selecionados apresentam alguns parâmetros e devido à diversidade de possíveis valores para cada parâmetro, necessita-se otimizar os parâmetros dos modelos. Nesse estudo, implementa-se uma variação da meta-heurística Otimização por Enxame de Partículas (PSO - \textit{Particle Swarm Optimization}) com coeficiente de constrição de Clerk \cite{engelbrecht2007computational} para a realização da tarefa de otimização dos parâmetros das RNAs. O espaço de busca do problema de otimização desses parâmetros é multi-dimensional, complexo e apresenta diversos mínimos locais. 

Na Figura \ref{fig:method2}, observa-se um esquema ilustrando um algoritmo de otimização que executa os quatro modelos em suas diversas configurações paramétricas para possibilitar a escolha do modelo mais eficiente para a base de dados adotada.

\begin{figure}[h]
	\centering
	\includegraphics[width=.45\textwidth]{image/MetodologiaDissertacao2.png}
	\caption{Esquema de Seleção da Melhor Rede Neural Artificial.s}
	\label{fig:method2}
\end{figure}

Em quinto lugar, um teste de validação dos resultados é realizado para verificar a eficiência dos modelos baseados em redes neurais artificiais em relação aos modelos de estado da arte em análise de riscos, tais como os modelos baseados em Simulação de Monte Carlo, Análise PERT, Modelo de Regressão Linear Múltipla, Modelo de Regressão em Árvore. Como não existe estudos anteriores para a estimativa do impacto para essa base de dados, os dois últimos modelos de regressão linear foram considerados como linha de base, por serem modelos mais simples. Os dois últimos modelos foram implementados e também são avaliados quanto ao erro na previsão.

As ferramentas utilizadas nesse estudo são Weka, Matlab, o Excel e o programa R. Nessas ferramentas existem bibliotecas de programas que implementam os modelos de regressão linear utilizados e as redes neurais utilizadas. No Excel implementamos a Análise PERT e a Simulação de Monte Carlo.

A seleção do melhor modelo é feita em termos da precisão na estimativa dos impactos dos riscos. É difícil obter uma métrica representativa da precisão de um modelo, no entanto, Engelbrecht \cite{engelbrecht2007computational} e Saxena \cite{saxena2012software} sugerem que a Raiz do Erro Médio Quadrático (REMQ) seja uma medida conveniente e aplicável para a maioria dos problemas. A precisão, nesse caso, significa o grau de proximidade de uma saída calculada para a esperada. A REMQ é representada pela Equação \ref{eq:RMSE}:

\begin{equation}\label{eq:RMSE}
    REMQ= \sqrt[2]{\frac{1}{n}\sum_{i=1}^{n} (e_i)^2} ,
\end{equation}
onde $e_i=f_i - y_i$, $f_i$ é o resultado calculado, $y_i$ é o resultado esperado e $n$ é o número de tuplas de dados.
%where $e_i=f_i - y_i$, $f_i$ is the calculated outcome, $y_i$ is the expected outcome and $n$ is the number of data pairs.

Todas as técnicas estudadas nesse trabalho estimam a saída para o impacto de riscos, a REMQ é calculada trinta vezes para cada abordagem e um Teste Estatístico de Wilcoxon Não-pareado \cite{siegel1956nonparametric} pode ser necessário para determinar qual é uma abordagem mais precisa para a base de dados (nesse estudo, o PERIL). Esse teste estatístico é utilizado porque não há evidências que as amostras sejam oriundas de uma população normalmente distribuída, como também não há relação de ordem nos valores pertencentes às amostras.
%The eight selected techniques have predicted the outcome to risk impacts. Root Mean Square Error was calculated thirty times for each method. Nevertheless, a Non-paired Wilcoxon Test \cite{siegel1956nonparametric} may be necessary to assert which is a more efficient approach to fit the data set (e.g. PERIL). Non-paired Wilcoxon Test is used because there were no evidence that the samples came from a normally distributed population, either there were no relation between outcomes from different samples.

A validação cruzada \cite{amari1996statistical} é utilizada para evitar a ocorrência de \textit{overfitting} ou \textit{underfitting} durante o treinamento das RNA's. Nesse caso, um treinamento com parada prematura é utilizado para identificar o início do \textit{overfitting}, já que esse método tem provado ser capaz de melhorar a capacidade de generalização da RNA em comparação com o treinamento exaustivo \cite{haykin-1994} \cite{engelbrecht2007computational} \cite{amari1996new}. Portanto, o método de validação cruzada é utilizado para cada abordagem, com exceção da Simulação de Monte Carlo e Análise PERT, por promover uma maior capacidade de generalização. Quando se adota o uso da validação cruzada, é necessário o particionamento da base de dados em três partes: conjunto de treinamento, conjunto de validação cruzada e conjunto de testes. O conjunto de treinamento é utilizado na fase de treinamento das RNA's, momento em que o aprendizado ocorre. O conjunto de validação cruzada é processado no mesmo tempo junto com o conjunto de treinamento e determina a parada no treinamento. Já o conjunto de testes, é utilizado para produzir a métrica de precisão do modelo (REMQ).
%Furthermore, cross-validation \cite{amari1996statistical} must be used to avoid the occurrence of overfitting of data training. For instance, \textit{early stopping} training was used to identify the beginning of overfitting because this method has been proved to be capable of improving the generalization performance of the ANN over exhaustive training \cite{haykin1994neural} \cite{amari1996new}. Therefore, cross-validation method are used for each alternative, excluding Monte Carlo Simulation, to promote higher generalization performance.

Por fim, a previsão do impacto do risco e a definição de um intervalo de confiança para uma amostra do conjunto de treinamento são obtidas utilizando o modelo de previsão mais preciso após os testes de validação. É necessário estabelecer um intervalo de confiança da previsão para que os gerentes de projetos e analistas de risco possam estabelecer o nível de confiança de acordo com a sua necessidade. Afinal de contas, esse é o resultado que eles esperam.

A Figura \ref{fig:method} apresenta um fluxograma com a metodologia estabelecida para esse estudo. O procedimento inicia-se com a seleção da base de dados utilizada como conjunto de dados de entrada, nesse caso o PERIL, e finaliza com a atividade ``Definição do Intervalo de Confiança". Todas as atividades são executadas sequencialmente, exceto as atividades ``Avaliação dos Modelos de Estado da Arte" e ``Seleção da Melhor Rede Neural Artificial" que são executadas paralelamente.

\begin{figure}[h]
	\centering
	\includegraphics[width=.45\textwidth]{image/MetodologiaDissertacao.png}
	\caption{Fluxograma do estudo realizado.}
	\label{fig:method}
\end{figure}

\section{Base de dados PERIL}
\label{sec:perildataset}

%A better risk management starts identifying potential problems, asserted here as risk factors. The adoption of available methods like: reviewing lessons learned, brainstorming, interviews and specialized judgment are relative efficient alternatives, otherwise in most of situations it involves high costs. A low cost, extensive and accessible proposal is to use Project Experience Risk Information Library (PERIL) database \cite{kendrick2003identifying}. The  PERIL database provides information and experience of other project risk management.
Um melhor gerenciamento de riscos se inicia com a identificação de problemas potenciais, atribuído como fatores de risco. A adoção dos métodos disponíveis como: revisar lições aprendidas, \textit{brainstorming}, entrevistas e opinião especializada são alternativas relativamente eficientes, no entanto, na maioria das situações elas envolvem alto custo. Uma proposta de baixo custo, extensiva e acessível é utilizar a base de dados \textit{Project Experience Risk Information Library}(PERIL) \cite{kendrick2003identifying}. A base de dados PERIL provê informações e experiências de outras gestões de risco de projetos.

%For more than a decade, in Risk Management Workshops, Kendrick have collected anonymous data from hundred of project leaders dealing with their past project problems. He has compiled this data in the PERIL database, which summarizes both a description of what went wrong and the amount of impact it had on each project. The dataset provides a sobering perspective on what future projects may face and is valuable in helping to identify at least some of what might otherwise be invisible risks or black swans \cite{kendrick2003identifying}.
Por mais de uma década, durante \textit{Workshops} de Gerenciamento de Riscos, Kendrick coletou dados anônimos de centenas de líderes de projetos lidando com seus problemas em projetos passados. Ele compilou essas informações na base de dados PERIL, que sumariza tanto uma descrição do que houve de errado quanto o impacto que ele teve em cada projeto. A base provê uma perspectiva preocupante do que projetos futuros podem enfrentar e é valiosa na ajuda para a identificação do que pelo menos poderiam ter sido riscos invisíveis ou \textit{black swans}, isto é, riscos com grande impacto, difíceis de prever e com ocorrência rara \cite{kendrick2003identifying}.

%In projects, the identified risks can be classified as "known", those anticipated during planning, or "unknown", further identified during project execution. The purpose of this dataset is to provide a framework to identify risks, in such a way to increase the number of "known", and decrease the amount of "unknown" risks.
Segundo Kendrick, em projetos, os riscos identificados podem ser classificados como ``conhecidos", aqueles antecipados durante o planejamento, ou ``desconhecidos", identificados futuramente durante a execução do projeto. O propósito dessa base de dados é prover um \textit{framework} para identificar riscos, de modo a aumentar o número de riscos "conhecidos" e diminuir o número de riscos "desconhecidos" \cite{kendrick2003identifying}.

%Some characteristics of PERIL are: 
%\begin{itemize}
%\item the data are not relational, they contain only a small fraction of the tens of thousands projects undertaken by the project leaders from whom they were collected;
%\item they present bias, the information was not collected randomly;
%\item they represent only the most significant risks;
%\item they are worldwide, with a majority from the Americas;
%\item they do not identify opportunities; 
%\item they contain six hundred and forty nine registers, whose relative impact is based on the number of weeks delayed the project schedule;
%\item typical project had a planned duration between six months and one year;
%\item typical staffing was rarely larger than about twenty people.
%\end{itemize}
Algumas características observadas na base de dados PERIL são:
\begin{itemize}
\item Os dados não estão relacionados, eles representam somente uma pequena fração de dezenas de milhares de projetos realizados por líderes de projetos, cujos riscos foram coletados;
\item Apresentam viés, a informação não foi coletada aleatoriamente;
\item Representam somente os riscos mais significativos;
\item Foram coletadas no mundo todo, majoritariamente nas Américas;
\item Não identificam oportunidades; 
\item Contém 772 registros, cujo impacto relativo é baseado no número de semanas de atraso no cronograma;
\item Projetos comuns tiveram um cronograma inicialmente planejado entre seis meses e um ano;
\item Tamanho da equipe raramente maior do que vinte pessoas.
\end{itemize}

%Risk registers are categorized as scope, schedule and resource. Scope is decomposed in change and defect subcategories. Schedule is decomposed in dependency, estimative and delay subcategories. Resources is decomposed in money, outsourcing and people subcategories. One benefit of PERIL is that the author contemplates "black swans": risks with large impact, difficult to predict and with rare occurrence \cite{taleb2001fooled}. 
Os registros de risco são categorizados em escopo, cronograma e recurso. O escopo é decomposto nas subcategorias mudança e defeito. O cronograma é decomposto em dependência, estimativa e atraso. Os recursos são decompostos em dinheiro, \textit{outsourcing} e pessoas. Um benefício da PERIL é que o autor contempla ``\textit{black swans}" que são riscos com grande impacto, difíceis de prever e de rara ocorrência \cite{taleb2001fooled}.

%Kendrick chose to normalize all the quantitative data using only time impact, measured in weeks of project slippage. This tactic made sense in light of today's obsession with meeting deadlines, and it was an easy choice because by far the most prevalent serious impact reported in this data was deadline slip. Focusing on time is also appropriate because among the project triple constraints of scope, time ans cost, time is the only one that's completely out of our control - when it's gone, it's gone \cite{KEND2003BOOK}.
Kendrick decidiu padronizar o impacto usando como métrica o tempo, medido em semanas de atraso no projeto. Essa estratégia faz sentido à luz da obsessão de hoje por cumprimento de prazos e foi uma escolha fácil, pois, de longe, o impacto mais sério relatado nesses dados foi o atraso no prazo. Focar-se em tempo também é apropriado porque entre as restrições triplas de projeto - escopo, tempo e custo -, tempo é o único que está completamente fora de nosso controle. Afinal, o tempo sempre passa e quando não se é utilizado, não é possível voltar atrás \cite{KEND2003BOOK}.

%\begin{table}[h]
%\caption{Raw project numbers in the PERIL database}\label{tab:peril_numbers} \centering
%\begin{tabular}{|l|c|c|c|c|}
% \hline
% \multicolumn{1}{|c|}{} &Americas &Asia &Europe/Middle East &Total \\
% \hline
%  IT/Solution Project & 256 & 57 & 18 & 331 \\
%  Product Development Project & 224 & 66 & 28 & 318 \\
% \hline
%  \textbf{Total} & \textbf{480} & \textbf{123} & \textbf{46} & \textbf{649} \\
% \hline
%\end{tabular}
%\end{table}
\begin{table}[h]
\caption{Número bruto de projetos no PERIL}\label{tab:peril_numbers} \centering
\begin{tabular}{|l|c|c|c|c|}
 \hline
 \multicolumn{1}{|c|}{} & Américas & Ásia & Europa/Oriente Médio & Total \\
 \hline
  Projeto de TI/Soluções & 280 & 77 & 37 & 394 \\
  Projeto de Des. de Produto & 249 & 86 & 43 & 378 \\
 \hline
  \textbf{Total} & \textbf{529} & \textbf{163} & \textbf{80} & \textbf{772} \\
 \hline
\end{tabular}
\end{table}

Na Tabela \ref{tab:peril_numbers}, apresenta-se o número de projetos em TI e de desenvolvimento de produtos nas Américas, Ásia e Europa.

\begin{table}[h]
\caption{Impacto total de projetos pelas causas-raiz de categorias e subcategorias \cite{KEND2003BOOK}}\label{tab:peril_pareto} \centering
\begin{tabular}{|l|c|c|c|c|}
 \hline
 Subcat. & & & Impacto & Impacto \\
 Causas-raiz & Definição & Casos & Cumul. & Médio \\
  &  &  & (sem.) & (sem.) \\
 \hline
  Escopo: & Revisão no escopo &  &  &  \\
  Mudanças & durante o projeto & 177 & 1460 & 8,2 \\
 \hline
  Recurso: &  &  &  &  \\
  Pessoas & Problemas de relacionamento interno & 123 & 706 & 5,7 \\
 \hline
  Escopo: & Falha em alcançar  &  &  &  \\
  Defeito & requisitos de entrega & 93 & 654 & 7,0 \\
 \hline
  Cronograma: & Atraso devido a fatores &  &  &  \\
  Atrasos & sob o controle do projeto & 102 & 509 & 5,0 \\
 \hline
  Cronograma: & Durações inadequadas alocadas &  &  &  \\
  Estimativas & para atividades do projeto & 49 & 370 & 7,6 \\
 \hline
  Recurso: &  &  &  &  \\
  \textit{Outsourcing} & Problemas de relacionamento externo & 47 & 316 & 6,7 \\
 \hline
  Cronograma: & Atraso no projeto devido &  &  &  \\
  Dependências & a fatores externos & 41 & 262 & 6,4 \\
 \hline
  Escopo: &  &  &  &  \\
  Mudanças & Financiamento insuficiente & 17 & 228 & 13,4 \\
 \hline
\end{tabular}
\end{table}
%\begin{table}[h]
%\caption{Total project impact by root-cause categories and subcategories \cite{KEND2003BOOK}}\label{tab:peril_pareto} \centering
%\begin{tabular}{|l|c|c|c|c|}
 %\hline
 %Root-Cause & & &Cumulative &Average \\
 %Subcategories &Definition &Cases &Impact(weeks) &Impact(weeks) \\
 %\hline
 % Scope: & Revision made to scope &  &  &  \\
 % Changes & during the project & 177 & 1,460 & 8.2 \\
 %\hline
 % Resource: &  &  &  &  \\
 % People & Issues arising from internal staffing & 123 & 706 & 5.7 \\
 %\hline
 % Scope: & Failure to meet deliverable &  &  &  \\
 % Defects & requirements & 93 & 654 & 7.0 \\
 %\hline
 % Schedule: & Project slippage due to factors &  &  &  \\
 % Delays & under the control of the project & 102 & 509 & 5.0 \\
 %\hline
 % Schedule: & Inadequate durations allocated &  &  &  \\
 % Estimates & to project activities & 49 & 370 & 7.6 \\
 %\hline
 % Resource: &  &  &  &  \\
 % Outsourcing & Issues arising from external staffing & 47 & 316 & 6.7 \\
 %\hline
 % Schedule: & Project slippage due to factors &  &  &  \\
 % Dependencies & outside the project & 41 & 262 & 6.4 \\
 %\hline
  %Scope: &  &  &  &  \\
  %Changes & Insufficient project funding & 17 & 228 & 13.4 \\
% \hline
%\end{tabular}
%\end{table}

A Tabela \ref{tab:peril_pareto} apresenta o número de casos, o impacto cumulativo e médio em semanas para cada categoria e sub-categoria de causa-raiz, além do significado de cada subcategoria.

Uma desvantagem dessa base de dados é que ela somente contabiliza riscos que tiveram impacto negativo no projeto. As oportunidades não foram identificadas e analisadas nesse estudo. No entanto, um dos grandes benefícios é que o autor apresenta alguns riscos como \textit{black swans} \cite{KEND2003BOOK}, então utilizando a PERIL responde-se a primeira questão dessa pesquisa. Se o risco tiver impacto negativo é conhecido como catástrofe, ao passo que, se tiver impacto positivo é conhecido como recompensa.
%A disadvantage of this database is that it only accounts for risks that negatively impact on the project. The opportunities were not identified and maximized in that study. However, A major benefit is that the author presents some risks as black swans \cite{KEND2003BOOK}: representing the idea of risks with broad impact, hard to predict and rare to occur. If the risk has negative impact, is known as a catastrophe, whereas, if you have positive impact, is known as a reward.

\subsection{\textit{Black Swans}}

%Calling some risks "black swans" has been popularized of late by the writings of Nassim Nicholas Taleb \cite{taleb2001fooled}. The notion of a "black swan" originated in Europe before there was much knowledge of the rest of the world. Because all the swans observed in Europe were white, a black swan was deemed impossible. It came as something of a shock when a species of black swans was later discovered in Australia. This realization gave rise to the metaphorical use of the term "black swan" to describe something erroneously believed to be impossible.
Denominar alguns riscos como \textit{black swans} têm sido popularizado desde os textos de Nassim Nicholas Taleb \cite{taleb2001fooled}. A noção de \textit{black swan} originou-se na Europa antes de ser popularizada pelo Mundo. Já que todos os cisnes observados na Europa eram brancos, um cisne negro era considerado impossível de existir. Porém, foi como um choque quando uma espécie de cisne negro foi descoberta na Austrália. Esse fato deu origem ao uso metafórico do termo \textit{black swan} para descrever algo erroneamente acreditado ser impossível.

%Taleb's concept of "black swan" is a large-impact, hard-to-predict, rare event. It is nonetheless applicable to project risk management. In projects, it is common for project leaders to discount major project risks because they are estimated to have extremely low probabilities. But these risks do occur - The PERIL database is full of them - and the severity of problems they cause means that ignoring them can be unwise. When these risks do occur, the same project managers who initially dismissed them come to perceive them as much more predictable - sometimes even inevitable \cite{KEND2003BOOK}.
O conceito de Taleb acerca de \textit{black swan} define-o como um evento raro, difícil de prever e de grande impacto. Mas não deixa de ser aplicável a gestão de risco do projeto. Nos projetos, é comum que os líderes de projeto descartem os principais riscos do projeto, devido as suas probabilidades serem extremamente baixas. No entanto, esses riscos ocorrem - a PERIL é cheia deles - e a severidade dos problemas que eles causam significa que ignorá-los pode ser imprudente. Quando esses riscos ocorrem, o mesmo gerente de projetos que inicialmente os negaram começam a percebê-los como muito mais previsíveis - às vezes até mesmo inevitáveis \cite{KEND2003BOOK}.

%In PERIL database, there are 127 cases representing the most schedule slippage. As the database shows, these most damaging risks are not as rare as might be thought, and they need not be so difficult for project managers to predict if they get appropriate attention in the risk management process \cite{KEND2003BOOK}. In many situations, the most difficult task is to identify and estimate "black swans" due to its characteristic: emergent, unexpected, unpredictable and extreme impact events. Therefore, "black swans" also will be included in this study.
Na PERIL, há cento e vinte e sete casos representando os maiores atrasos no cronograma. Como a base de dados mostra, estes riscos mais danosos não são tão raros quanto devem ser pensados, e não devem ser tão difíceis de ser previstos por gerentes se eles dedicarem a atenção apropriada para o processo de gerenciamento de riscos \cite{KEND2003BOOK}. Em muitas situações, a tarefa mais difícil é identificar e estimar \textit{black swans} devido a sua característica: emergente, inesperado, imprevisível e com alto impacto. Portanto, \textit{black swans} também são considerados nesse estudo.

\section{Pré-processamento dos Dados}
\label{sec:datapreprocessing}

%PERIL contains nominal and numeric values. So, nominal variables were expressed through binary variables. In that point, it is used fifteen binaries variables to represent nine nominal variables. Second, impact which represents the real output, are integer numbers. It has been noticed that impact probability distribution function fits with log-normal, gamma functions. Therefore, it was done a gamma data normalization \cite{han2006data}.
PERIL contém valores nominais e numéricos. As variáveis nominais foram expressas através de variáveis binárias. Nesse estudo, utilizam-se doze variáveis binárias para representar as variáveis nominais. O impacto que representa a saída esperada, são números inteiros. Foi observado que a função de distribuição de probabilidade do impacto ajusta-se às funções log-normal e gamma. Portanto, foi realizada uma normalização gamma \cite{han2006data}. A seleção das variáveis mais significativas para o estudo foi realizada após o resultado da análise promovida pelo algoritmo \textit{Random Forest} proposto no livro de Luís Torgo \cite{torgo2003data}.

%Figure \ref{fig:input16} e \ref{fig:input712} introduced input variables in histograms. All data are binary values represented by bar graphs, that means the number of occurrences for each value interval. Figure \ref{fig:impacthistogram} presents gamma normalized real outcome from PERIL in a histogram. A shape of the distribution fitting function is also presented in a curve under the histogram.
A Figura \ref{fig:input16} e \ref{fig:input712} apresentam os histogramas das variáveis de entrada. Todos os dados encontram-se binarizados como pode ser observado nos gráficos em barras, que contém o número de ocorrências para cada intervalo de valores. A Figura \ref{fig:impacthistogram} apresenta a saída normalizada pela função gamma para a PERIL num histograma. A forma da função de distribuição de probabilidade é exibida como uma curva sobre o histograma.

%\begin{figure}[h]
%  \vspace{-0.2cm}
%  \centering
%  \includegraphics[width=0.7\columnwidth]{image/input1_6.pdf}
%  \caption{First six input variables}
%  \label{fig:input16}
%\end{figure}
\begin{figure}[h]
  \vspace{-0.2cm}
  \centering
  \includegraphics[width=0.7\columnwidth]{image/input1_6.pdf}
  \caption{Primeiras seis variáveis de entrada.}
  \label{fig:input16}
\end{figure}

%\begin{figure}[h]
%  \vspace{-0.2cm}
%  \centering
%  \includegraphics[width=0.7\columnwidth]{image/input7_12.pdf}
%  \caption{Last six input variables}
%  \label{fig:input712}
%\end{figure}
\begin{figure}[h]
  \vspace{-0.2cm}
  \centering
  \includegraphics[width=0.7\columnwidth]{image/input7_12.pdf}
  \caption{Últimas seis variáveis de entrada.}
  \label{fig:input712}
\end{figure}

%\begin{figure}[h]
%  \vspace{-0.2cm}
%  \centering
%  \includegraphics[width=0.5\columnwidth]{image/impact_histogram.pdf}
%  \caption{Histogram of impact and shape of the distribution fitting function}
%  \label{fig:impacthistogram}
%\end{figure}
\begin{figure}[h]
  \vspace{-0.2cm}
  \centering
  \includegraphics[width=0.5\columnwidth]{image/impact_histogram.pdf}
  \caption{Histograma do impacto e forma da função de distribuição de ajuste.}
  \label{fig:impacthistogram}
\end{figure}

%For our purpose, PERIL was split into three disjoint subsets - training, cross-validation and test subsets, corresponding to fifty, twenty-five and twenty five percent of the dataset, respectively. \textit{Split-sample} cross-validation method was used for MLRM and RTM models. Whereas \textit{early stopping} and \textit{split-sample} cross-validation methods were combined and used for MLP, SVM, RBF and ANFIS training \cite{priddy2005artificial}.
Para o nosso objetivo, PERIL foi dividida em três subconjuntos disjuntos - treinamento, validação cruzada e teste, correspondendo a cinquenta, vinte e cinco porcento e o restante da base de dados, respectivamente. O método de validação-cruzada com divisão de amostras foi utilizado tanto para o MRLM e o MRA quanto para o MLP, SVM, RBF e o ANFIS; sendo que para os últimos o método da parada prematura também foi utilizada no treinamento \cite{priddy2005artificial}.

\section{Modelos de Estado da Arte}
\label{sec:models}

Nesta seção, são descritas as configurações adotadas para a experimentação dos modelos de estado da arte.

\subsection{Simulação de Monte Carlo}

%MCS technique used the entire dataset in order to increase the performance prediction. It was filtered only the possible real outcomes to generate the calculated outcome. Towards this decision, we have reduced prediction issues and have improved its performance.
A Simulação de Monte Carlo utilizou a base de dados completa com o objetivo de aumentar o desempenho do modelo durante a previsão. Além disso, foram filtradas somente as saídas possíveis para uma dada entrada para que uma nova saída pudesse ser calculada, através dessa configuração, o desempenho da estimativa do impacto dos riscos foi melhorado. Já que, a SMC é um método estatístico, a previsão de valores a partir de uma amostra pouco representativa como ocorre quando não se filtram somente as saídas possíveis geram previsões mais errôneas.

\subsection{Análise PERT}

A Análise PERT também utilizou a base de dados completa e somente as saídas possíveis para uma determinada entrada foi utilizada no cálculo da nova saída, tal qual a SMC. A partir dessa configuração o resultado desse modelo foi melhorado, em comparação com o cenário em que a base de dados não foi filtrada.

\section{Modelos de Regressão Linear}

Nesta seção, são descritas as configurações adotadas para a experimentação com os modelos de regressão linear. 

%The source code of MLR models were adapted from Torgo \cite{torgo2003data} in order to perform linear regression model training, cross-validation, outcome prediction and MAE evaluation. MLR and RTM models were analyzed statistically to define the baseline linear regression model for further analysis.
O código-fonte dos modelos de regressão linear foram adaptados de Torgo \cite{torgo2003data} para realização do treinamento, validação cruzada, teste e avaliação da Raiz do Erro Médio Quadrático. Os modelos MRLM e MAR foram comparados estatisticamente para que se pudesse definir um modelo de regressão linear padrão para estudos futuros com a base de dados correspondente.

Um estudo com esses modelos de regressão linear são necessários porque não há um estudo base para a estimativa de impactos de risco utilizando a PERIL. Portanto, durante os experimentos, a avaliação do melhor modelo de regressão linear tem como objetivo definir um limite superior de valores da REMQ. Isso significa que os modelos que obtiverem erros acima desse limite superior são insatisfatórios.

\subsection{Modelo de Regressão Linear Múltipla}

O MRLM é um  modelo de regressão linear mais simples para a estimativa do impactos dos riscos apresentados na PERIL. Após a otimização do Modelo de Regressão Linear através da seleção das melhores variáveis de entrada utilizando o critério \textit{Akaike Information Criterion}(AIC) sete das onze variáveis foram selecionadas, além do termo independente.

\subsection{Modelo de Regressão em Árvore}

O modelo MRA constrói uma árvore de classificação das variáveis de entrada, podendo não ser necessário utilizar todas as variáveis de entrada para a construção do modelo. Ele tenta obter erros de previsão menores que o MRLM através da seleção das variáveis de entrada que têm maior correlação linear com a saída. Na Seção \ref{cap:experiments}, três modelos de árvore de regressão foram analisados com o modelo de regressão linear múltipla.

\section{Otimização por Enxame de Partículas}

Para o PSO, os parâmetros descritos na Tabela \ref{tab:pso_configuration} foram utilizados. Esses parâmetros foram definidos experimentalmente. A variação do PSO utilizada é aquela que implementa o coeficiente de constrição de Clerk \cite{clerc1999swarm}, como definido no início desse capítulo.

\begin{table}[h]
\caption{Parâmetros do PSO.}\label{tab:pso_configuration} \centering
\begin{tabular}{|c|c|}
  \hline
  Parameter & Value \\
  \hline
  Coeficiente Cognitivo & 2,05 \\
  \hline
  Coeficiente Social & 2,05 \\
  \hline
  Fator de Inércia & 0,8 \\
  \hline
  Número de partículas & 30 \\
  \hline
  Número de ciclos & 600 \\
  \hline
\end{tabular}
\end{table}

Cada partícula no PSO representa uma configuração candidata. A função de avaliação das partículas é a REMQ de trinta avaliações da rede neural artificial analisada cujos parâmetros são cada partícula. Os parâmetros das redes MLP, SVM e RBF são determinados após a execução desse algoritmo.

\section{Redes Neurais Artificiais}
\label{sec:rnas}

Nesta seção, são descritas as configurações das redes neurais artificiais utilizadas nesse estudo.

\subsection{MLPs}

%A three layered back-propagation MLP model was established to model risk impact predictor. That model consists of one input layer, one hidden layer, and one output layer. The input layer had thirteen neurons, which represent the twelve independent variables plus the bias. The output layer has one neuron, which represents the single impact outcome. The transfer function in hidden and output layer was sigmoid-logistic. The architecture of the MLP is demonstrated in Figure \ref{fig:mlpmodelstudy}.
Algumas variações do modelo da MLP foram utilizadas nesse estudo. Diversos parâmetros podem ser alterados tais como a quantidade de camadas escondidas, o número de neurônios escondidos em cada camada escondida, a taxa de aprendizado, o momento, o número máximo de ciclos de treinamento e a regra de aprendizado. Um modelo MLP utilizado nesse estudo é apresentado na Figura \ref{fig:mlp_example}. Nesse modelo tem-se dez neurônios escondidos na única camada escondida.

\begin{figure}[!h]
  \vspace{-0.2cm}
  \centering
  \def \svgwidth{0.55\columnwidth}
  \input{image/mlp.pdf_tex}
  \caption{Um modelo MLP utilizado no estudo.}
  \label{fig:mlp_example}
\end{figure} 

%The number of neurons in the hidden layer and other paramenters were determined by trial and error, a fast approach aiming to achieve a more accurate performance of MLP. For the analysis, the maximum training epochs has been set at six hundreds.
A quantidade de camadas escondidas estudadas foram uma e duas camadas. O número de neurônios na(s) camada(s) escondida(s), a taxa de aprendizado e o momento foram determinados por um algoritmo de otimização como o PSO para aumentar a precisão na estimativa de erros. Para as análises, o número máximo de ciclos de treinamento foi configurado para seiscentos.

%Learning rate, momentum and neurons in hidden layer varied from values presented in Table \ref{tab:mlp_configuration_investigation}. A better parameters configuration solution is shown in Table \ref{tab:mlp_best_configuration}. Figure \ref{fig:mlpmodelstudy} presents MLP model with the better configuration for PERIL. The model contains ten neurons in hidden layer.
A taxa de aprendizado, o momento e a quantidade de neurônios na camada escondida variam de acordo com os valores apresentados na Tabela \ref{tab:mlp_configuration_investigation}.

%\begin{table}[h]
%\caption{Parameters intervals to MLP model.}\label{tab:mlp_configuration_investigation} \centering
%\begin{tabular}{|c|c|c|}
%  \hline
%  Parameter & Min. Value & Max. Value \\
%  \hline
%  Momentum & 0.1 & 0.9 \\
%  \hline
%  Learning rate & 0.1 & 0.9 \\
%  \hline
%  Hidden Neurons & 1 & 100 \\
%  \hline
%\end{tabular}
%\end{table}
\begin{table}[h]
\caption{Intervalos de parâmetros para a MLP.}\label{tab:mlp_configuration_investigation} \centering
\begin{tabular}{|c|c|c|}
  \hline
  Parâmetros & Valor Mínimo & Valor Máximo \\
  \hline
  Momento & 0,1 & 0,9 \\
  \hline
  Taxa de Aprendizado & 0,1 & 0,9 \\
  \hline
  Neurônios Cam. Escondida & 1 & 100 \\
  \hline
\end{tabular}
\end{table}

Por fim, as regras de aprendizado utilizadas nesse estudo são \textit{Backpropagation}, Levenberg-Marquardt, BFGS Quasi-Newton, \textit{Resilient Backpropagation}, \textit{Polak-Ribiére Conjugate Gradient}, Gradiente Conjugado Escalonado e \textit{One Step Secant}.

Em particular, uma MLP, chamada ``MLPRegressor", que tem uma camada escondida e cuja regra de aprendizado tem o objetivo de minimizar o erro médio quadrático mais uma penalidade quadrática através do método BFGS Quasi-Newton teve um melhor desempenho que as demais variações.

\subsection{SVM}

%RegSMOImproved is the optimization algorithm and PolyKernel is the kernel function as described in \cite{Shevade1999}. 
O algoritmo SVM para regressão utilizado é o SMOReg. Nesse algoritmo RegSMOImproved é o algoritmo de otimização e PolyKernel é a função de kernel como descrito em \cite{Shevade1999}. O pseudo-código para esse algoritmo é apresentado no Algoritmo \ref{alg:pseudocodigoSVM}.


\begin{algorithm}[H]
%\SetAlgoLined
\label{alg:pseudocodigoSVM}
\begin{verbatim}
Begin
     peril <- read_file();
     peril_train <- partition(peril, 0, 50);
     peril_crossvalidation <- partition(peril, 50, 75);
     peril_test <- partition(peril, 75, 100);
     smo <- SMOReg();
     options <- [peril_train, peril_crossvalidation, 
                 RegSMOImproved, PolyKernel]
     SMOReg.runClassifier(smo, options);
     for instance in peril_test:
          calculated <- smo.classifyInstance(instance);
          wished <- instance.classValue();
          REMQ <- REMQ + (wished - calculated)^2
     end
     n <- peril_test.size();
     REMQ <- REMQ/n;
     REMQ <- sqrt(REMQ);
 End
\end{verbatim}     
\caption{Algoritmo do SVM}
\end{algorithm}
\bigskip

%In Algorithm \ref{code:svm}, we read data on file, split data into training, cross-validation and testing subsets, instantiates SMOReg SVM regression model, run model training, generates a calculated outcome, get the correspondent real output in PERIL and calculate MAE.
No Algoritmo \ref{alg:pseudocodigoSVM}, os dados são lidos a partir de um arquivo, dividido nos subconjuntos treinamento, validação cruzada e teste. O modelo de regressão SMOReg é instanciado, o treinamento do modelo é executado, a saída calculada é gerada e é calculado o REMQ a partir da saída real e da saída calculada. Esse algoritmo é utilizado como a função de otimização para o algoritmo de otimização por enxame.

\subsection{RBF}

A rede neural RBF utilizada é a RBFRegressor. Ela minimiza o erro quadrático através do método BFGS. Os centros iniciais das gaussianas são encontrados utilizando SimpleKMeans, um algoritmo que implementa K-Médias. O sigma inicial é configurado para a maior distância entre qualquer centro e o vizinho mais próximo no conjunto de centros. O parâmetro de cume é usado para penalizar o tamanho dos pesos na camada de saída, o qual implementa uma combinação linear simples. O número de funções de base pode também ser especificado. Para esse estudo somente um sigma global é utilizado para todas as funções de base. O pseudo-código para esse algoritmo é apresentado no Algoritmo \ref{alg:pseudocodigoRBF}.

\begin{algorithm}[H]
%\SetAlgoLined
\label{alg:pseudocodigoRBF}
\begin{verbatim}
 Begin
     peril <- read_file();
     peril_train <- partition(peril, 0, 50);
     peril_crossvalidation <- partition(peril, 50, 75);
     peril_test <- partition(peril, 75, 100);
     rbf <- RBFRegressor();
     options <- [peril_train, peril_crossvalidation, peril_test]
     RBFRegressor.runClassifier(rbf, options);
     for instance in peril_test:
          calculated <- rbf.classifyInstance(instance);
          wished <- instance.classValue();
          REMQ <- REMQ + (wished - calculated)^2
     end
     n <- peril_test.size();
     REMQ <- REMQ/n;
     REMQ <- sqrt(REMQ);
 End
\end{verbatim}
\caption{Algoritmo do RBF}
\end{algorithm} 
\bigskip

%In Algorithm \ref{code:svm}, we read data on file, split data into training, cross-validation and testing subsets, instantiates SMOReg SVM regression model, run model training, generates a calculated outcome, get the correspondent real output in PERIL and calculate MAE.
No Algoritmo \ref{alg:pseudocodigoRBF}, os dados são lidos a partir de um arquivo, dividido nos subconjuntos treinamento, validação cruzada e teste. O modelo de regressão SMOReg é instanciado, o treinamento do modelo é executado, a saída calculada é gerada e a REMQ é obtida a partir das saídas real e calculada. Esse algoritmo é utilizado como a função de otimização para o algoritmo de otimização por enxame.

\subsection{ANFIS}

O ANFIS é um sistema \textit{neuro-fuzzy} desenvolvido por Sugeno \cite{jang1997neuro}. Ele utiliza um algoritmo de aprendizado híbrido para identificar parâmetros do sistema de inferência \textit{fuzzy} Sugeno. Ele aplica uma combinação do método dos mínimos quadrados e o método do gradiente descendente \textit{backpropagation} para o treinamento dos parâmetros da função de pertinência do sistema de inferência \textit{fuzzy}. O sistema de inferência \textit{fuzzy} utilizado foi o ``genfis2", já que há um número grande de variáveis de entrada. O pseudo-código para esse algoritmo é apresentado no Algoritmo \ref{alg:pseudocodigoANFIS}.

\begin{algorithm}[H]
%\SetAlgoLined
\label{alg:pseudocodigoANFIS}
\begin{verbatim}
 Begin
     inputs = csvread(peril,0,0,[0,0,648,10])
     targets = csvread(peril,0,11)
     tData = [inputs targets];
     in_fis = genfis2(inputs,targets, 0.7);
     trainOpts = [100,0.1,0.01,0.9,1.1]
     displayOpts = [1,1,1,1];
     chkData = []
     [fis,error,stepsize,chkFis,chkErr] = 
          anfis(tData,in_fis,trainOpts,displayOpts,
          chkData,1);
     for err in error:
          REMQ <- REMQ + (err)^2
     end
     n <- peril.size();
     REMQ <- REMQ/n;
     REMQ <- sqrt(REMQ);
 End
\end{verbatim}
\caption{Algoritmo do ANFIS}
\end{algorithm}
\bigskip

%In Algorithm \ref{code:svm}, we read data on file, split data into training, cross-validation and testing subsets, instantiates SMOReg SVM regression model, run model training, generates a calculated outcome, get the correspondent real output in PERIL and calculate MAE.
No Algoritmo \ref{alg:pseudocodigoANFIS}, os dados de entrada e saída são lidos a partir de um arquivo. O sistema de inferência \textit{fuzzy} é instanciado, o treinamento do modelo é executado, o erro é gerado e a REMQ é obtida a partir das saídas real e calculada. Esse algoritmo é utilizado como a função de otimização para o algoritmo de otimização por enxame.

\pagebreak
  \chapter{Experimentos}\label{cap:experiments}

Os experimentos realizados utilizaram os modelos descritos no Capítulo \ref{cap:methodology} e a base de dados PERIL. Eles foram estabelecidos, primeiramente, analisando os modelos de regressão que foram a linha de base para o estudo, e em seguida as técnicas comumente utilizadas na academia e na indústria foram abordadas. O modelo do estado da arte, o ANFIS, foi analisado e por fim, mas não menos importante, as RNAs apresentadas na Seção \ref{sec:rnas} foram analisadas. Os resultados para cada experimento são apresentados no Capítulo \ref{cap:results}.

O objetivo global é utilizar a metodologia proposta no Capítulo \ref{cap:methodology} para determinar uma abordagem mais precisa para a estimativa do impacto de riscos. A métrica utilizada para atingir esse objetivo é o erro de previsão, REMQ. As questões a serem respondidas foram apresentadas no Capítulo \ref{cap:introduction}.

Três foram as possíveis hipóteses para os resultados dos experimentos:
\begin{itemize}
\item $H_0$: não há diferença entre usar os modelos em Redes Neurais Artificiais e os modelos de Estado da Arte para as estimativas dos impactos de riscos;
\item $H_1$: os modelos em Redes Neurais Artificiais são mais precisos do que os modelos de Estado da Arte para as estimativas dos impactos de riscos;
\item $H_2$: os modelos em Redes Neurais Artificiais são menos precisos do que os modelos de Estado da Arte para as estimativas dos impactos de riscos.
\end{itemize}

Os objetos de controle foram os códigos-fonte de uma MLP e do PSO desenvolvidos para o experimento. Critérios de aleatoriedade, agrupamento e balanceamento foram adotados para facilitar a análise estatística. O objeto experimental foi a base de dados de risco PERIL. Por fim, os resultados foram analisados estatisticamente, através de testes de hipótese que foram conduzido tão logo os resultados foram obtidos.

\section{Pré-processamento}

Antes de se iniciar os experimentos, foi necessário preparar a base de dados PERIL para ser utilizada pelos modelos. Primeiro, inicia-se selecionando os registros de riscos classificados como comuns e \textit{black swam}, totalizando setecentos e setenta e dois registros de riscos. Esses registros foram apresentados no formato de uma tabela extensa completamente classificados através de um critério definido por Kendrick em seu livro \cite{kendrick2003identifying}. A tabela contém oito colunas, dentre as quais uma delas é a descrição do evento de risco ocorrido e outra o impacto do risco; as seis restantes representam as classes definidas por Kendrick que são ``\textit{Parameter}", ``\textit{Category}", ``\textit{Sub category}", ``\textit{Region}", ``\textit{Project}" e ``\textit{Date}". 

A primeira dificuldade enfrentada foi transformar os dados nominais em numéricos. Decidiu-se ``binarizar" os dados de modo, primeiramente, a termos um menor número de variáveis de entrada. Por exemplo, utilizando esse critério para ``binarizar" quatro classes é necessário um número binário de quatro dígitos. Após esse passo, obteve-se uma tabela com doze colunas. É importante lembrar que a coluna de descrição foi desprezada. 

A partir daí, a segunda dificuldade foi identificada que é escolher qual o método de normalização apropriada para que o histograma dos impactos normalizados pudesse se aproximar do histograma da função normal, ou seja, apresentasse a forma de sino. Os métodos analisados foram a normalização linear, normalização estatística, normalização log-normal e normalização gamma. Para auxiliar a escolha do melhor método, investigou-se qual a função de distribuição de probabilidade se ajusta ao histograma dos impactos da PERIL. Nesse caso, verificou-se que as função de distribuição de probabilidade log-normal e gamma se aproximaram mais após uma análise visual. A escolha da normalização gamma foi comprovada, logo em seguida, após a obtenção do histograma dos impactos dos riscos normalizados.

Em seguida, as melhores variáveis de entrada foram escolhidas. Um número excessivo de variáveis de entrada pode degradar o desempenho de um modelo de previsão, provocando interferências na estimativa, aumento na complexidade do modelo e longos intervalos de processamento para obtenção dos resultados. Para realizar essa atividade, foi escolhido o método ``Random Forest" apresentado e descrito no livro de Torgo \cite{torgo2003data}. Esse método ordena as variáveis de entrada que estão correlacionadas com a saída. Quatro configurações foram geradas e analisadas: sem a remoção de variáveis de entrada, removendo as cinco variáveis menos importantes, removendo as dez variáveis menos importantes e removendo as quinze variáveis menos importantes.

Após a análise, as quatro configurações da base de dados através da estimativa do impacto de riscos utilizando uma MLP com regra de aprendizado \textit{backpropagation}. Observou-se, que houve um aumento nos erros de previsão (REMQ) à medida que mais variáveis de entrada foram removidas. Então, nenhuma variável de entrada foi removida da base de dados pré-processada.

Por fim, um método de validação cruzada com divisão de amostras foi utilizado para prevenir a possibilidade da ocorrência do \textit{underfitting} e do \textit{overfitting} e a base de dados foi estratificada, em que cada subconjunto de dados continha a quantidade igual de registros de acordo com a variável com as três principais variáveis de entrada de maior importância obtidas com o algoritmo ``\textit{Random Forest}" (``Subcategory", ``Region" e ``Parameter"). A base de dados pré-processada foi dividida em três subconjuntos balanceados: subconjunto de treinamento, de validação cruzada e teste. Esses subconjuntos eram recriados a cada simulação e a cada modelo.

Portanto, após a binarização, a normalização utilizando a função gamma, a seleção das melhores variáveis de entrada, a estratificação e a divisão das amostras num subconjunto de validação cruzada, a base de dados PERIL encontrou-se preparada para ser utilizada nos experimentos subsequentes. As variáveis de entrada para o modelo são os valores 0 ou 1 das classes binarizadas, a saída esperada é o impacto do risco, o erro é calculado utilizando o impacto desejado (presente na base de dados) e o calculado.

\section{Regressão Linear Múltipla e Modelo de Regressão em Árvore}

%The first experiment set in this work is to establish a baseline method by comparing the performance between usual statistical linear regression model, Multiple Linear Regression (MLR) and Regression Tree Model (RTM).
O primeiro experimento definido para este trabalho consistiu no estabelecimento de uma linha de base para que fosse possível comparar o desempenho de outras abordagens com a base de dados selecionada. Os modelos MRLM e MRA foram analisados e os seus erros de previsão (REMQ) foram comparados sendo selecionado aquele que apresentou o menor valor do erro. Os resultados foram produzidos rapidamente, devido a baixa complexidade desses modelos de regressão linear.

Nenhum trabalho dentre os encontrados na literatura estabeleceram uma linha de base para os estudos posteriores. Além disso, não foi desenvolvido um \textit{benckmarking} para a estimativa do impacto de riscos. Logo, se faz necessário o estabelecimento de uma linha de base no início desse estudo.

%In this study, the source code of MLR model was adapted from Torgo \cite{torgo2003data}, in order to perform linear regression model training, cross-validation, outcome prediction and RMSE evaluation. MRL and RTM methods were analyzed statistically to define the baseline linear regression model for further analysis. The results for this previous analysis are presented in Chapter \ref{cap:results}.
Nessa análise, o código-fonte para análise dos modelos de regressão linear foram adaptados de Torgo \cite{torgo2003data} com o objetivo de realizar o treinamento, a validação cruzada, a geração das saídas previstas e o cálculo do REMQ.

\section{Simulação de Monte Carlo e Análise PERT}

O segundo experimento consistiu em analisar o desempenho das técnicas convencionais utilizadas na academia e na indústria, inclusive determinadas como boas práticas pelo PMBOK \cite{PMBOK2008}. Como explicado anteriormente, essas abordagens foram configuradas para obterem o melhor desempenho possível.

Esses modelos produziram os resultados mais rapidamente, devido ao fato deles utilizarem cálculos estatísticos extraídos da base de dados. Para esses modelos decidiu-se não dividir a base de dados, logo todas as setecentos e setenta e duas amostras foram utilizadas. Além disso, as amostras que apresentaram as mesmas variáveis de entrada foram filtradas para que se pudesse obter os menores REMQ possíveis.

\section{Perceptron de Múltiplas Camadas e suas variações}

O terceiro experimento teve como objetivo analisar quais das variações da MLP obteve o menor REMQ de previsão. Há numerosas combinações possíveis de configurações da MLP, no entanto somente um subconjunto delas foi avaliado. A melhor configuração da MLP foi utilizada no experimento subsequente.

Nos primeiros resultados, uma MLP simples utilizando o algoritmo \textit{backpropagation} apresentou uma REMQ média aproximadamente duas vezes maior (0,10007) que a ``MLPRegressor" obtida nos últimos resultados (0,05168). Após um estudo mais detalhado, a investigação das variações de MLPs que contém diferentes regras de aprendizado foi sugerida e, posteriormente, aprovada, já que os resultados obtidos foram duas vezes inferiores.

Esse experimento e o próximo foram os experimentos mais significativos desse trabalho. A importância da avaliação de diversas alternativas à MLP tradicional, baseada no algoritmo \textit{backpropagation} foi de fundamental importância já que nenhum dos trabalhos anteriores refinaram esse estudo. Além disso, eles não investigaram se outras abordagens como RBF e SVM poderiam ter um melhor desempenho.

\section{MLP, SVM, RBF e ANFIS}

%The forth experiment 
O quarto experimento teve como objetivo eleger a melhor técnica baseada em Redes Neurais Artificiais para a previsão do impacto de riscos, a partir da base de dados PERIL. As melhores configurações para cada uma das abordagens foram selecionadas e a REMQ foi calculada para cada técnica.

Paralelamente a análise das variações da MLP, outras RNAs foram analisadas nesse caso a SVM e a RBF. Por último, o modelo de previsão do estado da arte ANFIS proposto por Saxena \cite{saxena2012software} foi implementado e a REMQ para previsões foram geradas.

Conforme ilustrado na Figura \ref{fig:method2} do Capítulo \ref{cap:methodology}, um algoritmo de otimização foi utilizado para que os parâmetros dos modelos fossem definidos de modo a reduzir a REMQ para cada modelo. Para a MLP, os parâmetros otimizados foram a quantidade de neurônios escondidos nas duas camadas escondidas, o fator de penalidade e o parâmetro de tolerância dos valores delta. Para o SVM, os parâmetros otimizados foram a constante de complexidade, um parâmetro na função de perda, um parâmetro para o erro de arredondamento, um parâmetro de tolerância para o critério de parada e o expoente para as funções gaussianas. Já para a RBF, os parâmetros otimizados foram o número de funções de base gaussiana, um parâmetros de tolerância para os valores delta, o fator de penalidade dos pesos nas saídas e o tipo da escala de otimização.

\section{Validação do Melhor Modelo}

Por fim, um teste estatístico dos resultados da melhor abordagem com os oriundos do segundo experimento foi realizado para observar se o modelo baseado em Redes Neurais apresentava maior precisão do que os tradicionais. Tendo sido validada a hipótese nula, de que as redes neurais artificiais são mais precisas e que poderiam atender à real necessidade da indústria, conclui-se que através da metodologia proposta nesse trabalho que é possível a obtenção de uma RNA mais precisa para as estimativas dos impactos de riscos.

\pagebreak

  \chapter{Simulation Results}\label{cap:results}
This chapter presents the results achieved by the algorithms depicted in the Chapters \ref{cap:Swarm} and \ref{cap:contribution} using all optimization problems described in the Chapter \ref{cap:methodology}. This Chapter is divided in three parts:
\begin{itemize}
  \item A parametric analysis of the ABeePSO:
  \begin{itemize}
    \item Analysis of dispersion step;
    \item Analysis of acceleration coefficients ($c_1$ and $c_2$);
    \item Analysis of intervals of fuzzy membership functions;
    \item Analysis of stagnation counter;
  \end{itemize}
  \item Performance comparison among ABeePSO, PSO, APSO and ABC:
  \begin{itemize}
    \item Analysis of convergence;
    \item Analysis of the dimensionality;
    %\item Analysis of the diversity;
  \end{itemize}
  \item Initial analysis of the ClanABeePSO algorithm.
\end{itemize}

\section{Parametric Analysis of the ABeePSO}
This section presents the parametric analysis for the ABeePSO algorithm. This analysis aims to adapt the new operator and avoid explosion states in the ABeePSO algorithm. In this analysis, the number of dimensions is equal to 100 and the number of iterations is 1,000. All results were obtained over 10 trials.

\subsection{Analysis of Dispersion Step}
This analysis aims to investigate the influence of the parameters $\alpha$ and $\beta$ used to calculate the dispersion step of the guide bees (observe the Equation (\ref{eq:ABeePSO_Step})). We use one unimodal function ($F1$) and one multimodal function ($F16$). We selected a logistic function since we aim to have a step value around 1.0 for low $f_{evol}$ values and low step values for high $f_{evol}$ values.

We varied the value of $\alpha$ between 0 and 100, for $\beta$ = 0.1, 0.2 and 0.5. According to Figure \ref{fig:Dispersion_Step}, we observed that the best results were obtained for $\beta$ = 0.5. We also observed that the variation of $\alpha$ does not significantly interfere in the performance for the studied unimodal functions ($F1$) (Figure \ref{fig:Dispersion_F1}). In the multimodal function ($F16$), we observed that lower $\alpha$ values led to better results (Figure \ref{fig:Dispersion_F16}). Because of this, we assume $\alpha$ = 10.0 and $\beta$ = 0.5 for further simulations.

\begin{figure}[!h]
\centering
\subfigure[]{\includegraphics[scale=0.25]{results/dispersion/F1}\label{fig:Dispersion_F1}}
\hspace{1mm}
\subfigure[]{\includegraphics[scale=0.25]{results/dispersion/F16}\label{fig:Dispersion_F16}}
\caption{\small{Performance of the ABeePSO algorithm as a function of $\alpha$ for (a) $F1$ and (b) $F16$, considering $\beta$ = 0.1 (squares), $\beta$ = 0.2 (circles) and $\beta$ = 0.5 (triangle).}}
\label{fig:Dispersion_Step}
\end{figure}

\subsection{Analysis of Acceleration Coefficients ($c_1$ and $c_2$)}
Since the acceleration coefficients ($c_1$ and $c_2$) must be updated at each iteration and our proposal presents a different dynamic behavior, it is necessary to analyze the impact of the variation rate of $c_1$ and $c_2$ on the performance of our proposal. $\delta$ defines the increment (or decrement) of $c_1$ and $c_2$ per iteration. As lightly increment (or decrement) is performed by using $\delta/2$.

Table \ref{tab:acceleration_coefficients} shows the average fitness value for the functions $F4$ (unimodal) and $F13$ (multimodal) for different policies to determine $\delta$. The influence on the performance is not significant for the unimodal function, but the best result was achieved for a $\delta$ = 0.2. For the multimodal ($F13$), we the best results where achieved when $\delta$ is randomly chosen from the intervals $[0.01;0.05]$ and $[0.05;0.1]$. Since the difference between the results is not significant, we chose the same interval defined in the original APSO for further simulations, \textit{i.e.} $\delta$ randomly chosen from the interval $[0.05;0.1]$.

\begin{table}[!h]
\caption{\small{Impact of variation of acceleration rate in quality of search.}}
\centering
\begin{tabular}{c c c}
\hline
Value of $\delta$  &   $F4$               &  $F13$ \\
\hline
Random [0.01;0.1]    & 1.23E+14          & 1.02E+08 \\
Random [0.01;0.05]   & 1.33E+14          & \textbf{5.26E+07} \\
Random [0.02;0.1]    & 1.21E+14          & 5.91E+07 \\
Random [0.02;0.2]    & 1.38E+14          & 8.42E+07  \\
Random [0.05;0.1]    & 1.32E+14          & 5.72E+07  \\
Random [0.05;0.2]    & 1.37E+14          & 5.82E+07  \\
Fixed at 0.1          & 1.30E+14          & 8.91E+07  \\
Fixed at 0.2          & \textbf{1.13E+14} & 6.56E+07  \\
Fixed at 0.05         & 1.32E+14          & 1.05E+08  \\
\hline
\end{tabular}
\label{tab:acceleration_coefficients}
\end{table}

\subsection{Analysis of Intervals of Fuzzy Membership Functions}
This analysis is important since it determines the shift between APSO and ABeePSO. We performed some simulations for the functions $F4$ (unimodal) and $F13$ (multimodal) for different Fuzzy membership functions. The configuration 1 uses the Fuzzy membership functions depicted in Figure \ref{fig:factor_ABeePSO}. In the configuration 2 we modified the interval of the \emph{Convergence} state from 0.2 to 0.3 and the threshold of the \emph{Exploitation} state from 0.1 to 0.2, as one can observe in Figure \ref{fig:factor_ABeePSO_Test1}. This variation allows the algorithm to anticipate the start of the execution of the ABC based operator. In the configuration 3 we modified the interval of \emph{Exploitation} state from 0.5 to 0.6 and the threshold of the \emph{Exploration} state from 0.4 to 0.5, one can observe Figure \ref{fig:factor_ABeePSO_Test2}. This variation allows the execution of the ABC based operator for a longer time.

\begin{figure}[!h]
\centering
\subfigure[]{\includegraphics[scale=0.55]{results/fuzzy/factor_ABPSO_Test1.jpg}\label{fig:factor_ABeePSO_Test1}}
\hspace{1mm}
\subfigure[]{\includegraphics[scale=0.55]{results/fuzzy/factor_ABPSO_Test2.jpg}\label{fig:factor_ABeePSO_Test2}}
\caption{\small{Different configurations to analysis of intervals of fuzzy membership functions: (a) Configuration 2 and (b) Configuration 3.}}
\label{fig:Analysis_Fuzzy}
\end{figure}

For the unimodal function ($F7$), the configurations 1, 2 and 3 achieved average fitness (standard deviation) equal to $4.2E+10(8.53E+09)$, $4.59E+10(5.43E+09)$ and $4.9E+10(1.31E+09)$, respectively. For the multimodal function ($F20$), the configurations 1, 2 and 3 achieved the following average fitness (standard deviation) equal to $8.23E+08(2.5E+08)$, $1.02E+09(6.39E+08)$ and $1.29E+09(4.6E+08)$, respectively. The results indicate the configuration depicted in Figure \ref{fig:factor_ABeePSO}, \textit{i.e.} configuration 1. In other benchmark functions we observed that the dependence on the fuzzy intervals is higher for the multimodal functions, but the configuration depicted in Figure \ref{fig:factor_ABeePSO} is still the best option.

\subsection{Analysis of Stagnation Counter}
This initial analysis aims to estimate the initial value for stagnation counter and evaluate the behavior of algorithm with this information. We used one unimodal functions ($F4$) and multimodal function ($F2$).

We varied the value of stagnation counter is ranging between 0 and 100. According to results, we observed a similar performance for both functions. For the multimodal function, when the stagnation counter value is medium or high, the fitness value does not improve, \textit{i.e.} it is necessary to expand the swarm by using the ABC based operator and generate diversity. For the unimodal function, the best results were obtained when the stagnation counter value is low or medium. When the value is very high, the obtained fitness is mitigated. From this analysis, we defined a initial value equal 10.0 for stagnation step.

%\begin{figure}[!h]
%\centering
%\subfigure[]{\includegraphics[scale=0.35]{results/stagnation/F4.jpg}\label{fig:Stagnation_F4}}
%\hspace{1mm}
%\subfigure[]{\includegraphics[scale=0.35]{results/stagnation/F2.jpg}\label{fig:Stagnation_F2}}
%\caption{\small{Performance of the ABPSO algorithm varying the value of stagnation counter in: (a) $F4$ and (b) $F2$.}}
%\label{fig:Stagnation}
%\end{figure}

\section{Performance comparison among ABeePSO, PSO, APSO and ABC}
This section presents a performance analysis of the ABeePSO algorithm and a comparison to the  PSO, APSO and ABC algorithms. We assessed the convergence velocity and scalability.
In the analysis of convergence, the number of dimensions is equal to 100 and the number of iterations is equal to 1,000. In the analysis of dimensionality, we performed 3,000, 5,000 and 10,000 iterations for 300, 500 and 1,000 dimensions, respectively. All results were obtained over 50 trials.
 %and the diversity analysis over 10 trials.

\subsection{Analysis of Convergence}
We compared the performance of various algorithms along the iterations to analyze the convergence velocity. Figure \ref{fig:Convergence} depicts the evolution of ABC, APSO PSO and ABeePSO algorithms for multimodal and unimodal benchmark functions. The Figures \ref{fig:Convergence_F3} and \ref{fig:Convergence_F5} are multimodal functions F3 and F5, respectively and Figures \ref{fig:Convergence_F12} and \ref{fig:Convergence_F14} are unimodal functions F12 and F14, respectively. We did not observe a significant improvement in the convergence velocity for unimodal function. However, our proposal outperformed the other approaches for the multimodal problems, mainly for the F3 function.

\begin{figure}[!h]
\centering
\subfigure[]{\includegraphics[scale=0.22]{results/convergence/F3}\label{fig:Convergence_F3}}
\hspace{1mm}
\subfigure[]{\includegraphics[scale=0.22]{results/convergence/F5}\label{fig:Convergence_F5}}
\hspace{1mm}
\subfigure[]{\includegraphics[scale=0.22]{results/convergence/F12}\label{fig:Convergence_F12}}
\hspace{1mm}
\subfigure[]{\includegraphics[scale=0.22]{results/convergence/F14}\label{fig:Convergence_F14}}
\caption{\small{Analysis of Convergence: Fitness function evolution of the ABC, APSO, PSO and ABeePSO algorithms for the $D$ = 100 in (a) $F3$, (b) $F5$, (c) $F12$ and (d) $F14$ functions.}}
\label{fig:Convergence}
\end{figure}

\subsection{Analysis of Scalability}
We also analyzed the performance of the algorithms as a function of the number of dimensions of the problem. We aim to analyze the scalability of the algorithms. As the dimensionality increases, the difficulty of the problems also increases. Figures \ref{fig:Dimensionality_F7} and \ref{fig:Dimensionality_F11} show the boxplot of the fitness as a function of the dimensionality for the benchmark functions $F7$ (unimodal) and $F11$ (multimodal), respectively.

For the unimodal function, one can observe that the ratio between the performance of the algorithms is maintained, but it is smaller for $D$ = 1,000.

For the multimodal function, we obtained better results when compared to other approaches, \textit{i.e.} the difference in the performance between our proposal and other techniques is maintained as the dimensionality of the problem also increases.

\begin{figure}[!h]
\centering
\subfigure[]{\includegraphics[scale=0.22]{results/dimensionality/F7-100D}\label{fig:Dimensionality_F7_100D}}
\hspace{1mm}
\subfigure[]{\includegraphics[scale=0.22]{results/dimensionality/F7-300D}\label{fig:Dimensionality_F7_300D}}
\hspace{1mm}
\subfigure[]{\includegraphics[scale=0.22]{results/dimensionality/F7-500D}\label{fig:Dimensionality_F7_500D}}
\hspace{1mm}
\subfigure[]{\includegraphics[scale=0.22]{results/dimensionality/F7-1000D}\label{fig:Dimensionality_F7_1000D}}
\caption{\small{Analysis of the Dimensionality: Boxplot of the fitness for the $F7$ function varying the number of dimensions for (a) $D$ = 100, (b) $D$ = 300, (c) $D$ = 500 and (d) $D$ = 1,000.}}
\label{fig:Dimensionality_F7}
\end{figure}

\begin{figure}[!h]
\centering
\subfigure[]{\includegraphics[scale=0.22]{results/dimensionality/F11-100D}\label{fig:Dimensionality_F11_100D}}
\hspace{1mm}
\subfigure[]{\includegraphics[scale=0.22]{results/dimensionality/F11-300D}\label{fig:Dimensionality_F11_300D}}
\hspace{1mm}
\subfigure[]{\includegraphics[scale=0.22]{results/dimensionality/F11-500D}\label{fig:Dimensionality_F11_500D}}
\hspace{1mm}
\subfigure[]{\includegraphics[scale=0.22]{results/dimensionality/F11-1000D}\label{fig:Dimensionality_F11_1000D}}
\caption{\small{Analysis of the Dimensionality: Boxplot of the fitness for the $F11$ function varying the number of dimensions for (a) $D$ = 100, (b) $D$ = 300, (c) $D$ = 500 and (d) $D$ = 1,000.}}
\label{fig:Dimensionality_F11}
\end{figure}

We realized experiments with all benchmark functions, varying the number of dimensions. We made the comparison the ABeePSO with ABC, APSO and PSO algorithms. Tables \ref{tab:Comparison_100D}, \ref{tab:Comparison_300D}, \ref{tab:Comparison_500D} and \ref{tab:Comparison_1000D} present the average fitness and (standard deviation) obtained from simulations for 100, 300, 500 and 1,000 dimensions, respectively. The best average fitness values is highlighted for each benchmark function.

\begin{table}[!h]
\caption{\small{Performance comparison in terms of average fitness value and (standard deviation) between ABeePSO, PSO, APSO and ABC for all the 20 benchmark functions for 100 dimensions.}}
\label{tab:Comparison_100D}
\begin{center}
\begin{tabular}{p{0.5cm}|p{2.5cm}|p{2.5cm}|p{2.5cm}|p{2.5cm}}
\hline\noalign{\smallskip}
\textbf{F\#}	& ABeePSO & ABC & APSO & PSO    \\		
\noalign{\smallskip}
\hline
\noalign{\smallskip}
\textit{F1}  & \textbf{2.86E+08 (8.22E+07)} & 9.20E+09 (1.92E+09) & 3.23E+09 (1.19E+09) & 9.16E+08 (1.62E+08)\\
\textit{F2}  & 9.58E+02 (63.09667) & 1.31E+03 (85.5529) & \textbf{9.10E+02 (70.90304)} & 1.21E+03 (62.80075)\\
\textit{F3}  & \textbf{12.07975 (0.656529)} & 20.62107 (0.08368) & 19.62592 (0.401446) & 17.45967 (0.635132)\\
\textit{F4}  & \textbf{1.32E+14 (3.05E+13)} & 1.58E+15 (3.05E+14) & 3.11E+14 (1.24E+14) & 2.84E+14 (5.87E+13)\\
\textit{F5}  & \textbf{3.81E+08 (2.64E+07)} & 7.28E+08 (3.32E+07) & 4.45E+08 (4.70E+07) & 4.81E+08 (2.75E+07)\\
\textit{F6}  & \textbf{7.59E+06 (8.71E+05)} & 2.09E+07 (6.01E+04) & 1.97E+07 (3.27E+05) & 1.28E+07 (6.37E+05)\\
\textit{F7}  & \textbf{4.33E+10 (9.10E+09)} & 2.19E+11 (2.79E+10) & 8.45E+10 (1.61E+10) & 7.50E+10 (1.2E+10)\\
\textit{F8}  & \textbf{6.78E+13 (5.49E+05)} & 7.98E+16 (2.01E+16) & 2.99E+16 (9.34E+15) & 4.49E+14 (1.19E+14)\\
\textit{F9}  & \textbf{4.02E+08 (9.68E+07)} & 2.16E+09 (6.70E+08) & 1.15E+09 (4.64E+08) & 1.11E+09 (1.52E+08)\\
\textit{F10} & 9.73E+02 (39.83971) & 1.64E+03 (68.497) & \textbf{8.54E+02 (77.301)} & 1.23E+03 (64.016)\\
\textit{F11} & \textbf{25.93234 (2.34576)} & 41.67951 (0.132561) & 38.83987 (0.875927) & 34.68351 (0.87051)\\
\textit{F12} & \textbf{9.26E+04 (1.67E+04)} & 5.86E+05 (7.60E+04) & 3.08E+05 (5.03E+04) & 1.95E+05 (2.19E+04)\\
\textit{F13} & \textbf{8.08E+07 (7.19E+07)} & 6.46E+10 (1.52E+10) & 2.00E+10 (5.99E+09) & 4.69E+08 (1.19E+08)\\
\textit{F14} & \textbf{6.29E+08 (1.21E+08)} & 5.36E+09 (9.68E+08) & 1.26E+09 (3.63E+08) & 1.32E+09 (2.21E+08)\\
\textit{F15} & \textbf{9.82E+02 (40.05412)} & 1.75E+03 (57.49683) & 1.15E+03 (62.5245) & 1.24E+03 (67.00242)\\
\textit{F16} & \textbf{25.48114 (2.66121)} & 42.02208 (0.098697) & 40.38697 (0.411129) & 35.10031 (0.811823)\\
\textit{F17} & \textbf{1.84E+05 (2.29E+04)} & 6.79E+05 (7.42E+04) & 3.04E+05 (3.85E+04) & 2.77E+05 (3.40E+04)\\
\textit{F18} & \textbf{1.02E+09 (6.82E+08)} & 2.01E+11 (3.15E+10) & 8.16E+10 (1.37E+10) & 1.16E+10 (3.25E+09)\\
\textit{F19} & \textbf{2.09E+05 (2.82E+04)} & 8.01E+05 (7.82E+04) & 4.03E+05 (9.40E+04) & 3,08E+05 (3.96E+04)\\
\textit{F20} & \textbf{1.09E+09 (6.41E+08)} & 2.58E+11 (4.56E+10) & 1.12E+11 (1.81E+10) & 1.17E+10 (5.35E+09)\\
\hline
\end{tabular}
\end{center}
\end{table}

\begin{table}[!h]
\caption{\small{Performance comparison in terms of average fitness value and (standard deviation) between ABeePSO, PSO, APSO and ABC for all the 20 benchmark functions for 300 dimensions.}}
\label{tab:Comparison_300D}
\begin{center}
\begin{tabular}{p{0.5cm}|p{2.5cm}|p{2.5cm}|p{2.5cm}|p{2.5cm}}
\hline\noalign{\smallskip}
\textbf{F\#}	& ABeePSO & ABC & APSO & PSO    \\		
\noalign{\smallskip}
\hline
\noalign{\smallskip}
\textit{F1}	& \textbf{1.41E+09 (2.11E+08)} & 6.12E+10 (4.48E+09) & 1.41E+09 (1.2E+09) & 6.14E+09 (7.23E+08) \\
\textit{F2}	& 4.03E+03 (1.52E+02) & 5.40E+03 (1.43E+02) & \textbf{1.53E+03 (1.27E+02)} & 4.39E+03 (1.26E+02)\\		
\textit{F3}	& \textbf{13.81304 (0.40062)} & 21.04023 (0.041704) & 17.86329 (0.629742) & 20.49093 (0.160888) \\
\textit{F4}& \textbf{5.88E+13 (1.64E+13)} & 2.44E+15 (5.44E+14) & 9.07E+13 (4.39E+13) & 1.62E+14 (3.97E+13) \\
\textit{F5}& \textbf{3.29E+08 (3.68E+07)} & 8.14E+08 (3.59E+07) & 4.08E+08 (4.82E+07) & 4.11E+08 (2.45E+07) \\
\textit{F6}& \textbf{4.81E+06 (3.94E+05)} & 2.08E+07 (9.46E+04) & 1.91E+07 (3.26E+05) & 9.62E+06 (5.56E+05) \\
\textit{F7}& \textbf{2.03E+10 (4.10E+09)} & 2.01E+11 (2.92E+10) & 5.16E+10 (1.19E+10) & 4.51E+10 (6.51E+09) \\
\textit{F8}& \textbf{3.13E+13 (9.01E+12)} & 6.78E+16 (1.56E+16) & 1.79E+16 (8.31E+15) & 6.3E+13 (1.75E+13) \\
\textit{F9}& \textbf{2.42E+09 (3.48E+08)} & 6.09E+10 (5.72E+09) & 3.18E+09 (2.02E+09) & 6.66E+09 (7.21E+08) \\
\textit{F10}& \textbf{8.50E+02 (3.96E+01)} & 6.06E+03 (1.27E+02) & 2.65E+03 (1.30E+02) & 4.59E+03 (1.12E+02)\\
\textit{F11}& \textbf{60.82139 (6.383927)} & 84.81496 (0.08274) & 79.68142 (1.100449) & 74.01696 (1.241357) \\
\textit{F12}& \textbf{6.02E+05 (4.74E+04)} & 2.31E+06 (1.70E+05) & 7.39E+05 (9.22E+04) & 1.06E+06 (5.85E+04)\\
\textit{F13}& \textbf{4.14E+08 (9.32E+07)} & 6.71E+11 (5.86E+10) & 1.07E+11 (3.14E+10) & 9.34E+09 (2.64E+09)\\
\textit{F14}& \textbf{2.39E+09 (2.57E+08)} & 3.96E+10 (5.34E+09) & 2.50E+09 (3.30E+08) & 7.28E+09 (7.06E+08)\\
\textit{F15}& 4.01E+03 (1.42E+02) & 6.00E+03 (1.18E+02) & \textbf{3.16E+03 (1.47E+02)} & 4.49E+03 (1.04E+02)\\
\textit{F16}& \textbf{96.59136 (7.406842)} & 127.2318 (0.128599) & 123.3011 (0.418354) & 118.5161 (1.745353)\\
\textit{F17}& \textbf{8.41E+05 (8.24E+04)} & 4.76E+06 (6.23E+05) & 9.26E+05 (1.04E+04) & 1.45E+06 (1.59E+05)\\
\textit{F18}& \textbf{4.23E+09 (1.39E+09)} & 1.35E+12 (1.01E+11) & 7.53E+10 (5.31E+10) & 4.70E+12 (8.02E+10)\\
\textit{F19}& 1.37E+06 (1.35E+05) & 6.82E+06 (7.44E+05) & \textbf{1.30E+06 (3.48E+05)} & 2.11E+06 (2.41E+05)\\
 \textit{F20}& \textbf{6.04E+09 (2.87E+09)} & 1.51E+12 (1.18E+11) & 1.57E+11 (3.14E+10) & 1.78E+11 (2.19E+10)\\
\hline
\end{tabular}
\end{center}
\end{table}


\begin{table}[!h]
\caption{\small{Performance comparison in terms of average fitness value and (standard deviation) between ABeePSO, PSO, APSO and ABC for all the 20 benchmark functions for 500 dimensions.}}
\label{tab:Comparison_500D}
\begin{center}
\begin{tabular}{p{0.5cm}|p{2.5cm}|p{2.5cm}|p{2.5cm}|p{2.5cm}}
\hline\noalign{\smallskip}
\textbf{F\#}	& ABeePSO & ABC & APSO & PSO    \\		
\noalign{\smallskip}
\hline
\noalign{\smallskip}
\textit{F1}	& \textbf{2.17E+09 (3.24E+08)} & 1.22E+11 (7.33E+09) & 2.86E+09 (2.27E+09) & 1.19E+10 (8.12E+08)\\		
\textit{F2}	& 6.57E+03 (1.70E+02) & 9.91E+03 (2.04E+02) & \textbf{2.86E+03 (2.27E+02)} & 7.59E+03 (1.27E+02)\\		
\textit{F3}	& \textbf{14.58456 (0.524085)} & 21.17949 (0.043387) & 18.05753 (0.4994) & 20.77854 (0.058559) \\
\textit{F4} & \textbf{4.81E+13 (1.26E+13)} & 2.30E+15 (4.41E+14) & 6.83E+13 (2.67E+13) & 1.28E+14 (2.84E+13)\\		
\textit{F5} & \textbf{3.09E+08 (3.86E+07)} & 7.68E+08 (3.67E+07) & 3.42E+08 (4.52E+07) & 3.97E+08 (2.40E+07)\\
\textit{F6} & \textbf{4.47E+06 (5.79E+05)} & 2.09E+07 (9.32E+04) & 1.94E+07 (2.65E+05) & 8.73E+06 (3.13E+05)\\
\textit{F7} & \textbf{1.41E+10 (3.68E+09)} & 2.36E+11 (3.26E+10) & 3.91E+10 (9.61E+09) & 3.58E+10 (5.45E+09)\\
\textit{F8} & \textbf{9.43E+12 (2.90E+12)} & 1.22E+17 (1.88E+16) & 5.51E+15 (3.65E+15) & 2.64E+13 (8.72E+12)\\
\textit{F9} & 3.82E+09 (5.49E+08) & 1.25E+11 (9.68E+09) & \textbf{2.67E+09 (5.02E+08)} & 1.52E+10 (1.20E+09)\\
\textit{F10}& 7.20E+03 (1.90E+02) & 1.06E+04 (1.39E+02) & \textbf{4.21E+03 (1.48E+02)} & 7.87E+03 (1.34E+02)\\
\textit{F11}& \textbf{99.34556 (8.08220)} & 127.5809 (0.10924) & 120.7821 (0.91662) & 115.0733 (1.05427) \\
\textit{F12}& 1.13E+06 (6.99E+04) & 5.69E+06 (4.68E+05) & \textbf{8.53E+05 (8.87E+04)} & 2.05E+06 (1.11E+05)\\
\textit{F13}& \textbf{5.21E+08 (1.25E+08)} & 1.30E+12 (7.93E+10) & 1.67E+10 (1.10E+10) & 4.00E+10 (9.53E+09)\\
\textit{F14}& 4.37E+09 (4.61E+08) & 1.04E+11 (7.25E+09) & \textbf{3.89E+09 (2.89E+08)} & 1.55E+10 (1.00E+09)\\
\textit{F15}& 7.34E+03 (1.50E+02) & 1.07E+04 (1.13E+02) & \textbf{6.43E+03 (2.41E+02)} & 8.01E+03 (1.20E+02)\\
\textit{F16}& \textbf{166.3537 (9.092158)} & 212.7954 (0.119684) & 206.1731 (0.543216) & 205.4262 (1.845581)\\
\textit{F17}& \textbf{1.46E+06 (1.16E+05)} & 1.05E+07 (1.30E+06) & 1.73E+06 (1.16E+05) & 2.91E+06 (2.04E+05)\\
\textit{F18}& \textbf{1.38E+10 (2.01E+09)} & 2.77E+12 (1.59E+11) & 1.32E+11 (5.89E+10) & 3.48E+11 (2.90E+10)\\
\textit{F19}& 4.23E+06 (3.37E+05) & 1.75E+07 (1.98E+06) & \textbf{3.85E+06 (8.90E+05)} & 5.40E+06 (4.68E+05)\\
\textit{F20}& \textbf{1.43E+10 (2.20E+09)} & 3.03E+12 (1.39E+11) & 1.33E+11 (6.61E+10) & 4.24E+11 (3.70E+10)\\
\hline
\end{tabular}
\end{center}
\end{table}

\begin{table}[!h]
\caption{\small{Performance comparison in terms of average fitness value and (standard deviation) between ABeePSO, PSO, APSO and ABC for all the 20 benchmark functions for 1,000 dimensions.}}
\label{tab:Comparison_1000D}
\begin{center}
\begin{tabular}{p{0.5cm}|p{2.5cm}|p{2.5cm}|p{2.5cm}|p{2.5cm}}
\hline\noalign{\smallskip}
\textbf{F\#}	& ABeePSO & ABC & APSO & PSO    \\		
\noalign{\smallskip}
\hline
\noalign{\smallskip}
\textit{F1}	& \textbf{9.49E+09 (7.41E+08)} & 2.93E+11 (9.25E+09) & 1.02E+10 (4.10E+09) & 3.34E+10 (2.09E+09)\\		
\textit{F2}	& 14295.42 (305.095) & 21862.81 (414.0550) & \textbf{5646.419 (251.3274)} & 15878.37 (201.1898)\\		
\textit{F3}	& 20.5246 (0.2398) & 21.3259 (0.0189) & \textbf{18.2617 (0.3523)} & 20.9554 (0.0209)\\
\textit{F4} & \textbf{5.55E+13 (1.52E+13)} & 2.29E+15 (4.13E+14) & 6.37E+13 (2.35E+13) & 1.05E+14 (2.39E+13)\\		
\textit{F5} & \textbf{2.73E+08 (4.07E+07)} & 8.38E+08 (2.99E+07) & 3.42E+08 (4.52E+07) & 3.38E+08 (2.58E+07)\\
\textit{F6} & \textbf{1.36E+06 (3.81E+05)} & 2.08E+07 (9.16E+04) & 1.90E+07 (3.35E+05) & 6.41E+06 (4.34E+05)\\
\textit{F7} & \textbf{1.22E+10 (3.52E+09)} & 6.17E+11 (2.13E+11) & 2.51E+10 (6.68E+09) & 2.38E+10 (5.68E+09)\\
\textit{F8} & \textbf{3.34E+12 (1.14E+12)} & 9.58E+16 (1.49E+16) & 1.99E+15 (2.13E+15) & 6.62E+12 (2.83E+12)\\
\textit{F9} & 9.98E+09 (7.77E+08) & 3.07E+11 (1.43E+10) & \textbf{4.47E+09 (9.55E+08)} & 3.89E+10 (2.91E+09)\\
\textit{F10}& 15171.77 (225.8321) & 22536.49 (250.1126) & \textbf{8758.037 (234.3429)} & 16291.16 (223.1056)\\
\textit{F11}& \textbf{193.5644 (12.4481)} & 234.4677 (0.1527) & 222.5258 (0.7495) & 219.8308 (2.6664)\\
\textit{F12}& 3.80E+06 (2.05E+05) & 1.86E+07 (1.71E+06) & \textbf{1.71E+06 (1.55E+05)} & 4.76E+06 (2.12E+05)\\
\textit{F13}& \textbf{3.61E+09 (4.71E+08)} & 2.87E+12 (1.03E+11) & 2.87E+10 (1.24E+10) & 1.75E+11 (1.88E+10)\\
\textit{F14}& 1.22E+10 (3.52E+09) & 3.03E+11 (1.95E+10) & \textbf{7.41E+09 (5.23E+08)} & 4.09E+10 (2.52E+09)\\
\textit{F15}& 15229.42 (210.007) & 22634.39 (172.2981) & \textbf{12939.51 (253.0937)} & 16404.02 (211.2478)\\
\textit{F16}& \textbf{372.8392 (9.7793)} & 426.9702 (0.2063) & 413.1886 (0.9254) & 417.6073 (0.7588)\\
\textit{F17}& 4.07E+06 (3.43E+05) & 4.29E+07 (3.97E+06) & \textbf{3.38E+06 (3.07E+05)} & 7.61E+06 (4.14E+05)\\
\textit{F18}& 2.96E+10 (4.17E+09) & 6.49E+12 (2.09E+11) & \textbf{2.94E+10 (1.33E+10)} & 9.16E+11 (6.02E+10)\\
\textit{F19}& 1.42E+07 (1.16E+06) & 8.03E+07 (8.87E+06) & \textbf{6.40E+06 (5.90E+05)} & 1.84E+07 (1.57E+06)\\
\textit{F20}& \textbf{2.94E+10 (4.18E+09)} & 7.12E+12 (1.99E+11) & 3.88E+10 (1.24E+10) & 1.10E+12 (4.28E+10)\\
\hline
\end{tabular}
\end{center}
\end{table}

Since the difference between the algorithm is not large in some few cases, we performed a comparison by using a statistical test. We show the results for the Wilcoxon Test with significance level of 0.05. Up-triangle means that our approach is better, Down-triangle means that our approach is worst and ``-'' means that there is no statistical difference. Tables \ref{tab:Wilcoxon_100D}, \ref{tab:Wilcoxon_300D}, \ref{tab:Wilcoxon_500D} and \ref{tab:Wilcoxon_1000D} present the average fitness and (standard deviation) obtained from simulations for 100, 300, 500 and 1,000 dimensions, respectively. The test shows that our proposal is superior in most of cases, except for $F2$ and $F10$ in 100 dimensions, $F2$, $F15$ and $F19$ in 300 dimensions and $F2$, $F9$, $F10$, $F12$, $F14$, $F15$ and $F19$ in 500 dimensions. Our algorithm is superior in nine cases ($F4$, $F5$, $F6$, $F7$, $F8$, $F11$, $F13$, $F16$ and $F20$) in the search space with 1,000 dimensions.

\begin{table}[!h]
\caption{\small{Results of Wilcoxon test for the comparison of our proposal to the other approaches for $100D$ and $1,000$ iterations.}}
\label{tab:Wilcoxon_100D}
\begin{center}
\begin{tabular}{p{0.5cm}|p{1.5cm}|p{1.5cm}|p{1.5cm}}
\hline\noalign{\smallskip}
\textbf{F\#} & ABC & APSO & PSO    \\		
\noalign{\smallskip}
\hline
\noalign{\smallskip}
\textit{F1}  & $\blacktriangle$  & $\blacktriangle$  & $\blacktriangle$ \\
\textit{F2}  & $\blacktriangle$  & $\triangledown$  & $\blacktriangle$ \\
\textit{F3}  & $\blacktriangle$  & $\blacktriangle$  & $\blacktriangle$ \\
\textit{F4}  & $\blacktriangle$  & $\blacktriangle$  & $\blacktriangle$ \\
\textit{F5}  & $\blacktriangle$  & $\blacktriangle$  & $\blacktriangle$ \\
\textit{F6}  & $\blacktriangle$  & $\blacktriangle$  & $\blacktriangle$ \\
\textit{F7}  & $\blacktriangle$  & $\blacktriangle$  & $\blacktriangle$ \\
\textit{F8}  & $\blacktriangle$  & $\blacktriangle$  & $\blacktriangle$ \\
\textit{F9}  & $\blacktriangle$  & $\blacktriangle$  & $\blacktriangle$ \\
\textit{F10} & $\blacktriangle$  & $\triangledown$  & $\blacktriangle$ \\
\textit{F11} & $\blacktriangle$  & $\blacktriangle$  & $\blacktriangle$ \\
\textit{F12} & $\blacktriangle$  & $\blacktriangle$  & $\blacktriangle$ \\
\textit{F13} & $\blacktriangle$  & $\blacktriangle$  & $\blacktriangle$ \\
\textit{F14} & $\blacktriangle$  & $\blacktriangle$  & $\blacktriangle$ \\
\textit{F15} & $\blacktriangle$  & $\blacktriangle$  & $\blacktriangle$ \\
\textit{F16} & $\blacktriangle$  & $\blacktriangle$  & $\blacktriangle$ \\
\textit{F17} & $\blacktriangle$  & $\blacktriangle$  & $\blacktriangle$ \\
\textit{F18} & $\blacktriangle$  & $\blacktriangle$  & $\blacktriangle$ \\
\textit{F19} & $\blacktriangle$  & $\blacktriangle$  & $\blacktriangle$ \\
\textit{F20} & $\blacktriangle$  & $\blacktriangle$  & $\blacktriangle$ \\
\hline
\end{tabular}
\end{center}
\end{table}

\begin{table}[!h]
\caption{\small{Results of Wilcoxon test for the comparison of our proposal to the other approaches for $300D$ and $3,000$ iterations.}}
\label{tab:Wilcoxon_300D}
\begin{center}
\begin{tabular}{p{0.5cm}|p{1.5cm}|p{1.5cm}|p{1.5cm}}
\hline\noalign{\smallskip}
\textbf{F\#} & ABC & APSO & PSO    \\		
\noalign{\smallskip}
\hline
\noalign{\smallskip}
\textit{F1}	& $\blacktriangle$  & -  & $\blacktriangle$  \\
\textit{F2}	& $\blacktriangle$  & $\triangledown$  & $\blacktriangle$  \\
\textit{F3}	& $\blacktriangle$  & $\blacktriangle$  & $\blacktriangle$  \\
\textit{F4}&  $\blacktriangle$  & $\blacktriangle$  & $\blacktriangle$  \\
\textit{F5}&  $\blacktriangle$  & $\blacktriangle$  & $\blacktriangle$   \\
\textit{F6}&  $\blacktriangle$  & $\blacktriangle$  & $\blacktriangle$   \\
\textit{F7}&  $\blacktriangle$  & $\blacktriangle$  & $\blacktriangle$   \\
\textit{F8}&  $\blacktriangle$  & $\blacktriangle$  & $\blacktriangle$ \\
\textit{F9}&  $\blacktriangle$  & -  & $\blacktriangle$ \\
\textit{F10}& $\blacktriangle$  & $\blacktriangle$  & $\blacktriangle$ \\
\textit{F11}& $\blacktriangle$  & $\blacktriangle$  & $\blacktriangle$  \\
\textit{F12}& $\blacktriangle$  & $\blacktriangle$  & $\blacktriangle$ \\
\textit{F13}& $\blacktriangle$  & $\blacktriangle$  & $\blacktriangle$  \\
\textit{F14}& $\blacktriangle$  & $\blacktriangle$  & $\blacktriangle$   \\
\textit{F15}& $\blacktriangle$  & $\triangledown$  & $\blacktriangle$ \\
\textit{F16}& $\blacktriangle$  & $\blacktriangle$  & $\blacktriangle$  \\
\textit{F17}& $\blacktriangle$  & $\blacktriangle$  & $\blacktriangle$   \\
\textit{F18}& $\blacktriangle$  & $\blacktriangle$  & $\blacktriangle$  \\
\textit{F19}&  $\blacktriangle$  & $\triangledown$  & $\blacktriangle$  \\
\textit{F20}& $\blacktriangle$  & $\blacktriangle$  & $\blacktriangle$ \\
\hline
\end{tabular}
\end{center}
\end{table}

\begin{table}[!h]
\caption{\small{Results of Wilcoxon test for the comparison of our proposal to the other approaches for $500D$ and $5,000$ iterations.}}
\label{tab:Wilcoxon_500D}
\begin{center}
\begin{tabular}{p{0.5cm}|p{1.5cm}|p{1.5cm}|p{1.5cm}}
\hline\noalign{\smallskip}
\textbf{F\#} & ABC & APSO & PSO    \\		
\noalign{\smallskip}
\hline
\noalign{\smallskip}
\textit{F1}	& $\blacktriangle$  & -  & $\blacktriangle$ \\
\textit{F2}	& $\blacktriangle$  & $\triangledown$  & $\blacktriangle$\\
\textit{F3}	& $\blacktriangle$  & $\blacktriangle$  & $\blacktriangle$  \\
\textit{F4} & $\blacktriangle$  & $\blacktriangle$  & $\blacktriangle$  \\
\textit{F5} & $\blacktriangle$  & $\blacktriangle$  & $\blacktriangle$  \\
\textit{F6} & $\blacktriangle$  & $\blacktriangle$  & $\blacktriangle$ \\
\textit{F7} & $\blacktriangle$  & $\blacktriangle$  & $\blacktriangle$  \\
\textit{F8} & $\blacktriangle$  & $\blacktriangle$  & $\blacktriangle$  \\
\textit{F9} & $\blacktriangle$  & $\triangledown$  & $\blacktriangle$ \\
\textit{F10}& $\blacktriangle$  & $\triangledown$  & $\blacktriangle$ \\
\textit{F11}& $\blacktriangle$  & $\blacktriangle$  & $\blacktriangle$ \\
\textit{F12}& $\blacktriangle$  & $\triangledown$  & $\blacktriangle$  \\
\textit{F13}& $\blacktriangle$  & $\blacktriangle$  & $\blacktriangle$   \\
\textit{F14}& $\blacktriangle$  & $\triangledown$  & $\blacktriangle$   \\
\textit{F15}& $\blacktriangle$  & $\triangledown$  & $\blacktriangle$   \\
\textit{F16}& $\blacktriangle$  & $\blacktriangle$  & $\blacktriangle$  \\
\textit{F17}& $\blacktriangle$  & $\blacktriangle$  & $\blacktriangle$  \\
\textit{F18}& $\blacktriangle$  & $\blacktriangle$  & $\blacktriangle$ \\
\textit{F19}& $\blacktriangle$  & $\triangledown$  & $\blacktriangle$ \\
\textit{F20}& $\blacktriangle$  & $\blacktriangle$  & $\blacktriangle$ \\
\hline
\end{tabular}
\end{center}
\end{table}

\begin{table}[!h]
\caption{\small{Results of Wilcoxon test for the comparison of our proposal to the other approaches for $1000D$ and $10,000$ iterations.}}
\label{tab:Wilcoxon_1000D}
\begin{center}
\begin{tabular}{p{0.5cm}|p{1.5cm}|p{1.5cm}|p{1.5cm}}
\hline\noalign{\smallskip}
\textbf{F\#} & ABC & APSO & PSO    \\		
\noalign{\smallskip}
\hline
\noalign{\smallskip}
\textit{F1}	& $\blacktriangle$  & -  & $\blacktriangle$ \\
\textit{F2}	& $\blacktriangle$  & $\triangledown$  & $\blacktriangle$ \\
\textit{F3}	& $\blacktriangle$  & $\triangledown$  & $\blacktriangle$ \\
\textit{F4} & $\blacktriangle$  & $\blacktriangle$  & $\blacktriangle$ \\
\textit{F5} & $\blacktriangle$  & $\blacktriangle$  & $\blacktriangle$ \\
\textit{F6} & $\blacktriangle$  & $\blacktriangle$  & $\blacktriangle$ \\
\textit{F7} & $\blacktriangle$  & $\blacktriangle$  & $\blacktriangle$\\
\textit{F8} & $\blacktriangle$  & $\blacktriangle$  & $\blacktriangle$\\
\textit{F9} & $\blacktriangle$  & $\triangledown$  & $\blacktriangle$ \\
\textit{F10}& $\blacktriangle$  & $\triangledown$  & $\blacktriangle$\\
\textit{F11}& $\blacktriangle$  & $\blacktriangle$  & $\blacktriangle$\\
\textit{F12}& $\blacktriangle$  & $\triangledown$  & $\blacktriangle$\\
\textit{F13}& $\blacktriangle$  & $\blacktriangle$  & $\blacktriangle$\\
\textit{F14}& $\blacktriangle$  & $\triangledown$  & $\blacktriangle$ \\
\textit{F15}& $\blacktriangle$  & $\triangledown$  & $\blacktriangle$\\
\textit{F16}& $\blacktriangle$  & $\blacktriangle$  & $\blacktriangle$\\
\textit{F17}& $\blacktriangle$  & $\triangledown$  & $\blacktriangle$\\
\textit{F18}& $\blacktriangle$  & -  & $\blacktriangle$ \\
\textit{F19}& $\blacktriangle$  & $\triangledown$  & $\blacktriangle$  \\
\textit{F20}& $\blacktriangle$  & $\blacktriangle$  & $\blacktriangle$  \\
\hline
\end{tabular}
\end{center}
\end{table}

One can observe with this results, that our proposal achieved better results than PSO and ABC algorithms, but as the number of dimensions increases (\textit {e.g.} from 500 dimensions), the gain of performance of the algorithm is mitigated when compared to the APSO algorithm. %With this behavior, we proposed the use of stagnation counter, it mentioned previously. We believe that our proposal loses the exploitation capability with the increase of dimensions.

%\subsection{Analysis of the Diversity}
%This subsection describes the results of the our metric, Diversity Factor. Actually, we analyzed the behavior of the metric in different algorithms and parameters setup. We realized a comparison of diversity with ABPSO between other classical swarm intelligence algorithms, such as: PSO, FSS and ABC. As mentioned previously, the FSS algorithm presents two different ranges for the individual step ($step_ind$), the first range is called scenario 1 and second, scenario 2. In this analysis, the ABPSO algorithm presents the stagnation counter and its value is 10.0. We used an unimodal function ($F1$).
%
%The Figure \ref{fig:Diversity_ABPSO_F1} presents the results for ABPSO algorithm. The Figure \ref{fig:Diversity_ABC_F1_Factor} presents the behavior of the swarm, according to the diversity factor. One can see that this metric does not express when swarm expand with the execution of ABC. However, in the Figure \ref{fig:Diversity_ABC_F1_Radius}, we can see the variation of radius value of the swarm during search process. The variation of radius value is because the ABC algorithm, that spread the particles by the search space. Thus, maintain the diversity of swarm. In the Figure \ref{fig:Diversity_ABC_F1_Fitness}, we can see the fitness value along of iterations and the algorithm keeps with the convergence capability.
%\begin{figure}[h]
%\centering
%\subfigure[]{\includegraphics[scale=0.42]{results/diversity/ABPSO/F1/ABPSOF1.png}\label{fig:Diversity_ABPSO_F1_Factor}}
%\hspace{1mm}
%\subfigure[]{\includegraphics[scale=0.4]{results/diversity/ABPSO/F1/Radius.png}\label{fig:Diversity_ABPSO_F1_Radius}}
%\hspace{1mm}
%\subfigure[]{\includegraphics[scale=0.42]{results//diversity/ABPSO/F1/Fitness.png}\label{fig:Diversity_ABPSO_F1_Fitness}}
%\caption{\small{The evaluation of diversity of swarm for ABPSO algorithm: (a) diversity factor value, (b) radius value and (c) fitness value.}}
%\label{fig:Diversity_ABPSO_F1}
%\end{figure}
%
%The Figure \ref{fig:Diversity_ABC_F1} presents the results for ABC algorithm. The Figure \ref{fig:Diversity_ABC_F1_Factor} presents the behavior of the swarm, according to the diversity factor. One can see a expressive diversity generated by scout bees, when they find new food sources. In the Figure \ref{fig:Diversity_ABC_F1_Radius}, there is a variation of radius value of the swarm, but the radius value is high, so the swarm does not have a strong convergence capability. As explained before, the variation of radius value is because the scout bees. In the Figure \ref{fig:Diversity_ABC_F1_Fitness}, we can see the fitness value along of iterations and the limited convergence ability of algorithm, mainly if compared with the result of ABPSO algorithm.
%
%\begin{figure}[h]
%\centering
%\subfigure[]{\includegraphics[scale=0.45]{results/diversity/ABC/F1/ABCF1.png}\label{fig:Diversity_ABC_F1_Factor}}
%\hspace{1mm}
%\subfigure[]{\includegraphics[scale=0.4]{results/diversity/ABC/F1/Radius.png}\label{fig:Diversity_ABC_F1_Radius}}
%\hspace{1mm}
%\subfigure[]{\includegraphics[scale=0.4]{results//diversity/ABC/F1/Fitness.png}\label{fig:Diversity_ABC_F1_Fitness}}
%\caption{\small{The evaluation of diversity of swarm for ABC algorithm: (a) diversity factor value, (b) radius value and (c) fitness value.}}
%\label{fig:Diversity_ABC_F1}
%\end{figure}
%
%The Figure \ref{fig:Diversity_PSO_F1} presents the results for PSO algorithm. The Figure \ref{fig:Diversity_PSO_F1_Factor} presents the behavior of the swarm, according to the diversity factor. One cannot see a significant diversity, because the algorithm does not have a mechanism to generate diversity, but the local communication topology helps to avoid a premature convergence of swarm. In the Figure \ref{fig:Diversity_PSO_F1_Radius}, there is a variation of radius value of the swarm, one can observe that the variation is low and the values is decreasing, this means that the swarm is contracting. In the Figure \ref{fig:Diversity_PSO_F1_Fitness}, we can see the fitness value along of iterations. Because of contraction state of swarm the algorithm has a good convergence ability.
%\begin{figure}[h]
%\centering
%\subfigure[]{\includegraphics[scale=0.45]{results/diversity/PSO/F1/PSOF1.png}\label{fig:Diversity_PSO_F1_Factor}}
%\hspace{1mm}
%\subfigure[]{\includegraphics[scale=0.4]{results/diversity/PSO/F1/Radius.png}\label{fig:Diversity_PSO_F1_Radius}}
%\hspace{1mm}
%\subfigure[]{\includegraphics[scale=0.42]{results//diversity/PSO/F1/Fitness.png}\label{fig:Diversity_PSO_F1_Fitness}}
%\caption{\small{The evaluation of diversity of swarm for PSO algorithm: (a) diversity factor value, (b) radius value and (c) fitness value.}}
%\label{fig:Diversity_PSO_F1}
%\end{figure}
%
%The Figure \ref{fig:Diversity_FSS1_F1} presents the results for FSS algorithm, scenario 1. The Figure \ref{fig:Diversity_FSS1_F1_Factor} presents the behavior of the swarm, according to the diversity factor. One can observe the maintain of diversity, through the execution of collective-volitive movement that expand the school.  In the Figure \ref{fig:Diversity_FSS1_F1_Radius}, there is a variation of radius value of the swarm. We can see the contraction of swarm because the radius value decreased, however the variation is small because the volitive step value is small (this value is the double of individual step), consequently there is not expansion. In the Figure \ref{fig:Diversity_FSS1_F1_Fitness}, we can see the fitness value along of iterations. The maintain of diversity has consequences, the algorithm does not have a good convergence ability.
%\begin{figure}[h]
%\centering
%\subfigure[]{\includegraphics[scale=0.45]{results/diversity/FSS1/F1/FSSF1.png}\label{fig:Diversity_FSS1_F1_Factor}}
%\hspace{1mm}
%\subfigure[]{\includegraphics[scale=0.4]{results/diversity/FSS1/F1/Radius.png}\label{fig:Diversity_FSS1_F1_Radius}}
%\hspace{1mm}
%\subfigure[]{\includegraphics[scale=0.4]{results//diversity/FSS1/F1/Fitness.png}\label{fig:Diversity_FSS1_F1_Fitness}}
%\caption{\small{The evaluation of diversity of swarm for FSS algorithm scenario 1: (a) diversity factor value, (b) radius value and (c) fitness value.}}
%\label{fig:Diversity_FSS1_F1}
%\end{figure}
%
%The Figure \ref{fig:Diversity_FSS2_F1} presents the results for FSS algorithm, scenario 2. The Figure \ref{fig:Diversity_FSS2_F1_Factor} presents the behavior of the swarm, according to the diversity factor. One can observe the maintain of diversity, through the execution of collective-volitive movement that expand the school, but with the variation greater than scenario 1.  In the Figure \ref{fig:Diversity_FSS2_F1_Radius}, there is a variation of radius value of the swarm. We can see the contraction of swarm because the radius value decreased. As this scenario has the range values greater than scenario 1, there is a significant variation of radius value, but until a determined number of iterations. In the Figure \ref{fig:Diversity_FSS2_F1_Fitness}, we can see the fitness value along of iterations. Differently of the scenario 1, a big individual step value improve the convergence ability. When we compare the results of scenario 1 with scenario 2, the convergence speed of scenario 2 is better.
%\begin{figure}[h]
%\centering
%\subfigure[]{\includegraphics[scale=0.45]{results/diversity/FSS2/F1/FSSF1.png}\label{fig:Diversity_FSS2_F1_Factor}}
%\hspace{1mm}
%\subfigure[]{\includegraphics[scale=0.4]{results/diversity/FSS2/F1/Radius.png}\label{fig:Diversity_FSS2_F1_Radius}}
%\hspace{1mm}
%\subfigure[]{\includegraphics[scale=0.4]{results//diversity/FSS2/F1/Fitness.png}\label{fig:Diversity_FSS2_F1_Fitness}}
%\caption{\small{The evaluation of diversity of swarm for FSS algorithm scenario 2: (a) diversity factor value, (b) radius value and (c) fitness value.}}
%\label{fig:Diversity_FSS2_F1}
%\end{figure}

\section{Initial Analysis of the ClanABeePSO Algorithm}
This initial analysis shows the comparison of performance of ClanABeePSO with the ClanAPSO and ClanPSO algorithms. We used 100 dimensions and 1,000 iterations. This analysis was realized in five functions, the unimodal functions $F4$, $F7$ and $F19$, and the multimodal functions $F2$ and $F16$. We selected these functions in order to represent all classes of problems described in the entire benchmark.

In the conference of leaders, we executed the PSO algorithm and we tested two communication topologies, local and global. We adopted the following nomenclature for the ClanABeePSO based algorithms: ``$<$Name of algorithm$>$ - L'' means the local communication topology  and ``$<$Name of algorithm$>$- G'', global topology.

Table \ref{tab:ClanABeePSO_Local} presents the average fitness and (standard deviation) for local topology. One can observe that, in the most of cases, our proposal achieved better results than other techniques (the exception is $F2$ function). According to Table \ref{tab:ClanABeePSO_Global}, we can conclude the same for the global topology. One can observe that the local topology achieved the best results in most of cases when compared to the global topology. %But, the different is small, so it is necessary to make a deeper analysis, with statistical tests.


\begin{table}[!h]
\caption{\small{Comparison of ClanABeePSO, ClanAPSO and ClanPSO algorithms with conference of leaders executing a Local PSO.}}
\centering
\begin{tabular}{p{0.5cm}|p{3.0cm}|p{3.0cm}|p{3.0cm}}
\hline
\textbf{F\#}  & ClanABeePSO - L &  ClanAPSO - L & ClanPSO - L \\
\hline
$F2$      &  1031.6137 (56.0135) &  \textbf{771.6618 (101.8197)}   &  1249.418 (61.9373) \\
$F4$      &  \textbf{1.56E+14 (4.1E+13)}  &  2.66E+14 (1.04E+04)   &  4.97E+14 (1.62E+14) \\
$F7$      &  \textbf{3.99E+10 (1.06E+10)} &  7.34E+10 (1.44E+10)   &  8.87E+10 (1.86E+10) \\
$F16$     &  \textbf{30.1947 (1.5911)}    &  39.5529  (0.8987)     &  40.6395 (0.27026) \\
$F19$     &  \textbf{2.18E+05 (2.85E+04)} &  3.03E+05 (6.91E+04)   &  3.56E+05 (6.01E+04)\\
\hline
\end{tabular}
\label{tab:ClanABeePSO_Local}
\end{table}

\begin{table}[!h]
\caption{\small{Comparison of ClanABeePSO, ClanAPSO and ClanPSO algorithms with conference of leaders executing a global PSO.}}
\centering
\begin{tabular}{p{0.5cm}|p{3.0cm}|p{3.0cm}|p{3.0cm}}
\hline
\textbf{F\#}  & ClanABeePSO - G &  ClanAPSO - G & ClanPSO - G\\
\hline
$F2$      &   1026.3147 (58.1114)   &  \textbf{838.1519 (71.2261)}   & 1291.227 (64.2355) \\
$F4$      &   \textbf{1.65E+04 (3.54E+13)}   &  2.87E+14 (1.00E+14)  & 4.99E+14 (1.94E+14) \\
$F7$      &   \textbf{3.85E+10 (6.81E+09)}   &  8.58E+10 (1.82E+10)    & 9.61E+10 (2.11E+10) \\
$F16$     &   \textbf{30.3466 (1.5815)}      &  39.6915 (0.6011)     &  40.7952 (0.3673)\\
$F19$     &   \textbf{2.29E+05 (3.57E+04)}   &  3.32E+05 (7.83E+04)  & 3.39E+05 (8.67E+04) \\
\hline
\end{tabular}
\label{tab:ClanABeePSO_Global}
\end{table}

We also performed an analysis executing the APSO algorithm in the conference of leaders. We tested in all benchmark functions.
Table \ref{tab:Comparison_ClanABeePSO_PSO} presents the results with the use the PSO algorithm in conference of leaders. The results for the PSO algorithm used in the conference of leaders are better than when the APSO algorithm is used (observe table \ref{tab:Comparison_ClanABeePSO_APSO}). For the unimodal functions, the ClanABeePSO achieved better results than the ClanAPSO. For the multimodal functions, the ClanABeePSO algorithm did not achieved the desired performance for the Rastrigin-based functions ($F2$, $F5$, $F10$ and $F15$).

\begin{table}[!h]
\caption{\small{Performance comparison in terms of average fitness value and (standard deviation) between ClanABeePSO and ClaAPSO with execution of PSO in the conference of leaders for all the 20 benchmark functions for 100 dimensions.}}
\label{tab:Comparison_ClanABeePSO_PSO}
\begin{center}
\begin{tabular}{p{0.5cm}|p{2.5cm}|p{2.5cm}|p{2.5cm}|p{2.5cm}}
\hline\noalign{\smallskip}
\textbf{F\#}	& ClanABeePSO - L & ClanABeePSO - G & ClanAPSO - L & ClanAPSO - G    \\		
\noalign{\smallskip}
\hline
\noalign{\smallskip}
\textit{F1}  & 5.45E+08 (1.08E+08)& \textbf{5.29E+08 (1.33E+08)} & 2.61E+09 (1.14E+09) & 2.42E+09 (1.03E+09)\\
\textit{F2}  & 1031.6137 (56.0135) & 1026.3147 (58.1114) & \textbf{771.6618 (101.8197)} & 838.1519 (71.2261)\\
\textit{F3}  & 13.1624 (0.5596) & \textbf{12.5989 (0.4668)} & 19.0352 (0.6809) & 19.3434 (0.5242)\\
\textit{F4}  & \textbf{1.56E+14 (4.1E+13)} & 1.65E+14 (3.54E+13) & 2.66E+14 (1.04E+14) & 2.87E+14 (1.00E+14)\\
\textit{F5}  & 4.2E+08 (2.35E+07) & 4.17E+08 (2.48E+07) & \textbf{3.50E+08 (4.72E+07)} & 3.92E+08 (4.79E+07)\\
\textit{F6}  & 1.04E+07 (8.58E+05) & \textbf{9.41E+06 (5.57E+05)} & 1.87E+07 (9.67E+05) & 1.89E+07 (8.04E+05)\\
\textit{F7}  & 3.99E+10 (1.06E+10) & \textbf{3.85E+10 (6.81E+09)} & 7.34E+10 (1.44E+10) & 8.58E+10 (1.82E+10)\\
\textit{F8}  & 1.14E+14 (7.28E+13) & \textbf{7.22E+13 (2.33E+13)} & 2.09E+16 (9.78E+15) & 2.32E+16 (1.05E+16)\\
\textit{F9}  & 7.62E+08 (2.51E+08) & \textbf{5.8E+08 (1.07E+08)} & 1.11E+09 (5.34E+08) & 1.64E+09 (5.92E+08)\\
\textit{F10} & 1050.3797 (52.3948) & 1021.8675 (51.4933) & \textbf{704.9859 (87.0795)} & 930.2839 (83.4512)\\
\textit{F11} & 29.8896 (1.8033) & \textbf{29.5806 (1.6640)} & 37.6465 (1.3326) & 38.5982 (0.8247)\\
\textit{F12} & \textbf{7.77E+04 (9.97E+03)} & 1.04E+05 (2.09E+04) & 2.59E+05 (4.19E+04) & 2.73E+05 (4.84E+04)\\
\textit{F13} & \textbf{8.12E+07 (2.12E+07)} & 1.29E+08 (1.09E+08) & 1.61E+10 (5.28E+09) & 1.58E+10 (6.89E+09)\\
\textit{F14} & \textbf{8.09E+08 (1.3E+08)} & 8.52E+08 (1.45E+08) & 1.14E+09 (3.50E+08) & 1.18E+09 (3.73E+08)\\
\textit{F15} & 1070.2954 (48.2117) & 1067.4445 (53.4160) & \textbf{988.4879 (79.7370)} & 1012.5407 (79.9971)\\
\textit{F16} & \textbf{30.1947 (1.5911)} & 30.3466 (1.5815) & 39.5529 (0.8987) & 39.6915 (0.6011)\\
\textit{F17} & \textbf{1.91E+05 (2.50E+04)} & 2.08E+05 (3.21E+04) & 2.65E+05 (5.46E+04) & 2.70E+05 (4.30E+04)\\
\textit{F18} & \textbf{8.92E+08 (1.9E+08)} & 1.25E+09 (5.44E+08) & 7.83E+10 (1.65E+10) & 7.61E+10 (1.46E+10)\\
\textit{F19} & \textbf{2.18E+05 (2.85E+04)} & 2.29E+05 (3.57E+04) & 3.03E+05 (6.91E+04) & 3.32E+05 (7.83E+04)\\
\textit{F20} & \textbf{7.07E+08 (1.59E+08)} & 1.19E+09 (6.34E+08) & 1.04E+11 (2.04E+10) & 1.00E+11 (2.19E+10)\\
\hline
\end{tabular}
\end{center}
\end{table}

\begin{table}[!h]
\caption{\small{Performance comparison in terms of average fitness value and (standard deviation) between ClanABeePSO and ClaAPSO with execution of APSO in the conference of leaders for all the 20 benchmark functions for 100 dimensions.}}
\label{tab:Comparison_ClanABeePSO_APSO}
\begin{center}
\begin{tabular}{p{0.5cm}|p{2.5cm}|p{2.5cm}|p{2.5cm}|p{2.5cm}}
\hline\noalign{\smallskip}
\textbf{F\#}	& ClanABeePSO - L & ClanABeePSO - G & ClanAPSO - L & ClanAPSO - G \\		
\noalign{\smallskip}
\hline
\noalign{\smallskip}
\textit{F1}  & \textbf{2.14E+09 (8.18E+08)} & 2.21E+09 (8.77E+08) & 3.56E+09 (1.38E+09) & 3.33E+09 (1.29E+09)\\
\textit{F2}  & \textbf{818.9345 (62.1110)} & 854.3629 (72.2428) & 833.5397 (80.1530) & 845.7962 (88.5562)\\
\textit{F3}  & 19.5079 (0.3109) & 19.5476 (0.2776) & \textbf{19.4254 (0.3693)} & 19.4973 (0.3233)\\
\textit{F4}  & \textbf{2.21E+14 (9.49E+13)} & 2.5E+14 (9.76E+13) & 3.84E+14 (1.52E+14) & 4.12E+14 (1.79E+14)\\
\textit{F5}  & \textbf{3.85E+08 (5.14E+07)} & 3.97E+08 (5.20E+07) & 4.14E+08 (5.72E+07) & 4.11E+08 (5.47E+07)\\
\textit{F6}  & \textbf{1.94E+07 (3.89E+05)} & 1.95E+07 (3.28E+05) & 1.95E+07 (3.34E+05) & 1.96E+07 (3.54E+05)\\
\textit{F7}  & \textbf{7.6E+10 (1.63E+10)} & 8.07E+10 (1.68E+10) & 9.69E+10 (1.97E+10) & 1.01E+11 (2.69E+10)\\
\textit{F8}  & 2.16E+16 (9.48E+15) & \textbf{2.14E+16 (8.29E+15)} & 3.14E+16 (1.21E+16) & 2.93E+16 (8.79E+15)\\
\textit{F9}  & 1.58E+09 (5.22E+08) & \textbf{1.53E+09 (4.9E+08)} & 2.82E+09 (1.08E+09) & 2.47E+09 (6.68E+08)\\
\textit{F10} & 914.0123 (105.0599) & 936.1181 (95.7550) & \textbf{887.2911 (113.8297)} & 928.4704 (82.8355)\\
\textit{F11} & 38.8131 (0.8369) & 38.7006 (0.7980) & \textbf{38.3081 (1.0786)} & 38.7691 (0.7347)\\
\textit{F12} & \textbf{2.51E+05 (3.72E+04)} & 2.59E+05 (4.30E+04) & 3.14E+05 (4.55E+04) & 3.48E+05 (5.33E+04)\\
\textit{F13} & 1.38E+10 (5.15E+09) & \textbf{1.27E+10 (4.36E+09)} & 1.95E+10 (6.57E+09) & 1.94E+10 (5.98E+09)\\
\textit{F14} & \textbf{1.09E+09 (3.73E+08)} & 1.24E+09 (3.48E+08) & 1.28E+09 (4.40E+08) & 1.54E+09 (4.73E+08)\\
\textit{F15} & 1026.1148 (67.3962) & 1043.2662 (90.3853) & \textbf{1019.0113 (70.6286)} & 1025.8771 (91.8249)\\
\textit{F16} & 39.8903 (0.3489) & \textbf{39.6015 (0.3553)} & 39.7034 (0.3414) & 39.7101 (0.4693)\\
\textit{F17} & 2.72E+05 (4.79E+04) & \textbf{2.62E+05 (4.61E+04)} & 3.26E+05 (6.58E+04) & 3.19E+05 (7.37E+04)\\
\textit{F18} & 7.49E+10 (1.36E+10) & \textbf{7.19E+10 (1.37E+10)} & 8.24E+10 (1.46E+10) & 8.47E+10 (1.27E+10)\\
\textit{F19} & \textbf{3.02E+05 (6.55E+04)} & 3.12E+05 (6.05E+04) & 3.75E+05 (9.05E+04) & 3.82E+05 (8.33E+04)\\
\textit{F20} & 1.08E+11 (2.14E+10) & \textbf{1.02E+11 (1.82E+10)} & 1.19E+11 (2.23E+10) & 1.19E+11 (1.85E+10)\\
\hline
\end{tabular}
\end{center}
\end{table}

%\begin{table}[!h]
%\caption{\small{Performance comparison in terms of average fitness value and (standard deviation) between ClanABPSO and ClaAPSO with execution of PSO }}
%%in the conference of leaders for all the 20 benchmark functions for 100 dimensions.
%\label{tab:Comparison_ClanABPSO_PSO}
%\begin{center}
%\begin{tabular}{p{0.5cm}|p{2.1cm}|p{2.1cm}|p{2.1cm}|p{2.1cm}|p{2.1cm}}
%\hline\noalign{\smallskip}
%\textbf{F\#}	& ABPSO & ClanABPSO - L & ClanABPSO - G & ClanAPSO - L & ClanAPSO - G    \\		
%\noalign{\smallskip}
%\hline
%\noalign{\smallskip}
%\textit{F1}  & \textbf{2.86E+08 (8.22E+07)} & 5.45E+08 (1.08E+08)& 5.29E+08 (1.33E+08) & 2.61E+09 (1.14E+09) & 2.42E+09 (1.03E+09)\\
%\textit{F2}  & 958.6346 (63.09667) & 1031.6137 (56.0135) & 1026.3147 (58.1114) & \textbf{771.6618 (101.8197)} & 838.1519 (71.2261)\\
%\textit{F3}  & \textbf{12.07975 (0.656529)} & 13.1624 (0.5596) & 12.5989 (0.4668) & 19.0352 (0.6809) & 19.3434 (0.5242)\\
%\textit{F4}  & \textbf{1.32E+14 (3.05E+13)} & 1.56E+14 (4.1E+13) & 1.65E+14 (3.54E+13) & 2.66E+14 (1.04E+14) & 2.87E+14 (1.00E+14)\\
%\textit{F5}  & 3.81E+08 (2.64E+07) & 4.2E+08 (2.35E+07) & 4.17E+08 (2.48E+07) & \textbf{3.50E+08 (4.72E+07)} & 3.92E+08 (4.79E+07)\\
%\textit{F6}  & \textbf{7.59E+06 (8.71E+05)} & 1.04E+07 (8.58E+05) & 9.41E+06 (5.57E+05) & 1.87E+07 (9.67E+05) & 1.89E+07 (8.04E+05)\\
%\textit{F7}  & 4.33E+10 (9.10E+09) & 3.99E+10 (1.06E+10) & \textbf{3.85E+10 (6.81E+09)} & 7.34E+10 (1.44E+10) & 8.58E+10 (1.82E+10)\\
%\textit{F8}  & \textbf{6.78E+13 (5.49E+05)} & 1.14E+14 (7.28E+13) & 7.22E+13 (2.33E+13) & 2.09E+16 (9.78E+15) & 2.32E+16 (1.05E+16)\\
%\textit{F9}  & \textbf{4.02E+08 (9.68E+07)} & 7.62E+08 (2.51E+08) & 5.8E+08 (1.07E+08) & 1.11E+09 (5.34E+08) & 1.64E+09 (5.92E+08)\\
%\textit{F10} & 973.3662 (39.8397) & 1050.3797 (52.3948) & 1021.8675 (51.4933) & \textbf{704.9859 (87.0795)} & 930.2839 (83.4512)\\
%\textit{F11} & \textbf{25.93234 (2.34576)} & 29.8896 (1.8033) & 29.5806 (1.6640) & 37.6465 (1.3326) & 38.5982 (0.8247)\\
%\textit{F12} & 9.26E+04 (1.67E+04) & \textbf{7.77E+04 (9.97E+03)} & 1.04E+05 (2.09E+04) & 2.59E+05 (4.19E+04) & 2.73E+05 (4.84E+04)\\
%\textit{F13} & \textbf{8.08E+07 (7.19E+07)} & 8.12E+07 (2.12E+07) & 1.29E+08 (1.09E+08) & 1.61E+10 (5.28E+09) & 1.58E+10 (6.89E+09)\\
%\textit{F14} & \textbf{6.29E+08 (1.21E+08)} & 8.09E+08 (1.3E+08) & 8.52E+08 (1.45E+08) & 1.14E+09 (3.50E+08) & 1.18E+09 (3.73E+08)\\
%\textit{F15} & \textbf{982.0151 (40.05412)} & 1070.2954 (48.2117) & 1067.4445 (53.4160) & 988.4879 (79.7370) & 1012.5407 (79.9971)\\
%\textit{F16} & \textbf{25.48114 (2.66121)} & 30.1947 (1.5911) & 30.3466 (1.5815) & 39.5529 (0.8987) & 39.6915 (0.6011)\\
%\textit{F17} & \textbf{1.84E+05 (2.29E+04)} & 1.91E+05 (2.50E+04) & 2.08E+05 (3.21E+04) & 2.65E+05 (5.46E+04) & 2.70E+05 (4.30E+04)\\
%\textit{F18} & 1.02E+09 (6.82E+08) & \textbf{8.92E+08 (1.9E+08)} & 1.25E+09 (5.44E+08) & 7.83E+10 (1.65E+10) & 7.61E+10 (1.46E+10)\\
%\textit{F19} & \textbf{2.09E+05 (2.82E+04)} & 2.18E+05 (2.85E+04) & 2.29E+05 (3.57E+04) & 3.03E+05 (6.91E+04) & 3.32E+05 (7.83E+04)\\
%\textit{F20} & 1.09E+09 (6.41E+08) & \textbf{7.07E+08 (1.59E+08)} & 1.19E+09 (6.34E+08) & 1.04E+11 (2.04E+10) & 1.00E+11 (2.19E+10)\\
%\hline
%\end{tabular}
%\end{center}
%\end{table}
%
%\begin{table}[!h]
%\caption{\small{Performance comparison in terms of average fitness value and (standard deviation) between ClanABPSO and ClaAPSO with execution of APSO}}
%%in the conference of leaders for all the 20 benchmark functions for 100 dimensions.
%\label{tab:Comparison_ClanABPSO_APSO}
%\begin{center}
%\begin{tabular}{p{0.5cm}|p{2.1cm}|p{2.1cm}|p{2.1cm}|p{2.1cm}|p{2.1cm}}
%\hline\noalign{\smallskip}
%\textbf{F\#}	& ABPSO & ClanABPSO - L & ClanABPSO - G & ClanAPSO - L & ClanAPSO - G \\		
%\noalign{\smallskip}
%\hline
%\noalign{\smallskip}
%\textit{F1} & \textbf{2.86E+08 (8.22E+07)}  & 2.14E+09 (8.18E+08) & 2.21E+09 (8.77E+08) & 3.56E+09 (1.38E+09) & 3.33E+09 (1.29E+09)\\
%\textit{F2} & 958.6346 (63.09667) & \textbf{818.9345 (62.1110)} & 854.3629 (72.2428) & 833.5397 (80.1530) & 845.7962 (88.5562)\\
%\textit{F3} & \textbf{12.07975 (0.656529)} & 19.5079 (0.3109) & 19.5476 (0.2776) & 19.4254 (0.3693) & 19.4973 (0.3233)\\
%\textit{F4} & \textbf{1.32E+14 (3.05E+13)} & 2.21E+14 (9.49E+13) & 2.5E+14 (9.76E+13) & 3.84E+14 (1.52E+14) & 4.12E+14 (1.79E+14)\\
%\textit{F5} & \textbf{3.81E+08 (2.64E+07)} & 3.85E+08 (5.14E+07) & 3.97E+08 (5.20E+07) & 4.14E+08 (5.72E+07) & 4.11E+08 (5.47E+07)\\
%\textit{F6}  & \textbf{7.59E+06 (8.71E+05)} & 1.94E+07 (3.89E+05) & 1.95E+07 (3.28E+05) & 1.95E+07 (3.34E+05) & 1.96E+07 (3.54E+05)\\
%\textit{F7} & \textbf{4.33E+10 (9.10E+09)} & 7.6E+10 (1.63E+10) & 8.07E+10 (1.68E+10) & 9.69E+10 (1.97E+10) & 1.01E+11 (2.69E+10)\\
%\textit{F8}  & \textbf{6.78E+13 (5.49E+05)} & 2.16E+16 (9.48E+15) & 2.14E+16 (8.29E+15) & 3.14E+16 (1.21E+16) & 2.93E+16 (8.79E+15)\\
%\textit{F9} & \textbf{4.02E+08 (9.68E+07)} & 1.58E+09 (5.22E+08) & 1.53E+09 (4.9E+08) & 2.82E+09 (1.08E+09) & 2.47E+09 (6.68E+08)\\
%\textit{F10} & 973.3662 (39.8397) & 914.0123 (105.0599) & 936.1181 (95.7550) & \textbf{887.2911 (113.8297)} & 928.4704 (82.8355)\\
%\textit{F11} & \textbf{25.93234 (2.34576)} & 38.8131 (0.8369) & 38.7006 (0.7980) & 38.3081 (1.0786) & 38.7691 (0.7347)\\
%\textit{F12} & \textbf{9.26E+04 (1.67E+04)} & 2.51E+05 (3.72E+04) & 2.59E+05 (4.30E+04) & 3.14E+05 (4.55E+04) & 3.48E+05 (5.33E+04)\\
%\textit{F13} & \textbf{8.08E+07 (7.19E+07)} & 1.38E+10 (5.15E+09) & 1.27E+10 (4.36E+09) & 1.95E+10 (6.57E+09) & 1.94E+10 (5.98E+09)\\
%\textit{F14} & \textbf{6.29E+08 (1.21E+08)} & 1.09E+09 (3.73E+08) & 1.24E+09 (3.48E+08) & 1.28E+09 (4.40E+08) & 1.54E+09 (4.73E+08)\\
%\textit{F15} & \textbf{982.0151 (40.05412)} & 1026.1148 (67.3962) & 1043.2662 (90.3853) & 1019.0113 (70.6286) & 1025.8771 (91.8249)\\
%\textit{F16} & \textbf{25.48114 (2.66121)} & 39.8903 (0.3489) & 39.6015 (0.3553) & 39.7034 (0.3414) & 39.7101 (0.4693)\\
%\textit{F17} & \textbf{1.84E+05 (2.29E+04)} & 2.72E+05 (4.79E+04) & 2.62E+05 (4.61E+04) & 3.26E+05 (6.58E+04) & 3.19E+05 (7.37E+04)\\
%\textit{F18} & \textbf{1.02E+09 (6.82E+08)} & 7.49E+10 (1.36E+10) & 7.19E+10 (1.37E+10) & 8.24E+10 (1.46E+10) & 8.47E+10 (1.27E+10)\\
%\textit{F19} & \textbf{2.09E+05 (2.82E+04)} & 3.02E+05 (6.55E+04) & 3.12E+05 (6.05E+04) & 3.75E+05 (9.05E+04) & 3.82E+05 (8.33E+04)\\
%\textit{F20} & \textbf{1.09E+09 (6.41E+08)} & 1.08E+11 (2.14E+10) & 1.02E+11 (1.82E+10) & 1.19E+11 (2.23E+10) & 1.19E+11 (1.85E+10)\\
%\hline
%\end{tabular}
%\end{center}
%\end{table}

\pagebreak

%  \chapter{Case Study}\label{cap:casestudy}

In this chapter we present a real case study undertaken with a real software project. The project consists of the development of a middle size system to automate computers functional tests in a manufacturing production line.

\section{Project Scope and Schedule}

\section{Project Risk Management}

\subsection{Phase 1: Choosing between develop or customize a COTS}

\subsection{Phase 2: Deploying the COTS system}

\subsection{Phase 3: Including essential functionalities }

\subsection{Phase 4: Performing Acceptance Test}

\subsection{Phase 5: Including additional functionalities}

\subsection{Phase 6: Performing first statistical experiment to verify reliability}

\subsection{Phase 7: Optimizing network infrastructure}

\subsection{Phase 8: Optimizing server infrastructure}

\subsection{Phase 9: Optimizing web system}

\subsection{Phase 10: Performing second statistical experiment to verify reliability, time performance and resources consumption}

\subsection{Phase 11: Deploying the first release to production}

\pagebreak

  \chapter{Conclusões e Trabalhos Futuros}\label{cap:conclusion}

Este trabalho investigou o uso de redes neurais artificiais, como algoritmos SVM e MLP, para a estimativa do impacto do risco, através da proposta de uma metodologia para a análise de risco em projetos de \textit{software}, a partir de dados históricos de registros de riscos. 

Os resultados mostraram que as redes neurais artificiais apresentaram resultados promissores, principalmente porque um modelo MLP chamado ``MLPRegressor" obteve os menores erros de previsão. Além disso, alguns resultados apresentaram informações importantes: a SMC apresentou resultados piores comparado com o modelo de linha de base, o MRLM; o modelo ANFIS apresentou resultados piores comparado com a "MLPRegressor", porém é mais preciso 52\% na média comparado com o último modelo; é difícil melhorar ainda mais o desempenho da Análise PERT e da SMC já que todas as medidas foram tomadas para que com as informações disponíveis esses modelos pudessem apresentar os melhores resultados.

Além disso, uma pesquisa rápida realizada com duzentos profissionais (membros da comunidade de prática de gerenciamento de risco do PMI, dos mais diversos campos de aplicação e setores), os quais dez responderam às perguntas feitas, informaram que, no geral, entre 0\% e 5\% é um intervalo de erros ideal de previsão de impacto de risco; 5\% e 10\% é um intervalo aceitável de erro na estimativa e 10\% e 15\% é um intervalo indesejado. Portanto, como o ``MLPRegressor" apresentou alguns valores no primeiro intervalo, REMQ médio de 0.05168 ou 5,168\%, é possível afirmar que esse modelo apresenta resultados satisfatórios.

% Listar os problemas e dificuldades encontradas
Algumas dificuldades foram encontradas nesse estudo, principalmente quanto ao pré-processamento dos dados e a seleção das redes neurais e suas variações para se obter os resultados desejados. Porque não há qualquer trabalho científico utilizando a PERIL para se tomar como base, segundo a revisão bibliográfica feita.

Alguns resultados interessantes foram alcançados porém é preciso melhorar algumas técnicas utilizadas para que os resultados possam ser mais precisos e menores erros de previsão sejam obtidos. Além disso, há uma carga computacional elevada no procedimento de utilização de um algoritmo de otimização para otimizar o desempenho das redes neurais, o que demanda tempo. Como atividades futuras, foram identificadas:
\begin{itemize}
\item Realizar a validação prática dessa metodologia com informações reais para a estimativa e acompanhamento de riscos;
\item Desenvolver uma abordagem eficiente e precisa para quando houver poucas informações sobre os riscos identificados num projeto e nenhum registro de riscos em projetos similares anteriores;
\item Desenvolver uma abordagem inovadora e mais eficiente para a análise qualitativa dos riscos, baseada na classificação da natureza dos riscos;
\item Desenvolver uma técnica para a avaliação de estratégias de mitigação de risco, baseadas no impacto se o risco ocorrer, no esforço para mitigação de risco, nas interações entre os fatores de risco e nos recursos disponíveis para os planos de mitigação;
\item Desenvolver uma metodologia para a avaliação qualitativa, a avaliação quantitativa e planos de contingência de riscos em projetos, do ponto de vista do gerenciamento de portfólio de projetos para o alcance de objetivos estratégicos.
\item Desenvolvimento de um \textit{canvas} para o gerenciamento de riscos em projetos de forma ágil;
\item Implementar outros métodos de validação cruzada.
\end{itemize}

\pagebreak


%  \chapter*{Apêndice A}


\textbf{Publicações}
\\
\\


\textbf{Título:} Assessing Particle Swarm Optimizers Using Network Science Metrics.

\textbf{Autores:} Marcos A. C. Oliveira-Junior, Carmelo J. A. Bastos-Filho, Ronaldo Menezes.

\textbf{Publicado em:} Springer Complex Networks IV -- Studies in Computational Intelligence Volume 476, 2013, pp 173-184. Proceedings of the 4th Workshop on Complex Networks CompleNet 2013.

\textbf{Abstract:}Particle Swarm Optimizers (PSOs) have been widely used for optimization problems, but the scientific community still does not have sophisticated mechanisms to analyze the behavior of the swarm during the optimization process. We propose in this paper to use some metrics described in network sciences, specifically the R-value, the number of zero eigenvalues of the Laplacian Matrix, and the Spectral Density, in order to assess the behavior of the particles during the search and diagnose stagnation processes. Assessor methods can be very useful for designing novel PSOs or when one needs to evaluate the performance of a PSO variation applied to a specific problem. In order to apply these metrics, we observed that it is not possible to analyze the dynamics of the swarm by using the communication topology because it does not change. Therefore, we propose in this paper the definition of the influence graph of the swarm. We used this novel concept to assess the dynamics of the swarm. We tested our proposed methodology in three different PSOs in a well-known multimodal benchmark function. We observed that one can retrieve interesting information from the swarm by using this methodology.

\

\
\textbf{Título:} Using Network Science to Define a Dynamic Communication Topology for Particle Swarm Optimizers.

\textbf{Autores:} Marcos A. C. Oliveira-Junior, Carmelo J. A. Bastos-Filho, Ronaldo Menezes.

\textbf{Publicado em:} Springer Complex Networks -- Studies in Computational Intelligence Volume 424, 2013, pp 39-47. Proceedings of the 3rd Workshop on Complex Networks CompleNet 2012.

\textbf{Abstract:} We propose here to use network sciences, specifically an approach based on the Barabási-Albert model, to define a dynamic communication topology for Particle Swarm Optimizers. We compared our proposal to previous approaches, including a simpler Barabási-Albert-based approach and other most used approaches, and we obtained better results in average for well known benchmark functions.



\pagebreak

%  \input{chapter/AppendixD}
% Thesis Chapters [END]

% Bibliography [START]
  \bibliographystyle{unsrt}
  \bibliography{ppgec}
% Bibliography [END]
\end{document}
