\chapter{Conclusions and Future Works}\label{cap:conclusion}

This research has investigated the use of Artificial Neural Networks algorithms, like MLP, SVM and RBF, for risk impact estimation in software project risk analysis. We have carried out a statistical analysis using PERIL. The results were compared to MLRM, Monte Carlo Simulation, PERT analysis (proposed by \cite{PMBOK2008}) and ANFIS System. We have considered improving risk impact estimation accuracy during software project management, in terms of RMSE mean and standard deviation. We have observed that a MLP alternative called "MLPRegressor" had minor standard deviation estimation error, and showed to be a promissory technique. Therefore, the selected ANN algorithms outperformed both linear regression and MCS.

% Apresentar os melhores resultados obtidos.
% Apresentar as limitações desse estudo e trabalhos futuros.

Comentar sobre uma pesquisa rápida realizada com duzentos profissionais, os quais dez responderam às perguntas feitas e informaram que, no geral, entre 0\% e 5\% é o intervalo de erros ideal de previsão de impacto; 5\% e 10\% é um intervalo aceitável de erro na estimativa e 10\% e 15\% é um intervalo indesejado.

As perguntas realizadas foram:
\begin{enumerate}
\item What are the methods you utilize to risk analysis?
\item What are the advantages and disadvantages of those techniques in real-life problems?
\item Are they suitable for all your kind of projects?
\item Do you need historic risk management information (previous risk register) to risk analysis?
\item What is an acceptable estimation error percentage of risk impact? And what is considered an good estimation error percentage (10-15\%, 5\%-10\%, 0\%-5\%)?
\end{enumerate}

Portanto, resultados interessantes foram alcançados porém é preciso melhorar algumas técnicas utilizadas para que os resultados possam ser mais precisos e menores erros de previsão sejam gerados. Além disso, há uma carga computacional elevada no procedimento de utilização de um algoritmo de otimização para otimizar o desempenho das redes neurais, o que demanda tempo.

Como atividades futuras, foram identificadas.
\begin{itemize}
\item Realizar a validação prática dessa metodologia com informações reais para a estimativa e acompanhamento de riscos;
\item Desenvolver uma abordagem eficiente e precisa para quando houver poucas informações sobre os riscos identificados num projeto e nenhum registro de riscos em projetos similares anteriores;
\item Desenvolver uma abordagem inovadora e mais eficiente para a análise qualitativa dos riscos, baseada na classificação da natureza dos riscos;
\item Desenvolver uma técnica para a avaliação de estratégias de mitigação de risco, baseadas no impacto se o risco ocorrer, no esforço para mitigação de risco, nas interações entre os fatores de risco e nos recursos disponíveis para os planos de mitigação;
\item Desenvolver uma metodologia para a avaliação qualitativa, a avaliação quantitativa e planos de contingência de riscos em projetos, do ponto de vista do gerenciamento de portfólio de projetos para o alcance de objetivos estratégicos.
\item Desenvolvimento de um \textit{canvas} para o gerenciamento de riscos em projetos de forma ágil;
\end{itemize}
 
\pagebreak

