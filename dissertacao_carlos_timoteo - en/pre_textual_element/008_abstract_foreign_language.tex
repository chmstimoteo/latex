\begin{flushbottom}
\begin{flushleft}
{\huge \textbf{Abstract}}
\linebreak
%\linebreak
\end{flushleft}
%Software Project
Many software project management end in failure. 

%Risk Management. Risk Analysis
Risk analysis is an essential process to support project success. There is a growing need for systematic methods to supplement expert judgment in order to increase the accuracy in the prediction of risk likelihood and impact. 

%Artificial Neural Networks
Uma rede neural é um processador maciçamente paralelamente distribuído constituído de unidades de processamento simples, que têm a propensão natual para armazenar conhecimento experimental e torná-lo disponível para o uso.

%Purpose of this dissertation
In this dissertation, we evaluated support vector machine (SVM), multilayer perceptron (MLP), a linear regression model and monte carlo simulation to perform risk analysis based on PERIL data. We have conducted a statistical experiment to determine which is a more accurate method in risk impact estimation. 

%Results
Our experimental results showed that artificial neural network methods proposed in this study outperformed both linear regression and monte carlo simulation.
\\
\\
\textbf{Palavras-chave:} \\ Software Project, Risk Management, Artificial Neural Network, Monte Carlo Simulation, Regression Model.\end{flushbottom}
\newpage
