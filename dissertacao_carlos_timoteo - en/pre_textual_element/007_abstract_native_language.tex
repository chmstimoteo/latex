\begin{flushbottom}
\begin{flushleft}
{\huge \textbf{Resumo}}
\linebreak
%\linebreak
\end{flushleft}
%Software Project
Muitas vezes o gerenciamento de projetos de software finalizam com falha.

%Risk Management. Risk Analysis
A Análise de riscos é um processo essencial para promover o sucesso do projeto. Existe uma necessidade crescente por métodos sistemáticos para suplementar a opinião especializada com o objetivo de aumentar a precisão na previsão do impacto e probabilidade do risco.

%Artificial Neural Networks
Uma rede neural é um processador maciçamente paralelamente distribuído constituído de unidades de processamento simples, que têm a propensão natual para armazenar conhecimento experimental e torná-lo disponível para o uso.

%Purpose of this dissertation
Nessa dissertação, nós avaliamos Máquina de Vetor de Suporte (\textit{Support Vector Machine}), \textit{Perceptron} de Múltiplas Camadas (\textit{Multilayer Perceptron}), um Modelo de Regressão Linear (\textit{Linear Regression Model}) e Simulação de Monte Carlo (\textit{Monte Carlo Simulation}) para realizar a análise de risco baseada na base de dados do PERIL. Nós conduzimos um experimento estatístico para determinar qual é um método mais preciso para a estimativa do impacto de um risco.

%Results
Our experimental results showed that artificial neural network methods proposed in this study outperformed both linear regression and monte carlo simulation.
\\
\\
\textbf{Palavras-chave:} \\ Software Project, Risk Management, Artificial Neural Network, Monte Carlo Simulation, Regression Model.\end{flushbottom}
\newpage
