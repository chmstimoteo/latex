\chapter{Conclus�o}
O algoritmo MOPSO-CDR � recente e � comumente utilizado para otimizar fun��es cl�ssicas como ZDT-1, ZDT-2, Sphere-2D. Para este trabalho realizou-se um estudo de aplica��o do MOPSO-CDR para um problema de prioriza��o de riscos. 

Observou-se que o algoritmo foi capaz de fazer as part�culas convergirem para o pareto �timo representado pelo ponto de coordenadas (1.0, 1.0). No entanto, ele necessita de adapta��es para que as part�culas n�o convirgam para o ponto �timo. Porque na pr�tica, n�o temos recursos suficientes para contigenciar os riscos/cat�strofes que apresentam os maiores valores agregados.

Portanto, solu��es sub-�timas como as apresentadas nas itera��es 20.000 e 30.000, apresentadas nas Figuras \ref{fig:result1} e \ref{fig:result2}, em algumas execu��es mostram o resultado pr�tico esperado.

Partindo desses resultados preliminares, sugere-se um estudo mais aprofundado para obten��o de melhores resultados pr�ticos.